\documentclass{article}

\usepackage[utf8]{inputenc} % set input encoding (not needed with XeLaTeX)

%%% PAGE DIMENSIONS
\usepackage{geometry} % to change the page dimensions
\geometry{letterpaper} % or letterpaper (US) or a5paper or....
\geometry{margin=1.35in} % for example, change the margins to 2 inches all round

\usepackage{graphicx} % support the \includegraphics command and options

\usepackage[parfill]{parskip} % Activate to begin paragraphs with an empty line rather than an indent

%%% PACKAGES
\usepackage{booktabs} % for much better looking tables
\usepackage{array} % for better arrays (eg matrices) in maths
\usepackage{paralist} % very flexible & customisable lists (eg. enumerate/itemize, etc.)
\usepackage{verbatim} % adds environment for commenting out blocks of text & for better verbatim
\usepackage{subfig} % make it possible to include more than one captioned figure/table in a single float
% These packages are all incorporated in the memoir class to one degree or another...

%%% HEADERS & FOOTERS
\usepackage{fancyhdr} % This should be set AFTER setting up the page geometry
\pagestyle{fancy} % options: empty , plain , fancy/
\renewcommand{\headrulewidth}{0pt} % customise the layout...
\lhead{}\chead{}\rhead{}
\lfoot{}\cfoot{\thepage}\rfoot{}

%%% SECTION TITLE APPEARANCE
\usepackage{sectsty}
%\allsectionsfont{\sffamily\mdseries\upshape} % (See the fntguide.pdf for font help)
% (This matches ConTeXt defaults)

%%% ToC (table of contents) APPEARANCE
\usepackage[nottoc,notlof,notlot]{tocbibind} % Put the bibliography in the ToC
\usepackage[titles,subfigure]{tocloft} % Alter the style of the Table of Contents
\renewcommand{\cftsecfont}{\rmfamily\mdseries\upshape}
\renewcommand{\cftsecpagefont}{\rmfamily\mdseries\upshape} % No bold!

% JPR added
\usepackage{fontawesome}
\usepackage{amsfonts}
%\usepackage{amsmath}
\usepackage{mathtools}% includes amsmath
\usepackage{changepage}
\usepackage{enumerate}
%\usepackage{setspace}
\usepackage{relsize}
\usepackage{wasysym}
%\usepackage{romannum}

\newcommand{\jprVersion}{v01} % version
 
\usepackage[pdftex,
            pdfauthor={James Philip Rowell},
            pdftitle={\jobname.\jprVersion},
            pdfsubject={The Best Number System?},
            pdfkeywords={radix, decimal, base ten, base twenty one, unvigesimal, binary, base two, integers, theorem, proof, mathematics, number theory},
            pdfproducer={Latex},
            pdfcreator={pdflatex}]{hyperref}
\hypersetup{
    colorlinks=true,
    linkcolor=black,
    filecolor=magenta,      
    urlcolor=blue,
}
\usepackage{hyperxmp}
\hypersetup{
    pdfauthor={James Philip Rowell},
    pdfcopyright={Copyright 2019 by James Philip Rowell. All rights reserved.}
}
\usepackage{lipsum}

\newenvironment{jprIn}{\begin{adjustwidth}{2em}{}}{\end{adjustwidth}}
\addtolength{\skip\footins}{6pt}

\usepackage{alphalph}
\makeatletter
\newalphalph{\fnsymbolwrap}[wrap]{\@fnsymbol}{}
\makeatother
\renewcommand*{\thefootnote}{%
  \fnsymbolwrap{\value{footnote}}%
}

\usepackage{perpage}
\usepackage[bottom]{footmisc}
\MakePerPage{footnote}

%%% END Article customizations

\title{\vspace{-1.5cm}The Best Number System}
\author{James Philip Rowell}
\date{\vspace{-0.5cm}\footnotesize\today\ (\jprVersion)} % Activate to display a given date or no date (if empty),
% otherwise the current date is printed

\begin{document}
\maketitle
\begin{em}
\centerline{\small{}What might the ``best'' number system be?}
\end{em}

If we cleared the slate and could hit the reset-button on our conventions for
counting in decimal, plus we were tasked with finding the \emph{best} number system for use,
would we pick decimal?

I know, I know, until we attempt to define what ``best'' means, it's impossible to
answer such a question, especially for a mathematically inclined reader such as yourself.
But please bear with me as we flesh this out a bit.

Let's ignore that the fact
that Sesame Street drilled into several generations of kids heads
that ``10'' means ten things. Oh, and that the most of the world 
has been using decimal for about 600 to 800 years or so; So ignoring those
insurmountable obstacles\dots{}

Then what criteria might we use to select the method by which we count
and give names to each individual integer?

I think we can discount many esoteric systems, for example mixed radix systems, as
being impractical for day-to-day use. They are more mathematical curiosities than
practical systems for everyday use.

We can also discount more ancient number systems which introduced
new symbols for larger and larger numbers (e.g. Roman Numerals) since you can't cover
all the integers with such a system.

I believe the choice boils down to which base should we choose given the...

\section*{Basis Representation Theorem}
\begin{jprIn}
Let $b$ be a positive integer greater than 1.

For every positive integer $n$ there is a unique sequence
of integers $d_0, d_1, d_2,\dots{},d_k$ such that:

\hspace{3em}$n=d_kb^k+d_{k-1}b^{k-1}+\dots+d_2b^2+d_1b^1+d_0b^0$,

where $0\le{}d_i<b$ for all $i$ in $\{0,1,2,\dots{},k\}$ and $d_k\ne0$.

Definition: $n$ is represented in base-$b$ by the string
of base-$b$-digits $(d_kd_{k-1}{\cdots}d_2d_1d_0)_b$
\end{jprIn}

Our beloved theorem doesn't say anything about the practicalities behind
using a given base. For example, it's convenient to have 
simple symbols to represent each integer from \(0, 1, \dots{}, (b-1)\) so
that any number can be written in way that's easy for us to parse.

We don't live in a world where we routinely deal with numbers up in the nose-bleed
heights. Most of us know that 1 billion is equal to \(10^9\), and probably as many
people know that 1 Trillion is \(10^12\), but it starts getting hazy beyond that.
For example, Quadrillion and Quintillion mean what?
We also have a few names for some special cases like Googol which is \(10^100\).
Anyway, the point is, we live
down near zero, not up in the heights, and we need words to help talk about our
day-to-day numbers.

The words we use a intimately tied to our choice of base-ten.

\end{document}
