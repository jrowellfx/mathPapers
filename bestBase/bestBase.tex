\documentclass{article}

\usepackage[utf8]{inputenc} % set input encoding (not needed with XeLaTeX)

%%% PAGE DIMENSIONS
\usepackage{geometry} % to change the page dimensions
\geometry{letterpaper} % or letterpaper (US) or a5paper or....
\geometry{margin=1.35in} % for example, change the margins to 2 inches all round

\usepackage{graphicx} % support the \includegraphics command and options

\usepackage[parfill]{parskip} % Activate to begin paragraphs with an empty line rather than an indent

%%% PACKAGES
\usepackage{booktabs} % for much better looking tables
\usepackage{array} % for better arrays (eg matrices) in maths
\usepackage{paralist} % very flexible & customisable lists (eg. enumerate/itemize, etc.)
\usepackage{verbatim} % adds environment for commenting out blocks of text & for better verbatim
\usepackage{subfig} % make it possible to include more than one captioned figure/table in a single float
% These packages are all incorporated in the memoir class to one degree or another...

%%% HEADERS & FOOTERS
\usepackage{fancyhdr} % This should be set AFTER setting up the page geometry
\pagestyle{fancy} % options: empty , plain , fancy/
\renewcommand{\headrulewidth}{0pt} % customise the layout...
\lhead{}\chead{}\rhead{}
\lfoot{}\cfoot{\thepage}\rfoot{}

%%% SECTION TITLE APPEARANCE
\usepackage{sectsty}
%\allsectionsfont{\sffamily\mdseries\upshape} % (See the fntguide.pdf for font help)
% (This matches ConTeXt defaults)

%%% ToC (table of contents) APPEARANCE
\usepackage[nottoc,notlof,notlot]{tocbibind} % Put the bibliography in the ToC
\usepackage[titles,subfigure]{tocloft} % Alter the style of the Table of Contents
\renewcommand{\cftsecfont}{\rmfamily\mdseries\upshape}
\renewcommand{\cftsecpagefont}{\rmfamily\mdseries\upshape} % No bold!

% JPR added
\usepackage{fontawesome}
\usepackage{amsfonts}
%\usepackage{amsmath}
\usepackage{mathtools}% includes amsmath
\usepackage{changepage}
\usepackage{enumerate}
%\usepackage{setspace}
\usepackage{relsize}
\usepackage{wasysym}
%\usepackage{romannum}

\newcommand{\jprVersion}{v01} % version
 
\usepackage[pdftex,
            pdfauthor={James Philip Rowell},
            pdftitle={\jobname.\jprVersion},
            pdfsubject={The Best Number System?},
            pdfkeywords={radix, decimal, base ten, base twenty one, unvigesimal, binary, base two, integers, theorem, proof, mathematics, number theory},
            pdfproducer={Latex},
            pdfcreator={pdflatex}]{hyperref}
\hypersetup{
    colorlinks=true,
    linkcolor=black,
    filecolor=magenta,      
    urlcolor=blue,
}
\usepackage{hyperxmp}
\hypersetup{
    pdfauthor={James Philip Rowell},
    pdfcopyright={Copyright 2019 by James Philip Rowell. All rights reserved.}
}
\usepackage{lipsum}

\newenvironment{jprIn}{\begin{adjustwidth}{2em}{}}{\end{adjustwidth}}
\addtolength{\skip\footins}{6pt}

\usepackage{alphalph}
\makeatletter
\newalphalph{\fnsymbolwrap}[wrap]{\@fnsymbol}{}
\makeatother
\renewcommand*{\thefootnote}{%
  \fnsymbolwrap{\value{footnote}}%
}

\usepackage{perpage}
\usepackage[bottom]{footmisc}
\MakePerPage{footnote}

%%% END Article customizations

\title{\vspace{-1.5cm}The Best Number System}
\author{James Philip Rowell}
\date{\vspace{-0.5cm}\footnotesize\today\ (\jprVersion)} % Activate to display a given date or no date (if empty),
% otherwise the current date is printed

\begin{document}
\maketitle
\begin{em}
\centerline{\small{}What might the ``best'' number system be?}
\end{em}

If we hit the reset button on our conventions about
how we count, and were tasked with finding the BEST number system
for how we denote individual integers,
should we pick decimal?

Let's ignore that the fact
that Sesame Street drilled into several generations of kids heads
that ``10'' means ten things. Oh, and that the entire world 
has been using it for about 600 to 800 years or so. So ignoring those
insurmountable obstacles\dots{}

Then what criteria might we use to select the method by which we count
and give names to each individual integer?

I think we can discount many esoteric systems, like mixed radix systems, as
being impractical for casual use.

We can also discount systems like more ancient number systems that introduced
new symbols for larger and larger numbers (e.g. Roman Numerals) since you can't cover
the integers with such a system.

I think the choice simply boils down to which base should we choose given the...

\section*{Basis Representation Theorem}
\begin{jprIn}
Let $b$ be a positive integer greater than 1.

For every positive integer $n$ there is a unique sequence
of integers $d_0, d_1, d_2,\dots{},d_k$ such that:

\hspace{3em}$n=d_kb^k+d_{k-1}b^{k-1}+\dots+d_2b^2+d_1b^1+d_0b^0$,

where $0\le{}d_i<b$ for all $i$ in $\{0,1,2,\dots{},k\}$ and $d_k\ne0$.

Definition: $n$ is represented in base-$b$ by the string
of base-$b$-digits $(d_kd_{k-1}{\cdots}d_2d_1d_0)_b$
\end{jprIn}


\end{document}
