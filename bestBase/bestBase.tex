\documentclass{article}

\usepackage[utf8]{inputenc} % set input encoding (not needed with XeLaTeX)
\usepackage{lmodern}

%%% PAGE DIMENSIONS
\usepackage{geometry} % to change the page dimensions
\geometry{letterpaper} % or letterpaper (US) or a5paper or....
\geometry{margin=1.35in} % for example, change the margins to 2 inches all round

\usepackage{graphicx} % support the \includegraphics command and options

\usepackage[parfill]{parskip} % Activate to begin paragraphs with an empty line rather than an indent

%%% PACKAGES
\usepackage{booktabs} % for much better looking tables
\usepackage{array} % for better arrays (eg matrices) in maths
\usepackage{paralist} % very flexible & customisable lists (eg. enumerate/itemize, etc.)
\usepackage{verbatim} % adds environment for commenting out blocks of text & for better verbatim
\usepackage{subfig} % make it possible to include more than one captioned figure/table in a single float
% These packages are all incorporated in the memoir class to one degree or another...

%%% HEADERS & FOOTERS
\usepackage{fancyhdr} % This should be set AFTER setting up the page geometry
\pagestyle{fancy} % options: empty , plain , fancy/
\renewcommand{\headrulewidth}{0pt} % customise the layout...
\lhead{}\chead{}\rhead{}
\lfoot{}\cfoot{\thepage}\rfoot{}

%%% SECTION TITLE APPEARANCE
\usepackage{sectsty}
%\allsectionsfont{\sffamily\mdseries\upshape} % (See the fntguide.pdf for font help)
% (This matches ConTeXt defaults)

%%% ToC (table of contents) APPEARANCE
\usepackage[nottoc,notlof,notlot]{tocbibind} % Put the bibliography in the ToC
\usepackage[titles,subfigure]{tocloft} % Alter the style of the Table of Contents
\renewcommand{\cftsecfont}{\rmfamily\mdseries\upshape}
\renewcommand{\cftsecpagefont}{\rmfamily\mdseries\upshape} % No bold!

% JPR added
\usepackage{fontawesome}
\usepackage{amsfonts}
%\usepackage{amsmath}
\usepackage{mathtools}% includes amsmath
\usepackage{changepage}
\usepackage{enumerate}
%\usepackage{setspace}
\usepackage{relsize}
\usepackage{wasysym}
%\usepackage{romannum}

\newcommand{\jprVersion}{v01} % version
 
\usepackage[pdftex,
            pdfauthor={James Philip Rowell},
            pdftitle={\jobname.\jprVersion},
            pdfsubject={The Best Number System?},
            pdfkeywords={radix, decimal, base ten, base twenty one, unvigesimal, binary, base two, integers, theorem, proof, mathematics, number theory},
            pdfproducer={Latex},
            pdfcreator={pdflatex}]{hyperref}
\hypersetup{
    colorlinks=true,
    linkcolor=black,
    filecolor=magenta,      
    urlcolor=blue,
}
\usepackage{hyperxmp}
\hypersetup{
    pdfauthor={James Philip Rowell},
    pdfcopyright={Copyright 2019 by James Philip Rowell. All rights reserved.}
}
\usepackage{lipsum}

\newenvironment{jprIn}{\begin{adjustwidth}{2em}{}}{\end{adjustwidth}}
\addtolength{\skip\footins}{6pt}

\usepackage{alphalph}
\makeatletter
\newalphalph{\fnsymbolwrap}[wrap]{\@fnsymbol}{}
\makeatother
\renewcommand*{\thefootnote}{%
  \fnsymbolwrap{\value{footnote}}%
}

\usepackage{perpage}
\usepackage[bottom]{footmisc}

%%% END Article customizations

\title{\vspace{-1.5cm}The Best Number System}
\author{James Philip Rowell}
\date{\vspace{-0.5cm}\footnotesize\today\ (\jprVersion)} % Activate to display a given date or no date (if empty),
% otherwise the current date is printed

\begin{document}
\maketitle
\begin{em}
\centerline{\small{}What might the ``best'' number system be?}
\end{em}

If we could hit the reset-button on the fact that we count
using decimal numbers, and were tasked with finding the \emph{best} number system for day-to-day use,
would we still end up picking decimal?

I know, I know, until we attempt to define what ``best'' means, it's impossible to
answer such a question, especially for a mathematically inclined reader such as yourself.
But please bear with me as we flesh this out a bit.

Let's ignore that the fact
that Sesame Street drilled into several generations of kids heads
that ``10'' means ten things. Oh, and that the most of the world
has been using decimal for about 600 to 800 years; So ignoring those
trivialities
\dots{}

So what criteria might we use to select the ``best'' way to denote individual integers?

I think we can discount esoteric systems, for example mixed radix systems,
or base-\(e\), as being impractical for day-to-day use. They are interesting
and instructive mathematical curiosities but not suitable for day-to-day use.

We can also discount ancient number systems with their 
use of
different symbols for larger and larger numbers, since there's no way to practically
denote arbitrarily large integers. 
For example Roman Numerals only effectively allowed counting up to 4999.

I believe our choice boils down to the question: ``Which value for \(b\) should we choose given the following theorem?''

\section*{Basis Representation Theorem}
\begin{jprIn}
Let $b$ be a positive integer greater than 1.

For every positive integer $n$ there is a unique sequence
of integers $d_0, d_1, d_2,\dots{},d_k$ such that:

\hspace{3em}$n=d_kb^k+d_{k-1}b^{k-1}+\dots+d_2b^2+d_1b^1+d_0b^0$,

where $0\le{}d_i<b$ for all $i$ in $\{0,1,2,\dots{},k\}$ and $d_k\ne0$.

Definition: $n$ is represented in base-$b$ by the string
of base-$b$-digits $(d_kd_{k-1}{\cdots}d_2d_1d_0)_b$
\end{jprIn}

The theorem doesn't say anything about the practicalities behind
using a given base. For example, it's convenient to have 
simple symbols to represent each integer from \(0, 1, \dots{}, (b-1)\) so
that any number can be written in way that's easy for us to read. Needless to say,
in decimal we use these symbols for the integers zero through nine: 0, 1, 2, 3, 4, 5, 6, 7, 8, 9.

Computer Scientists add the letters A, B, C, D, E, F to represent the integers ten through fifteen
when writing numbers in base-sixteen. For example the number (10F2)\(_{16}\) is the decimal integer
4,338.

So whatever \(b\) we pick, then it better be small enough that we can remember all the digits
from zero to \((b-1)\). For example, \(b = 4096\) is probably a poor choice.

Also we possibly need new words for the numbers that go along with our choice for \(b\).
For example the words ``fourteen'' and ``twenty-seven'' are intimately tied with their meaning in decimal.
It's unclear how you'd even say ``14'' or ``27'' in a different base. For example ``14'' in base-5 is really the
integer nine.

Anyway, whatever we pick, we need to consider those things for practical use. We need symbols, and words to
describe the numbers and a small enough base such that we can fairly easily remember all the digits 
from zero to \((b-1)\).
% potentially be able to fairly easily picture those many things in our mind.

\end{document}
