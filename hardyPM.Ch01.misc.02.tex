\documentclass{article}

\usepackage[utf8]{inputenc} % set input encoding (not needed with XeLaTeX)

%%% PAGE DIMENSIONS
\usepackage{geometry} % to change the page dimensions
\geometry{letterpaper} % or letterpaper (US) or a5paper or....
\geometry{margin=1.35in} % for example, change the margins to 2 inches all round

\usepackage{graphicx} % support the \includegraphics command and options

\usepackage[parfill]{parskip} % Activate to begin paragraphs with an empty line rather than an indent

%%% PACKAGES
\usepackage{booktabs} % for much better looking tables
\usepackage{array} % for better arrays (eg matrices) in maths
\usepackage{paralist} % very flexible & customisable lists (eg. enumerate/itemize, etc.)
\usepackage{verbatim} % adds environment for commenting out blocks of text & for better verbatim
\usepackage{subfig} % make it possible to include more than one captioned figure/table in a single float
% These packages are all incorporated in the memoir class to one degree or another...

%%% HEADERS & FOOTERS
\usepackage{fancyhdr} % This should be set AFTER setting up the page geometry
\pagestyle{fancy} % options: empty , plain , fancy/
\renewcommand{\headrulewidth}{0pt} % customise the layout...
\lhead{}\chead{}\rhead{}
\lfoot{}\cfoot{\thepage}\rfoot{}

%%% SECTION TITLE APPEARANCE
\usepackage{sectsty}
\allsectionsfont{\mdseries} % (See the fntguide.pdf for font help)
% (This matches ConTeXt defaults)

%%% ToC (table of contents) APPEARANCE
\usepackage[nottoc,notlof,notlot]{tocbibind} % Put the bibliography in the ToC
\usepackage[titles,subfigure]{tocloft} % Alter the style of the Table of Contents
\renewcommand{\cftsecfont}{\rmfamily\mdseries\upshape}
\renewcommand{\cftsecpagefont}{\rmfamily\mdseries\upshape} % No bold!

% JPR added
\usepackage{fontawesome}
\usepackage{amsfonts}
%\usepackage{amsmath}
\usepackage{mathtools}% includes amsmath
\usepackage{changepage}
\usepackage{enumerate}
%\usepackage{setspace}
\usepackage{relsize}
\usepackage{wasysym}
%\usepackage{romannum}
 
\usepackage[pdftex,
            pdfauthor={James Philip Rowell},
            pdftitle={\jobname},
            pdfsubject={Proof of Chapter 1, Miscellaneous Examples, Number 2 from G.H. Hardy `A Course of Pure Mathematics'.},
            pdfkeywords={rational numbers, theorem, proof, mathematics, number theory, factorial, factoradic, Hardy, Examples, A Course of Pure Mathematics},
            pdfproducer={Latex},
            pdfcreator={miktex or pdflatex}]{hyperref}
\hypersetup{
    colorlinks=true,
    linkcolor=black,
    filecolor=magenta,      
    urlcolor=blue,
}
\usepackage{hyperxmp}
\hypersetup{
    pdfauthor={James Philip Rowell},
    pdfcopyright={Copyright  2018 by James Philip Rowell. All rights reserved.}
}
\usepackage{lipsum}

\newenvironment{jprIn}{\begin{adjustwidth}{2em}{}}{\end{adjustwidth}}
\addtolength{\skip\footins}{6pt}

\usepackage{alphalph}
\makeatletter
\newalphalph{\fnsymbolwrap}[wrap]{\@fnsymbol}{}
\makeatother
\renewcommand*{\thefootnote}{%
  \fnsymbolwrap{\value{footnote}}%
}

\usepackage{perpage}
\MakePerPage{footnote}

\DeclarePairedDelimiter\abs{\lvert}{\rvert}

%%% END Article customizations

\author{James Philip Rowell}
\title{\vspace{-1.5cm}Factoradic Representation of Rational Numbers}
\date{} % Activate to display a given date or no date (if empty), otherwise the current date is printed 
\begin{document}
\maketitle
\begin{em}
\centerline{\small{}From `A Course in Pure Mathematics' by G. H. Hardy. Chapter 1, Miscellaneous Examples.}
%\par
%\setlength{\parskip}{0pt}
\end{em}
%\normalsize
\bigskip

Miscellaneous example\footnote{Hardy doesn't call them `Exercises' or `Questions', but that's what they
are, math exercises like calculations to perform, theorems to prove etc.} \#2 at the end
of chapter 1 in Hardy's `Pure Mathematics'
presents us with a fascinating result.
I had never seen it before, but upon seeing it felt like I
was looking at a kind of basis-representation-theorem but for rational numbers, \dots{} beautiful!

Here it is, followed by my proof starting with the lemma.

\section*{Theorem}

Any positive rational number can be expressed in one and only one way in the form

\[a_1 + \frac{a_2}{1\cdot{}2}
+ \frac{a_3}{1\cdot{}2\cdot{}3}
+ \dots{}
+ \frac{a_k}{1\cdot{}2\cdot{}3\cdot{}\dots{}\cdot{}k},\]

where $a_1, a_2, \dots{}, a_k$ are integers, and
\[0\le{}a_1,\quad 0\le{}a_2<2,\quad 0\le{}a_3<3,\quad \dots{},\quad 0<a_k<k \]

\section*{Lemma}

The set of rational numbers,\newline

\centerline{$\mathcal{S} = \{\frac{a_2}{2!} + \frac{a_3}{3!} + \dots{} + \frac{a_k}{k!}\enspace\mid\enspace 
0\! \le{}\! a_2\! <\! 2,\enspace 
0\! \le{}\! a_3\! <\! 3,\enspace 
\dots{},\enspace 0\! \le{}\! a_k\! <\! k\}$,}
\bigskip
is identical to the set of rational numbers,\newline

\centerline{$\mathcal{F} = \{\frac{0}{k!},\enspace \frac{1}{k!},\enspace \frac{2}{k!},\enspace \dots{},\enspace \frac{k!-1}{k!}\}$}

\section*{Proof of Lemma}

The number of values that the coefficient $a_2$ can assume is 2, $a_3$ can take on 3 values, \dots{},
up to $a_k$ which can take on $k$ values. So the total number of combinations of values
that can be assigned to all the coefficients is $2\cdot{}3\cdot{}4\cdot\dots{}\cdot{}k = k!\,$.

Suppose that the rational number 
$\frac{p}{q} = \frac{a_2}{2!} + \frac{a_3}{3!} + \dots{} + \frac{a_k}{k!}$ is not uniquely 
determined by the coefficients
$a_2, a_3, \dots{}, a_k$. That is, suppose there is a second DIFFERENT sequence of coefficients
$b_2, b_3, \dots{}, b_k$ such that
$\frac{p}{q} = \frac{b_2}{2!} + \frac{b_3}{3!} + \dots{} + \frac{b_k}{k!}$.



The size of $\mathcal{F}$ is clearly $k!$

\end{document}














\centerline{$\frac{1}{1}, \enspace\frac{3}{2}, \enspace\frac{7}{5}, \enspace\frac{17}{12}, \enspace\frac{41}{29}, \enspace\frac{99}{70}$}

So these numbers squared are:\newline

\centerline{$\frac{1}{1}, \enspace\frac{9}{4}, \enspace\frac{49}{25}, \enspace\frac{289}{144}, \enspace\frac{1681}{841}, \enspace\frac{9801}{4900}$}

And their differences from 2 are:\newline

\centerline{$-1, \enspace\frac{1}{4}, \enspace-\frac{1}{25}, \enspace\frac{1}{144}, \enspace-\frac{1}{841}, \enspace\frac{1}{4900}$}

\section*{Lemma}

Let $m$, $n$ and $a$ be positive integers

\[\frac{m^2}{n^2}+\frac{a}{n^2}=2 \text{\hspace{2.5em}} \Leftrightarrow \text{\hspace{2.5em}}\frac{(m+2n)^2}{(m+n)^2}-\frac{a}{(m+n)^2}=2\]

\section*{Proof of Lemma}

\begin{alignat*}{2}
  &&\frac{(m+2n)^2}{(m+n)^2}-\frac{a}{(m+n)^2}
  \text{\hspace{0.5em}}&=\text{\hspace{0.5em}}
  2\\
  &\Leftrightarrow\quad
  &(m+2n)^2 - a
  \text{\hspace{0.5em}}&=\text{\hspace{0.5em}}
  2(m+n)^2\\
  &\Leftrightarrow\quad
  &m^2 + 4mn + 4n^2
  \text{\hspace{0.5em}}&=\text{\hspace{0.5em}}
  2m^2 + 4mn + 2n^2 + a\\
  &\Leftrightarrow\quad
  &4n^2 - 2n^2
  \text{\hspace{0.5em}}&=\text{\hspace{0.5em}}
  2m^2 - m^2 + a\\
  &\Leftrightarrow\quad
  &2n^2
  \text{\hspace{0.5em}}&=\text{\hspace{0.5em}}
  m^2 + a\\
  &\Leftrightarrow\quad
  &2
  \text{\hspace{0.5em}}&=\text{\hspace{0.5em}}
  \frac{m^2}{n^2}+\frac{a}{n^2}\\
\end{alignat*}
\centerline{QED}

\section*{Proof of Theorem}
Define the ``error'' of a rational approximation to $\sqrt{2}$ as the difference between the
approximation-squared and 2.
So we learn from the Lemma that the error for the $\frac{m}{n}$ approximation of $\sqrt{2}$ is\newline
\centerline{$\abs*{\frac{a}{n^2}} = \abs*{2 - \frac{m^2}{n^2}}$,}

and the error for the $\frac{(m+2n)}{(m+n)}$ approximation of $\sqrt{2}$ is\newline
\centerline{$\abs*{\frac{-a}{(m+n)^2}} = \abs*{2 - \frac{(m+2n)^2}{(m+n)^2}}$.}

\bigskip
The denominator $(m+n)^2$ is always larger than $n^2$ therefore,
\[\abs*{\frac{-a}{(m+n)^2}} < \abs*{\frac{a}{n^2}} \text{,}\]
showing that the approximation $\frac{(m+2n)}{(m+n)}$ for
$\sqrt{2}$ 
has a smaller error that that of $\frac{m}{n}$ and is therefore
a better approximation to $\sqrt{2}$.

Furthermore, we also learn from the Lemma that the successive approximations flip
to either side of 2 because:

If $\frac{m^2}{n^2} < 2$ then $a$ is positive therefore $\frac{(m+2n)}{(m+n)} > 2$, similarly,

If $\frac{m^2}{n^2} > 2$ then $a$ is negative therefore $\frac{(m+2n)}{(m+n)} < 2$.

\bigskip
\centerline{QED}
