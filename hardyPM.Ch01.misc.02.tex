\documentclass{article}

\usepackage[utf8]{inputenc} % set input encoding (not needed with XeLaTeX)

%%% PAGE DIMENSIONS
\usepackage{geometry} % to change the page dimensions
\geometry{letterpaper} % or letterpaper (US) or a5paper or....
\geometry{margin=1.35in} % for example, change the margins to 2 inches all round

\usepackage{graphicx} % support the \includegraphics command and options

\usepackage[parfill]{parskip} % Activate to begin paragraphs with an empty line rather than an indent

%%% PACKAGES
\usepackage{booktabs} % for much better looking tables
\usepackage{array} % for better arrays (eg matrices) in maths
\usepackage{paralist} % very flexible & customisable lists (eg. enumerate/itemize, etc.)
\usepackage{verbatim} % adds environment for commenting out blocks of text & for better verbatim
\usepackage{subfig} % make it possible to include more than one captioned figure/table in a single float
% These packages are all incorporated in the memoir class to one degree or another...

%%% HEADERS & FOOTERS
\usepackage{fancyhdr} % This should be set AFTER setting up the page geometry
\pagestyle{fancy} % options: empty , plain , fancy/
\renewcommand{\headrulewidth}{0pt} % customise the layout...
\lhead{}\chead{}\rhead{}
\lfoot{}\cfoot{\thepage}\rfoot{}

%%% SECTION TITLE APPEARANCE
\usepackage{sectsty}
%\allsectionsfont{\sffamily\mdseries\upshape} % (See the fntguide.pdf for font help)
% (This matches ConTeXt defaults)

%%% ToC (table of contents) APPEARANCE
\usepackage[nottoc,notlof,notlot]{tocbibind} % Put the bibliography in the ToC
\usepackage[titles,subfigure]{tocloft} % Alter the style of the Table of Contents
\renewcommand{\cftsecfont}{\rmfamily\mdseries\upshape}
\renewcommand{\cftsecpagefont}{\rmfamily\mdseries\upshape} % No bold!

% JPR added
\usepackage{fontawesome}
\usepackage{amsfonts}
%\usepackage{amsmath}
\usepackage{mathtools}% includes amsmath
\usepackage{changepage}
\usepackage{enumerate}
%\usepackage{setspace}
\usepackage{relsize}
\usepackage{wasysym}
%\usepackage{romannum}
 
\usepackage[pdftex,
            pdfauthor={James Philip Rowell},
            pdftitle={\jobname},
            pdfsubject={Proof of Chapter 1, Miscellaneous Examples, Number 2 from G.H. Hardy `A Course of Pure Mathematics'.},
            pdfkeywords={rational numbers, theorem, proof, mathematics, number theory, factorial, factoradic, Hardy, Examples, A Course of Pure Mathematics},
            pdfproducer={Latex},
            pdfcreator={miktex or pdflatex}]{hyperref}
\hypersetup{
    colorlinks=true,
    linkcolor=black,
    filecolor=magenta,      
    urlcolor=blue,
}
\usepackage{hyperxmp}
\hypersetup{
    pdfauthor={James Philip Rowell},
    pdfcopyright={Copyright  2018 by James Philip Rowell. All rights reserved.}
}
\usepackage{lipsum}

\newenvironment{jprIn}{\begin{adjustwidth}{2em}{}}{\end{adjustwidth}}
\addtolength{\skip\footins}{6pt}

\usepackage{alphalph}
\makeatletter
\newalphalph{\fnsymbolwrap}[wrap]{\@fnsymbol}{}
\makeatother
\renewcommand*{\thefootnote}{%
  \fnsymbolwrap{\value{footnote}}%
}

\usepackage{perpage}
\MakePerPage{footnote}

\DeclarePairedDelimiter\abs{\lvert}{\rvert}

%%% END Article customizations

\author{James Philip Rowell}
\title{\vspace{-1.5cm}Factorial Basis Representation of Rational Numbers}
\date{} % Activate to display a given date or no date (if empty), otherwise the current date is printed 
\begin{document}
\maketitle
% \begin{em}
% \centerline{\small{}From `A Course of Pure Mathematics' by G. H. Hardy. Chapter 1, Miscellaneous Examples.}
%\par
%\setlength{\parskip}{0pt}
% \end{em}
%\normalsize
% \bigskip

Miscellaneous example-2\footnote{Hardy doesn't call them `Exercises' or `Questions', but that's what they
are, math exercises for the student.} at the end
of chapter 1 in G. H. Hardy's `A Course of Pure Mathematics'
presents us with a fascinating result.
The theorem feels like what the `\href {https://www.dropbox.com/s/bwmrffmkcidnf27/basisReprThm.pdf?dl=0}
{basis-representation-theorem}'
is for integers, but for rational numbers \dots{} beautiful!

\section*{Factorial Representation Theorem\footnote{The theorem is not named in the text, so I named it.}}

Any positive rational number can be expressed in one and only one way in the form
{\small
\bgroup                                  %% open the group
\setlength{\abovedisplayskip}{0pt}       %% effective inside the group    
\begin{center}
\begin{equation*}
a_1 + \frac{a_2}{1\cdot{}2}
+ \frac{a_3}{1\cdot{}2\cdot{}3}
+ \dots{}
+ \frac{a_k}{1\cdot{}2\cdot{}3\cdot{}\dots{}\cdot{}k},
\end{equation*}
\end{center}
\egroup
}

where \(a_1, a_2, \dots{}, a_k\) are integers, and

{\small
\begin{center}
\(0\le{}a_1,\quad 0\le{}a_2<2,\quad 0\le{}a_3<3,\quad \dots{},\quad 0<a_k<k\)
\end{center}
}

\section*{Observations that led me to the proof.}

Any positive rational number\footnote{Every variable,
or constant (eg. \(a_1, a_k, m, n, i, p, q\)) in this
paper is going to represent a non-negative integer. We aren't dealing with `real numbers' here, just
non-negative rational numbers which we will always discuss in terms of one integer divided
by another integer, like \(\frac{p}{q}\).}, say \(\frac{m}{q}\), can be
% \emph{uniquely}
written as an
integer part, \(i\), plus a fractional\newline part, \(\frac{p}{q}\), such that
\(\frac{m}{q} = i + \frac{p}{q}\), where \(0 \le{} \frac{p}{q} < 1\).
%(note that \(i\) can be zero).

So trying to represent any positive
rational number \(\frac{m}{q}\) in the form of the theorem, the integer \(a_1\)
wants to play the role of the integer part, \(i\),
and the remainder of the expression \(\frac{a_2}{2!} + \frac{a_3}{3!} + \dots{} + \frac{a_k}{k!}\)
looks to be playing the role of the rational part, \(\frac{p}{q}\), where,

{\small
\bgroup                                  %% open the group
\setlength{\abovedisplayskip}{0pt}       %% effective inside the group    
\begin{center}
\begin{equation*}
0 \le{} 
\frac{a_2}{2!}
+ \frac{a_3}{3!}
+ \dots{}
+ \frac{a_k}{k!}
< 1
\end{equation*}
\end{center}
\egroup
}

It seemed a good idea to forget about the integer \(a_1\)
and just focus on the integers \(a_2, a_3, \dots{}, a_k\).
In other words,
first prove a restricted form of the theorem for rational numbers between zero and one,
then later prove the full theorem by re-introducing the \(a_1\) to get all positive rational numbers.
Furthermore, including zero (that is, not just \emph{positive} rational numbers) was also
going to make things easier with this approach.

At first, it wasn't remotely obvious how I'd go about calculating
the values of the integers \(a_2, a_3, \dots{}, a_k\) for a given number
\(\frac{p}{q}\).
However, after playing around for a while I figured it out;
it's kinda like doing long-division. At this point
a few patterns started to jump out at me.
For example, look at these numbers,

{\small
\bgroup
\setlength{\abovedisplayskip}{0pt}
\setlength{\belowdisplayskip}{0pt}
\begin{gather*}
\frac{1}{1\cdot{}2}
= \frac{1}{2}
= \frac{2!-1}{2!}\\
\frac{1}{1\cdot{}2}
+ \frac{2}{1\cdot{}2\cdot{}3}
= \frac{1\cdot{}3}{1\cdot{}2\cdot{}3}
+ \frac{2}{1\cdot{}2\cdot{}3}
= \frac{3+2}{6}
= \frac{5}{6}
= \frac{3!-1}{3!}\\
\frac{1}{1\cdot{}2}
+ \frac{2}{1\cdot{}2\cdot{}3}
+ \frac{3}{1\cdot{}2\cdot{}3\cdot{}4}
= \frac{1\cdot{}3\cdot{}4}{1\cdot{}2\cdot{}3\cdot{}4}
+ \frac{2\cdot{}4}{1\cdot{}2\cdot{}3\cdot{}4}
+ \frac{3}{1\cdot{}2\cdot{}3\cdot{}4}
= \frac{12+8+3}{24}
= \frac{23}{24}
= \frac{4!-1}{4!}
\end{gather*}
\egroup
}

An obvious pattern emerged!
By assigning the largest possible values to the variables,
the sum is \(\frac{k!-1}{k!}\),
which is as close to 1 as possible without actually
hitting 1. (What is \(\frac{1}{k!} + \frac{k!-1}{k!}\)?)
This turned out to be
a pretty useful observation, and it became `Lemma 1' in what follows.

On the other end of the scale, assigning zeros to all the variables 
gives us a sum of zero.
Furthermore
I could make
the smallest \emph{positive} number, \(\frac{1}{k!}\), by
letting all the variables be zero \emph{except} for \(a_k = 1\).
Then I considered what happens to the variables by
adding \(\frac{1}{k!}\) to it, repeatedly, over and over.

So by making various assignments of values to
the variables \(a_2, a_3, \dots{}, a_k\) I could
generate the smallest numbers, \(\frac{0}{k!}, \frac{1}{k!}, \frac{2}{k!}\), etc.
as well as the largest,
\(\frac{k!-1}{k!}\).
Then I realized that I could count the number of possible sums that might be formed 
with the expression.

There are two choices for the \(a_2\) variable (0 and 1),
combined with three choices for the \(a_3\) variable (0, 1 and 2),
combined with four choices for the \(a_4\) variable (0, 1, 2, 3),
\dots{} combined with \(k\) choices for
the \(a_k\) variable (0, 1, 2, \dots{}, \(k\!-\!1\)), which
gives us \(2\cdot{}3\cdot{}4\cdot{}\cdot{}\cdot{}k = k!\) possibly different sums.

Hmmm, the following set has \(k!\) members,
\(\{\frac{0}{k!},\enspace \frac{1}{k!},\enspace \frac{2}{k!},
\enspace \dots{},\enspace \frac{k!-1}{k!}\}\).
By forming another set out of all possible sums from the expression
it seemed like it would be fairly easy to show that the two sets are identical.

Another insight was that in order to be able to represent
a fraction with a prime-number denominator (say \(p\))
then the sum in our
expression would \emph{have to contain} this term,
\(\frac{a_p}{p!}\),
with a non-zero value of \(a_p\).

This led me to the realization that any fraction \(\frac{m}{k}\)
should be able to be represented by the expression
if the terms went so far as to include \(\frac{a_k}{k!}\). For that
matter by using terms up to \(\frac{a_k}{k!}\), then I'd
also be able to represent all fractions with
\((k-1)\) as the denominator or 
\((k-2)\) as the denominator, or \dots{}
4, 3, or 2 as the denominator. In other words,
\(\frac{1}{2}\),
\(\frac{1}{3}\),
\(\frac{2}{3}\),
\(\frac{1}{4}\),
\(\frac{2}{4}\),
\(\frac{3}{4}\),
\(\frac{1}{5}\),
\(\frac{2}{5}\),
\(\frac{3}{5}\),
\(\frac{4}{5}\),
\(\dots{}\),
\(\frac{1}{k-1}\),
\(\frac{2}{k-1}\),
\(\frac{3}{k-1}\),
\(\frac{4}{k-1}\),
\(\dots{}\),
\(\frac{k-3}{k-1}\),
\(\frac{k-2}{k-1}\),
\(\frac{1}{k}\),
\(\frac{2}{k}\),
\(\frac{3}{k}\),
\(\frac{4}{k}\),
\(\dots{}\),
\(\frac{k-2}{k}\),
\(\frac{k-1}{k}\)
should be
representable with the expression 
\(\frac{a_2}{2!} + \frac{a_3}{3!} + \dots{} + \frac{a_k}{k!}\). 

So by using sets to collect all the numbers generated
by our expression, then letting \(k\) grow without bound
I should get a set that contains \emph{all} the rational numbers between zero and one!
Those are the ideas behind the proof that follows.

\break
\section*{Lemma 1}
{\small
\bgroup                                  %% open the group
\setlength{\abovedisplayskip}{0pt}       %% effective inside the group    
\begin{center}
\begin{equation*}
\frac{1}{2!} + \frac{2}{3!} + \dots{} + \frac{k-1}{k!} = \frac{k!-1}{k!},\quad \text{for all integers } k \ge{} 2
\end{equation*}
\end{center}
\egroup
}
\subsubsection*{Proof}

Induction: When \(k = 2\) it's clear that \(\frac{1}{2!} = \frac{2!-1}{2!}\), and noting that
\(\frac{2-2}{1!} = \frac{0}{1!} = \frac{1!-1}{1!}\) then
{\scriptsize
\begin{alignat*}{1}
\frac{1}{2!} + \frac{2}{3!} + \dots{} + \frac{k-2}{(k-1)!} + \frac{k-1}{k!}
  =\  &\frac{(k-1)!-1}{(k-1)!} + \frac{k-1}{k!}\\
  =\  &\frac{k((k-1)!-1)}{k(k-1)!} + \frac{k-1}{k!}\\
  =\  &\frac{k!-k+k-1}{k!}\\
  =\  &\frac{k!-1}{k!}
\end{alignat*}
}
\dots{}thus establishing Lemma 1 for all values of \(k \ge 2\).\qquad QED.

% \bigskip % Need bigger skip this time.
\vspace{18pt}
The following lemma captures an idea that is perhaps most easily grasped by analogy to the basis representation
theorem for integers. For base-ten numbers we can say,

\begin{center}
{\footnotesize\(1\cdot{}10^k >
9\cdot{}10^{k-1}
+ 9\cdot{}10^{k-2}
+ \dots{}
+ 9\cdot{}10^{2}
+ 9\cdot{}10^{1}
+ 9\cdot{}10^{0}
\)}
\end{center}

The above inequality 
is merely stating that any single power of ten is bigger than the sum of
every smaller power of ten, each times 9.  For example, 1000 is bigger than 999.
Read the statement of the inequality in Lemma 2 with this idea in mind.

\section*{Lemma 2}

For integers \(i,k\) where \(2\le{}i<k\),

{\footnotesize
\bgroup                                  %% open the group
\setlength{\abovedisplayskip}{0pt}       %% effective inside the group    
\begin{center}
\begin{equation*}
\frac{1}{i!} >
\frac{i}{(i+1)!}
+ \frac{i+1}{(i+2)!}
+ \dots{}
+ \frac{k-2}{(k-1)!}
+ \frac{k-1}{k!}
\end{equation*}
\end{center}
\egroup
}
\subsubsection*{Proof}
{\scriptsize
\bgroup
\setlength{\abovedisplayskip}{0pt}
\begin{align*}
\frac{i}{(i+1)!} + \frac{i+1}{(i+2)!} + \dots{} + \frac{k-2}{(k-1)!} + \frac{k-1}{k!}
  =\  &(\frac{1}{2!} + \frac{2}{3!} + \dots{} + \frac{k-1}{k!})
  - (\frac{1}{2!} + \frac{2}{3!} + \dots{} + \frac{i-1}{i!})\\
  =\  &\frac{k!-1}{k!} - \frac{i!-1}{i!}\qquad \qquad \text{(by Lemma 1)}\\
  =\  &\frac{k!}{k!} - \frac{1}{k!} - \frac{i!}{i!} + \frac{1}{i!}\\
  =\  &\frac{1}{i!} - \frac{1}{k!}\\
  <\  &\frac{1}{i!} \qquad \qquad \qquad \qquad \qquad \text{QED.}
\end{align*}
\egroup
}

\section*{Definitions}

For integer \(k \ge 2\), and integers \(a_2, a_3, \dots{}, a_k\), we define the following sets,

{\small
\bgroup                                  %% open the group
\setlength{\abovedisplayskip}{0pt}
\begin{center}
\begin{gather*}
\mathcal{S}_k = \{\frac{a_2}{2!} + \frac{a_3}{3!} + \dots{} + \frac{a_k}{k!}\enspace\mid\enspace 
0\! \le{}\! a_2\! <\! 2,\enspace 
0\! \le{}\! a_3\! <\! 3,\enspace 
\dots{},\enspace 0\! \le{}\! a_k\! <\! k\},\\
\mathcal{F}_k = \{\frac{0}{k!},\enspace \frac{1}{k!},\enspace \frac{2}{k!},\enspace \dots{},\enspace \frac{k!-1}{k!}\}
\end{gather*}
\end{center}
\egroup
}

Let \(a \in \mathcal{S}_k\) such that \(a = \frac{a_2}{2!} + \frac{a_3}{3!} + \dots{} + \frac{a_k}{k!}\),
for some \(a_2, a_3, \dots{}, a_k\), 
where
\(k \ge{} 2\), and {\footnotesize\(0\!\le{}\!a_2\!<\!2,\enspace
0\!\le{}\!a_3\!<\!3,\enspace
\dots{},\enspace 0\!\le{}\!a_k\!<\!k\)}.
The expression \(\frac{a_2}{2!} + \frac{a_3}{3!} + \dots{} + \frac{a_k}{k!}\) is called
``the factorial-representation of \(a\)''.

\section*{Lemma 3}

{\small
\bgroup                                  %% open the group
\setlength{\abovedisplayskip}{0pt}
\begin{center}
\begin{equation*}
\mathcal{S}_k = \mathcal{F}_k
\end{equation*}
\end{center}
\egroup
}

\subsubsection*{Proof}

Two sets are equal if and only if they have the same elements\footnote{If
that isn't clear:
\((\mathcal{X} = \mathcal{Y}) \Leftrightarrow (\forall a;
a \in \mathcal{X} \Leftrightarrow a \in \mathcal{Y})\).}.

We're going to establish equality by showing three properties
of the two sets \(\mathcal{S}_k\) and \(\mathcal{F}_k\).

First, they are the same size.
Sets can contain anything,
so obviously
that's not enough to show equality.

Secondly, all the elements of 
\(\mathcal{S}_k\) can be written in the exact same form as the elements of \(\mathcal{F}_k\).
That's getting us closer, but still isn't sufficient because the numbers in \(\mathcal{S}_k\)
could be anywhere on the number-line
so they might not be the same as those in \(\mathcal{F}_k\).

Third, by showing that the smallest and largest elements are the same, then this pins down exactly where
the numbers in \(\mathcal{S}_k\) sit on the number-line, and thus establish equality between the two sets.

To begin with, it's clear that the set \(\mathcal{F}_k\) contains every rational number
of the form \(\frac{p}{k!}\)
% with denominator \(k!\)
where \(p\) is an integer and \mbox{\(0\le{}\frac{p}{k!}<1\)} and
that the size of \(\mathcal{F}_k\) is \(k!\, \). That takes care of everything we need to know
about \(\mathcal{F}_k\); now on to show that \(\mathcal{S}_k\) has these same qualities.

The smallest member of the set \(\mathcal{S}_k\) 
is \(\frac{0}{k!}\) and
occurs when all the variables are set to zero.
Furthermore, the largest member of the set occurs when all the variables are set to
their maximum value, which sums to \(\frac{k!-1}{k!}\) as shown in Lemma 1.

We also note that every member of \(\mathcal{S}_k\) can be written as a rational number
with \(k!\) as the denominator, like so,

{\footnotesize
\bgroup                                  %% open the group
\setlength{\abovedisplayskip}{0pt}
\begin{center}
\begin{equation*}
\frac{a_2}{2!} + \frac{a_3}{3!} + \dots{} + \frac{a_{k-1}}{(k-1)!} + \frac{a_k}{k!}
= \frac{k\cdot{}(k-1)\cdot{}\cdot{}\cdot{}3\cdot{}a_2}{k!}
+ \frac{k\cdot{}(k-1)\cdot{}\cdot{}\cdot{}4\cdot{}a_2}{k!}
+ \dots{}
+ \frac{k\cdot{}a_{k-1}}{k!}
+ \frac{a_k}{k!}
\end{equation*}
\end{center}
\egroup
}

\break
Therefore any member of the set \(\mathcal{S}_k\) can be written as
\(\frac{p}{k!}\) for some integer \(p\), where

{\footnotesize
\bgroup                                  %% open the group
\setlength{\abovedisplayskip}{0pt}
\begin{center}
\begin{equation*}
0 = \frac{0}{k!} \le{} \frac{p}{k!} \le{} \frac{k!-1}{k!} < \frac{k!}{k!} = 1,
\text{\quad hence,\quad}
0 \le{} \frac{p}{k!} < 1
\end{equation*}
\end{center}
\egroup
}

We now show that each possible assignment of values to the variables of
the factorial-representation of \(\frac{p}{k!}\)
produces a \emph{unique} member of the set \(\mathcal{S}_k\).
% \footnote{\dots{} and a good thing too
% otherwise the set wouldn't be well defined!}.

For if this weren't true and both
\(\frac{p}{k!} = \frac{a_2}{2!} + \frac{a_3}{3!} + \dots{} + \frac{a_k}{k!}\) and
\(\frac{p}{k!} = \frac{b_2}{2!} + \frac{b_3}{3!} + \dots{} + \frac{b_k}{k!}\)
for different variables \(a_2, a_3, \dots{}, a_k\)
and \(b_2, b_3, \dots{}, b_k\), then we can arrive at a contradiction as follows.

Suppose \(a_i\ne{}b_i\), where \(i \le{} k\), is the first such pair of variables
that differ. In other words,
\(a_2\!=\!b_2,\ a_3\!=\!b_3, \dots{},\ a_{i-1}\!=\!b_{i-1},\ a_i\!\ne{}\!b_i\).
Without loss of generality, further suppose that \(a_i > b_i\).
It follows that,
% 
% Because
% of the equality of the two different representations for \(\frac{p}{k!}\) we can now write,

{\footnotesize
\bgroup
\setlength{\abovedisplayskip}{0pt}
\setlength{\belowdisplayskip}{0pt}
\begin{alignat}{3}
  &&\frac{a_i}{i!}
  + \frac{a_{i+1}}{(i+1)!}
  + \frac{a_{i+2}}{(i+2)!}
  + \dots{}
  % + \frac{a_{k-1}}{(k-1)!}
  + \frac{a_k}{k!}
  &= \frac{b_i}{i!}
  + \frac{b_{i+1}}{(i+1)!}
  + \frac{b_{i+2}}{(i+2)!}
  + \dots{}
  % + \frac{b_{k-1}}{(k-1)!}
  + \frac{b_k}{k!} \nonumber \\
  &\Leftrightarrow\quad &\frac{a_i - b_i}{i!}
  &= \frac{b_{i+1} - a_{i+1}}{(i+1)!}
  + \frac{b_{i+2} - a_{i+2}}{(i+2)!}
  + \dots{}
  % + \frac{b_{k-1} - a_{k-1}}{(k-1)!}
  + \frac{b_k - a_k}{k!} \label{eqn1}
\end{alignat}
\egroup
}%\vspace{-4mm}

But \(a_i - b_i \ge{} 1\), so

\begin{center}
\(\frac{a_i - b_i}{i!} \ge{} \frac{1}{i!}\).
\end{center}

% In the case that \(i = k\) we get an immediate contradiction because 
% equation \eqref{eqn1} tells us that \(\frac{a_i - b_i}{i!} = 0\) which is
% clearly false.
% 
% So let's carry on assuming that \(i < k\)
Let's examine
one of the terms
on the right-side of \eqref{eqn1}, say
the first one \(\frac{b_{i+1} - a_{i+1}}{(i+1)!}\). We can see that since
\(0 \le b_{i+1} \le{} i\) and \(0 \le a_{i+1} \le{} i\), so \((i - 0) \ge{} (b_{i+1} - a_{i+1})\), therefore,

\begin{center}
\(\frac{i - 0}{(i+1)!} \ge{} \frac{b_{i+1} - a_{i+1}}{(i+1)!}\).
\end{center}

Similarly by extending this idea to the other terms,

\begin{center}
\(
\frac{i}{(i+1)!}
+ \frac{i+1}{(i+2)!}
+ \dots{} + \frac{k-1}{k!} \ge{}
\frac{b_{i+1} - a_{i+1}}{(i+1)!}
+ \frac{b_{i+2} - a_{i+2}}{(i+2)!}
+ \dots{} + \frac{b_k - a_k}{k!}\).
\end{center}

Furthermore, Lemma 2 tells us that \(\frac{1}{i!} > 
\frac{i}{(i+1)!}
+ \frac{i+1}{(i+2)!}
+ \dots{} + \frac{k-1}{k!}\), so
we can string all our inequalities together as follows,

\begin{center}
\(\frac{a_i - b_i}{i!} \ge{} \frac{1}{i!} > \frac{i}{(i+1)!} + \dots{} + \frac{k-1}{k!} \ge{} \frac{b_{i+1} - a_{i+1}}{(i+1)!} + \dots{} + \frac{b_k - a_k}{k!}\),
\end{center}

and hence,

\begin{center}
\(\frac{a_i - b_i}{i!} > \frac{b_{i+1} - a_{i+1}}{(i+1)!} + \dots{} + \frac{b_k - a_k}{k!}\),
\end{center}

But in equation \eqref{eqn1},
\(\frac{a_i - b_i}{i!} = \frac{b_{i+1} - a_{i+1}}{(i+1)!} + \dots{} + \frac{b_k - a_k}{k!}\)
which provides the desired contradiction.

Therefore our assumption that there can be a different set
of variables representing the same rational number \(\frac{p}{k!}\) must be false.
Therefore any assignment of values to the variables of the sum
\(\frac{a_2}{2!} + \frac{a_3}{3!} + \dots{} + \frac{a_k}{k!}\)
produces a \emph{unique} number in the set \(\mathcal{S}_k\).

Now we can count the number of members of \(\mathcal{S}_k\), by looking at all the
possible combinations of values for the variables \(a_2, a_3, \dots{}, a_k\).
There are 2 choices for the variable \(a_2\),
combined with 3 choices for \(a_3\),
combined with 4 choices for \(a_4\),
\dots{},
combined with \(k\) choices for \(a_k\).

Therefore the total number of combinations of values
that can be assigned to the variables of
\(\frac{a_2}{2!}
+ \frac{a_3}{3!}
+ \frac{a_4}{4!}
+ \dots{} + \frac{a_k}{k!}\)
is \(2\cdot{}3\cdot{}4\cdot\cdot{}\cdot{}k = k!\,\). Since each set of assignments 
creates a \emph{unique} member of the set, then the size of the set \(\mathcal{S}_k\) is \(k!\)
which is also the size of the set \(\mathcal{F}_k\).
Recalling from above that any 
member of the set \(\mathcal{S}_k\), say \(\frac{a}{b}\),
can be written as \(\frac{a}{b} = \frac{p}{k!}\), for some \(p\) where 
\(0\le{}\frac{p}{k!}<1\) then it follows that
\(\mathcal{S}_k = \mathcal{F}_k\).

QED.

\section*{Corollary to Lemma 3}

The factorial-representation for every non-negative rational number \(\frac{p}{k!} \in \mathcal{F}_k\) is 
\emph{unique}.

The corollary is a direct result of the fact that
the sets
\(\mathcal{F}_k\) and
\(\mathcal{S}_k\) are equal.

\section*{Definitions}

{\small
\begin{gather*}
\mathcal{F} = \bigcup\limits_{k=2}^{\infty} \mathcal{F}_k \\
\mathcal{S} = \bigcup\limits_{k=2}^{\infty} \mathcal{S}_k \\
\mathbb{Q}_{01} = \{0\} \cup \{\frac{p}{q} \mid p, q \in \mathbb{Z},\text{ where } q \ge{} 2,\ 1 \le{} p < q,
\text{ and } \gcd(p,q) = 1\}
\end{gather*}
}

\(\mathcal{F}\) is the set of ALL non-negative rational numbers, less than one, that could be
formed with any factorial as the denominator.

\(\mathcal{S}\) is the set of ALL non-negative rational numbers, less than one,
formed by every factorial-representation possible.

Finally,
\(\mathbb{Q}_{01}\) is the set of ALL non-negative rational numbers \(\frac{p}{q} < 1\),
where \(p\) and \(q\) are co-prime. We specify the co-prime condition
so that our description
generates unique members of the set. This constraint will come in handy later in the proof.
Also, we want zero to be included so we toss it back into the set with the union.
We are not going to prove it here, but it is straightforward to show that \(\mathbb{Q}_{01}\) contains ALL the
rational numbers between zero and one, and we will take this as established.

Recall that two sets,
\(\mathcal{X}\) and \(\mathcal{Y}\)
are equal if and only if, for all \(a\),

{\normalsize
\bgroup                                  %% open the group
\setlength{\abovedisplayskip}{0pt}
\begin{center}
\begin{equation*}
a \in \mathcal{X} \Leftrightarrow a \in \mathcal{Y}.
\end{equation*}
\end{center}
\egroup
}

\section*{Lemma 4}

{\normalsize
\bgroup                                  %% open the group
\setlength{\abovedisplayskip}{0pt}
\begin{center}
\begin{equation*}
\mathbb{Q}_{01}
= \mathcal{S}
\end{equation*}
\end{center}
\egroup
}

\break
\subsubsection*{Proof}

Lemma 3 tells us that the sets
\(\mathcal{F}_k\) and \(\mathcal{S}_k\) are equal, so clearly,
{\small
\begin{gather*}
\mathcal{F} = \bigcup\limits_{k=2}^{\infty} \mathcal{F}_k
= \bigcup\limits_{k=2}^{\infty} \mathcal{S}_k = \mathcal{S},
\end{gather*}
}
hence,
{\small
\begin{gather}
\mathcal{F} = \mathcal{S}. \label{eqn2}
\end{gather}
}

For all \(\frac{p}{k} \in \mathbb{Q}_{01}\) we have
\(\frac{p}{k} \in{} \mathcal{F}_k\) because,
{\small
\bgroup                                  %% open the group
%\setlength{\abovedisplayskip}{0pt}       %% effective inside the group    
\begin{equation*}
\frac{p}{k}
= \frac{2\cdot{}3\cdot{}\cdot{}\cdot{}(k-1)}{2\cdot{}3\cdot{}\cdot{}\cdot{}(k-1)} \cdot{} \frac{p}{k}
= \frac{2\cdot{}3\cdot{}\cdot{}\cdot{}(k-1)\cdot{}p}{k!},
\end{equation*}
\egroup
}
and because \(\mathcal{F}_k \subset \mathcal{F}\), then \(\frac{p}{k} \in \mathcal{F}\).

Conversely for all \(\frac{p}{k!} \in \mathcal{F}\), then \(\frac{p}{k!} \in \mathbb{Q}_{01}\) as follows,

Let \(m = \frac{p}{\gcd(p,k!)}\)
and \(n = \frac{k!}{\gcd(p,k!)}\).
{\small
\bgroup                                  %% open the group
\setlength{\abovedisplayskip}{0pt}       %% effective inside the group    
\begin{equation*}
\frac{p}{k!} = \frac{p*\frac{1}{\gcd(p,k!)}}{k!*\frac{1}{\gcd(p,k!)}} = \frac{m}{n},
\end{equation*}
\egroup
}\newline
and since \(\gcd(m,n) = 1\) then \(\frac{m}{n} \in \mathbb{Q}_{01}\), that is; \(\frac{p}{k!} \in \mathbb{Q}_{01}\).

Therefore, for all \(a;\ a \in \mathcal{F} \Leftrightarrow a \in \mathbb{Q}_{01}\), which means,
\(\mathcal{F} = \mathbb{Q}_{01}\),
hence by the transitivity of equality, and equation \eqref{eqn2} above,

\begin{center}
\(\mathcal{S} = \mathbb{Q}_{01}\).
\end{center}

QED.

\section*{Corollary to Lemma 4}

The factorial-representation for every rational number in \(\mathbb{Q}_{01}\) is \emph{unique}.

The corollary is a direct result of the fact that
the sets \(\mathcal{S}\) and \(\mathbb{Q}_{01}\) are equal.

That statement is sufficient to prove the corollary, so you can skip to
the actual proof of the ``Factorial Representation Theorem'' if you like, however
to shed a little more light on why this corollary is true, it's worth looking at a few details.

We first note that
\(\mathcal{F}_2
\subset \mathcal{F}_3
\subset \mathcal{F}_4
\subset \dots{}\) because,
for all \(\frac{p}{k!} \in \mathcal{F}_k\),
since \(\frac{p}{k!} = \frac{(k+1)\cdot{}p}{(k+1)\cdot{}k!} = \frac{(k+1)\cdot{}p}{(k+1)!}\),
and \(\frac{(k+1)\cdot{}p}{(k+1)!} \in \mathcal{F}_{k+1}\),
therefore \(\frac{p}{k!} \in \mathcal{F}_{k+1}\),
hence \(\mathcal{F}_k \subset \mathcal{F}_{k+1}\).

% and
% Lemma 3 tells us that \(\mathcal{F}_k = \mathcal{S}_k\),
% so proving that 
% \(\mathcal{F}_i\) is a proper subset of its successor \(\mathcal{F}_{i+1}, \forall i \ge{} 2\)
% means that it must be true for \(\mathcal{S}_i, \forall i \ge{} 2\).
% However it's interesting to see the specific
% mechanisms that makes each sequence of supersets true without relying on Lemma 3.
% We'll start with \(\mathcal{F}_k\).

% Since each successive set is larger, then by induction it is straightforward to conclude that
% \(\mathcal{F}_2
% \subset \mathcal{F}_3
% \subset \mathcal{F}_4
% \subset \dots{}
% \), for all \(\mathcal{F}_i\), \(i \ge{} 2\).

Similarly
\(\mathcal{S}_2
\subset \mathcal{S}_3
\subset \mathcal{S}_4
\subset \dots{}\), because
for all \(\frac{a_2}{2!}\!+\!\frac{a_3}{3!}\!+\!\dots{}\!+\!\frac{a_k}{k!} \in \mathcal{S}_k\),
since\newline
\(\frac{a_2}{2!}\!+\!\frac{a_3}{3!}\!+\!\dots{}\!+\!\frac{a_k}{k!} =
\frac{a_2}{2!}\!+\!\frac{a_3}{3!}\!+\!\dots{}\!+\!\frac{a_k}{k!}\!+\!\frac{0}{(k+1)!}\)
and
\(\frac{a_2}{2!}\!+\!\frac{a_3}{3!}\!+\!\dots{}\!+\!\frac{a_k}{k!}\!+\!\frac{0}{(k+1)!}
\in \mathcal{S}_{k+1}\) then,\newline
\(\frac{a_2}{2!}\!+\!\frac{a_3}{3!}\!+\!\dots{}\!+\!\frac{a_k}{k!}
\in \mathcal{S}_{k+1}\). Hence 
\(\mathcal{S}_k \subset \mathcal{S}_{k+1}\).

% Since each successive set's size grow larger, then
% by induction it is straightforward to conclude that
% \(\mathcal{S}_2
% \subset \mathcal{S}_3
% \subset \mathcal{S}_4
% \subset \dots{}
% \), for all \(\mathcal{S}_i\), \(i \ge{} 2\).

These infinite chains of super-sets helps us to think about the
meaning of our new sets
\(\mathcal{F}\) and \(\mathcal{S}\), each of which are defined as a union of 
an infinite sequence of \(\mathcal{F}_i\)'s and \(\mathcal{S}_i\)'s.

We can imagine that as
we build up either master-set by including each successive
\(\mathcal{F}_i\) or \(\mathcal{S}_i\), that either we're only adding
new elements to the ones that we've already gathered up, or we're simply tossing
the whole previous collection out, then using the entire contents of
the new largest set in its place. Either way of thinking about
building up our infinite union of sets is valid.

% For all \(a \in \mathcal{S}\), the factorial-representation for \(a\) is unique.
% This is fairly trivial to see, because by definition an element of a set is unique,
% and since no elements ever disappear from a union of sets, then it must be true.
% However to shed more light on the matter, 
So once a particular rational number \(\frac{p}{q} \in \mathbb{Q}_{01}\)
finds it's way into one of the sets \(\mathcal{S}_i\) for some \(i\), then it will forever
be in all successive sets \(\mathcal{S}_n\) for \(n > i\), and it's factorial-representation is
unique within each of those successively larger sets.

\section*{Factorial Representation Theorem}

Any positive rational number can be expressed in one and only one way in the form

{\small
\bgroup                                  %% open the group
\setlength{\abovedisplayskip}{0pt}       %% effective inside the group    
\begin{center}
\begin{equation*}
a_1 + \frac{a_2}{1\cdot{}2}
+ \frac{a_3}{1\cdot{}2\cdot{}3}
+ \dots{}
+ \frac{a_k}{1\cdot{}2\cdot{}3\cdot{}\dots{}\cdot{}k},
\end{equation*}
\end{center}
\egroup
}

where \(a_1, a_2, \dots{}, a_k\) are integers, and

{\small
\begin{center}
\(0\le{}a_1,\quad 0\le{}a_2<2,\quad 0\le{}a_3<3,\quad \dots{},\quad 0<a_k<k\)
\end{center}
}

\subsubsection*{Proof}

Thanks to Euclid we know that for all integers \(j\ge{}0\) and \(q>0\),
there exist \emph{unique} integers \(i\) and \(p\) such that,

{\normalsize
\bgroup                                  %% open the group
\setlength{\abovedisplayskip}{0pt}       %% effective inside the group
\begin{alignat*}{3}
&&j = i\cdot{}q + p\ &;\quad 0\le{}p<q \\
&\Leftrightarrow\quad &\frac{j}{q} = i + \frac{p}{q}\ &;\quad 0\le{}\frac{p}{q}<1
\end{alignat*}
\egroup
}\par
Which tells us that all positive rational numbers \(\frac{j}{q}\) can be
\emph{uniquely}
written as
an integer part, \(i\), plus a fractional part \(\frac{p}{q}\), where \(0\le{}\frac{p}{q}<1\).

Apply the Euclidean Division Theorem to \(\frac{j}{q}\) and let \(a_1 = i\). If there is no fractional remainder, then
the theorem has been proven.

When there is a non-zero fractional remainder \(\frac{p}{q}\), then 
by the Corollary to Lemma 4 we know that there is a
unique factorial-representation for \(\frac{p}{q}\).

So \(\frac{p}{q} = \frac{a_2}{2!} + \frac{a_3}{3!} + \dots{} + \frac{a_n}{n!} \in \mathcal{S}_n\), for some
\(n \ge{} 2\), and this sum is uniquely associated with \(\frac{p}{q}\).
If we choose \(k\) such that \(a_k \ne{} 0 \) but \(a_{k+1} =
a_{k+2} =
a_{k+3} =
\dots{} =
a_{n-1} =
a_{n} = 0\) then we can satisfy the condition that the last term in the sum is non-zero, and hence:

\(a_1 + \frac{a_2}{2!} + \frac{a_3}{3!} + \dots{} + \frac{a_k}{k!}\)
is uniquely associated with all positive rational numbers \(\frac{j}{q}\).

QED.

\break
\section*{Additional Observations}

We're guaranteed that \(\frac{p}{q} \in{} \mathcal{S}_q\), but \(\mathcal{S}_q\) is not necessarily 
the smallest such set for which \(\frac{p}{q}\) is a member.

For example,
the smallest set containing \(\frac{p}{5}\), where \(0 \le{} \frac{p}{5} < 1\), is \(\mathcal{S}_5\)
however
the smallest set containing \(\frac{p}{6}\), where \(0 \le{} \frac{p}{6} < 1\) is \(\mathcal{S}_3\),
which is easy to see when we list the contents of a couple of sets,
%  1,  2,  3,  4,  5,  6,  7,  8,  9, 10, 11, 12, 13, 14, 15, 16, 17, 18, 19 20, 21, 22, 23
% 24, 24, ...
%  1   1   1   1   5   1   7   1   3   5  11   1  13   7   5   2  17   3  19  5   7  11  23
% 24  12   8   6  24   4  24   3   8  12  24   2  24  12   8   3  24   4  24  6   8  12  24

{\footnotesize
\bgroup                                  %% open the group
\setlength{\abovedisplayskip}{0pt}
\begin{alignat*}{1}
\mathcal{S}_4 &= \{
\frac{ 0}{24},
\frac{ 1}{24},
\frac{ 2}{24},
\frac{ 3}{24},
\frac{ 4}{24},
\frac{ 5}{24},
\frac{ 6}{24},
\frac{ 7}{24},
\frac{ 8}{24},
\frac{ 9}{24},
\frac{10}{24},
\frac{11}{24},
\frac{12}{24},
\frac{13}{24},
\frac{14}{24},
\frac{15}{24},
\frac{16}{24},
\frac{17}{24},
\frac{18}{24},
\frac{19}{24},
\frac{20}{24},
\frac{21}{24},
\frac{22}{24},
\frac{23}{24}
\} \\
&= \{
\frac{ 0}{24},
\frac{ 1}{24},
\frac{ 1}{12},
\frac{ 1}{ 8},
\frac{ 1}{ 6},
\frac{ 5}{24},
\frac{ 1}{ 4},
\frac{ 7}{24},
\frac{ 1}{ 3},
\frac{ 3}{ 8},
\frac{ 5}{12},
\frac{11}{24},
\frac{ 1}{ 2},
\frac{13}{24},
\frac{ 7}{12},
\frac{ 5}{ 8},
\frac{ 2}{ 3},
\frac{17}{24},
\frac{ 3}{ 4},
\frac{19}{24},
\frac{ 5}{ 6},
\frac{ 7}{ 8},
\frac{11}{12},
\frac{23}{24}
\}
\end{alignat*}
\egroup
}

By examination \(\mathcal{S}_4\) doesn't contain \(\frac{1}{5}\), but it's
definitely in \(\mathcal{S}_5\) because,

{\small
\bgroup                                  %% open the group
\setlength{\abovedisplayskip}{0pt}       %% effective inside the group    
\begin{center}
\begin{equation*}
% \frac{1}{5} = 
\frac{0}{2}
+ \frac{1}{2\cdot{}3}
+ \frac{0}{2\cdot{}3\cdot{}4}
+ \frac{4}{2\cdot{}3\cdot{}4\cdot{}5}
= \frac{1}{6} + \frac{1}{30}
= \frac{5 + 1}{30}
= \frac{6}{30}
= \frac{1}{5}
\end{equation*}
\end{center}
\egroup
}

Also, \(
\mathcal{S}_3
= \{
\frac{ 0}{6},
\frac{ 1}{6},
\frac{ 2}{6},
\frac{ 3}{6},
\frac{ 4}{6},
\frac{ 5}{6}
\}
% = \{
% \frac{ 0}{6},
% \frac{ 1}{6},
% \frac{ 1}{3},
% \frac{ 1}{2},
% \frac{ 2}{3},
% \frac{ 5}{6}
% \}
\),
which demonstrates the claim that \(\mathcal{S}_3\)
contains \(\frac{p}{6}\),\newline where \(0 \le{} \frac{p}{6} < 1\).

I believe that for a given \(q \ge 2\) the smallest set \(\mathcal{S}_k\) for which \(\frac{p}{q}\)
is a member
will be found by choosing the smallest
\(k\) such that \(q\) divides \(k!\)\! .

However, I'll leave that proof for another day.

A final thought is that the number 
\(2\!+\!\frac{1}{2!}\!+\!\frac{1}{3!}\!+\!\frac{1}{4!}\!+\!\frac{1}{5!}\!+\!\dots{}\) can't
be a rational number, as its fractional-part is not in \(\mathcal{S}\)
because we can't point to any particular \(\mathcal{S}_k\) that the fractional-part would be a member of.
Not surprising really, since this number is the well known constant \(e\).
Furthermore I think we just proved that \(e\) is irrational!!

\end{document}
