\documentclass{article}

\usepackage[utf8]{inputenc} % set input encoding (not needed with XeLaTeX)

%%% PAGE DIMENSIONS
\usepackage{geometry} % to change the page dimensions
\geometry{letterpaper} % or letterpaper (US) or a5paper or....
\geometry{margin=1.35in} % for example, change the margins to 2 inches all round

\usepackage{graphicx} % support the \includegraphics command and options

\usepackage[parfill]{parskip} % Activate to begin paragraphs with an empty line rather than an indent

%%% PACKAGES
\usepackage{booktabs} % for much better looking tables
\usepackage{array} % for better arrays (eg matrices) in maths
\usepackage{paralist} % very flexible & customisable lists (eg. enumerate/itemize, etc.)
\usepackage{verbatim} % adds environment for commenting out blocks of text & for better verbatim
\usepackage{subfig} % make it possible to include more than one captioned figure/table in a single float
% These packages are all incorporated in the memoir class to one degree or another...

%%% HEADERS & FOOTERS
\usepackage{fancyhdr} % This should be set AFTER setting up the page geometry
\pagestyle{fancy} % options: empty , plain , fancy/
\renewcommand{\headrulewidth}{0pt} % customise the layout...
\lhead{}\chead{}\rhead{}
\lfoot{}\cfoot{\thepage}\rfoot{}

%%% SECTION TITLE APPEARANCE
\usepackage{sectsty}
%\allsectionsfont{\sffamily\mdseries\upshape} % (See the fntguide.pdf for font help)
% (This matches ConTeXt defaults)

%%% ToC (table of contents) APPEARANCE
\usepackage[nottoc,notlof,notlot]{tocbibind} % Put the bibliography in the ToC
\usepackage[titles,subfigure]{tocloft} % Alter the style of the Table of Contents
\renewcommand{\cftsecfont}{\rmfamily\mdseries\upshape}
\renewcommand{\cftsecpagefont}{\rmfamily\mdseries\upshape} % No bold!

% JPR added
\usepackage{fontawesome}
\usepackage{amsfonts}
%\usepackage{amsmath}
\usepackage{mathtools}% includes amsmath
\usepackage{changepage}
\usepackage{enumerate}
%\usepackage{setspace}
\usepackage{relsize}
\usepackage{wasysym}
%\usepackage{romannum}
 
\usepackage[pdftex,
            pdfauthor={James Philip Rowell},
            pdftitle={\jobname},
            pdfsubject={Proof of Chapter 1, Miscellaneous Examples, Number 2 from G.H. Hardy `A Course of Pure Mathematics'.},
            pdfkeywords={rational numbers, theorem, proof, mathematics, number theory, factorial, factoradic, Hardy, Examples, A Course of Pure Mathematics},
            pdfproducer={Latex},
            pdfcreator={miktex or pdflatex}]{hyperref}
\hypersetup{
    colorlinks=true,
    linkcolor=black,
    filecolor=magenta,      
    urlcolor=blue,
}
\usepackage{hyperxmp}
\hypersetup{
    pdfauthor={James Philip Rowell},
    pdfcopyright={Copyright  2018 by James Philip Rowell. All rights reserved.}
}
\usepackage{lipsum}

\newenvironment{jprIn}{\begin{adjustwidth}{2em}{}}{\end{adjustwidth}}
\addtolength{\skip\footins}{6pt}

\usepackage{alphalph}
\makeatletter
\newalphalph{\fnsymbolwrap}[wrap]{\@fnsymbol}{}
\makeatother
\renewcommand*{\thefootnote}{%
  \fnsymbolwrap{\value{footnote}}%
}

\usepackage{perpage}
\MakePerPage{footnote}

\DeclarePairedDelimiter\abs{\lvert}{\rvert}

%%% END Article customizations

\author{James Philip Rowell}
\title{\vspace{-1.5cm}Factoradic Representation of Rational Numbers}
\date{} % Activate to display a given date or no date (if empty), otherwise the current date is printed 
\begin{document}
\maketitle
\begin{em}
\centerline{\small{}From `A Course in Pure Mathematics' by G. H. Hardy. Chapter 1, Miscellaneous Examples.}
%\par
%\setlength{\parskip}{0pt}
\end{em}
%\normalsize
\bigskip

Miscellaneous example\footnote{Hardy doesn't call them `Exercises' or `Questions', but that's what they
are, math exercises like calculations to perform, theorems to prove etc.} \#2 at the end
of chapter 1 in Hardy's `Pure Mathematics'
presents us with a fascinating result (which was new to me).
It feels like
a kind of basis-representation-theorem, but for rational numbers, \dots{} beautiful!

Here it is, followed by my proof which starts out with some lemmas to get us rolling.

\section*{Theorem}

Any positive rational number can be expressed in one and only one way in the form
{\small
\bgroup                                  %% open the group
\setlength{\abovedisplayskip}{0pt}       %% effective inside the group    
\begin{center}
\begin{equation*}
a_1 + \frac{a_2}{1\cdot{}2}
+ \frac{a_3}{1\cdot{}2\cdot{}3}
+ \dots{}
+ \frac{a_k}{1\cdot{}2\cdot{}3\cdot{}\dots{}\cdot{}k},
\end{equation*}
\end{center}
\egroup
}

where $a_1, a_2, \dots{}, a_k$ are integers, and

\begin{center}
\(0\le{}a_1,\quad 0\le{}a_2<2,\quad 0\le{}a_3<3,\quad \dots{},\quad 0<a_k<k\)
\end{center}

\section*{Lemma-1}
{\small
\bgroup                                  %% open the group
\setlength{\abovedisplayskip}{0pt}       %% effective inside the group    
\begin{center}
\begin{equation*}
\frac{1}{2!} + \frac{2}{3!} + \dots{} + \frac{k-1}{k!} = \frac{k!-1}{k!}
\end{equation*}
\end{center}
\egroup
}
\section*{Proof of Lemma-1}

This equality is fairly trivial to demonstrate by induction, since $\frac{1}{2!} = \frac{2!-1}{2!}$ and, 
{\footnotesize
\begin{alignat*}{1}
  &\frac{1}{2!} + \frac{2}{3!} + \dots{} + \frac{k-2}{(k-1)!} + \frac{k-1}{k!}\\
  =\  &\frac{(k-1)!-1}{(k-1)!} + \frac{k-1}{k!}\\
  =\  &\frac{k((k-1)!-1)}{k(k-1)!} + \frac{k-1}{k!}\\
  =\  &\frac{k!-k+k-1}{k!}\\
  =\  &\frac{k!-1}{k!}
\end{alignat*}
}
\dots{}thus establishing lemma-1 for all values of k.

\section*{Lemma-2}

For integers $i,k$ where $2\le{}i<k$ such that,
{\small
\bgroup                                  %% open the group
\setlength{\abovedisplayskip}{0pt}       %% effective inside the group    
\begin{center}
\begin{equation*}
\frac{1}{2!} + \frac{2}{3!}
+ \dots{}
+ \frac{i-1}{i!}
+ \frac{i}{(i+1)!}
+ \dots{}
+ \frac{k-1}{k!},
\end{equation*}
\end{center}
\egroup
}
then
{\small
\bgroup                                  %% open the group
\setlength{\abovedisplayskip}{0pt}       %% effective inside the group    
\begin{center}
\begin{equation*}
\frac{1}{i!} - \frac{1}{k!}
= \frac{i}{(i+1)!}
+ \dots{}
+ \frac{k-1}{k!}
\end{equation*}
\end{center}
\egroup
}
\section*{Proof of Lemma-2}
{\small
\bgroup
\setlength{\abovedisplayskip}{0pt}%
\setlength{\belowdisplayskip}{0pt}%
\begin{alignat*}{1}
  &\frac{i}{(i+1)!} + \dots{} + \frac{k-1}{k!}\\
  =\  &(\frac{1}{2!} + \frac{2}{3!} + \dots{} + \frac{k-1}{k!})
  - (\frac{1}{2!} + \frac{2}{3!} + \dots{} + \frac{i-1}{i!})\\
  =\  &\frac{k!-1}{k!} - \frac{i!-1}{i!}\\
  =\  &\frac{k!}{k!} - \frac{1}{k!} - \frac{i!}{i!} + \frac{1}{i!}\\
  =\  &\frac{1}{i!} - \frac{1}{k!}
\end{alignat*}%
\egroup}
\section*{Lemma-3}

The set of rational numbers,

\centerline{$\mathcal{S} = \{\frac{a_2}{2!} + \frac{a_3}{3!} + \dots{} + \frac{a_k}{k!}\enspace\mid\enspace 
0\! \le{}\! a_2\! <\! 2,\enspace 
0\! \le{}\! a_3\! <\! 3,\enspace 
\dots{},\enspace 0\! \le{}\! a_k\! <\! k\}$,}
\bigskip
is identical to the set of rational numbers,\newline

\centerline{$\mathcal{F} = \{\frac{0}{k!},\enspace \frac{1}{k!},\enspace \frac{2}{k!},\enspace \dots{},\enspace \frac{k!-1}{k!}\}$}

\section*{Proof of Lemma-3}

Now we make note of the fact that the smallest member of the set $\mathcal{S}$ 
occurs when all the coefficients of the sum are zero, i.e.; $\frac{0}{k!}$.
Furthermore, the largest member of the set occurs when all the coefficients are set to
their maximum value, which we have just seen gives us $\frac{k!-1}{k!}$.

We also note that every members of $\mathcal{S}$ can be written as a rational number
with $k!$ as the denominator, like so:\newline

\centerline{$\frac{a_2}{2!} + \frac{a_3}{3!} + \dots{} + \frac{a_k}{k!}
= \frac{k\cdot{}(k-1)\cdot{}\cdot{}\cdot{}3\cdot{}a_2}{k!}
+ \frac{k\cdot{}(k-1)\cdot{}\cdot{}\cdot{}4\cdot{}a_2}{k!}
+ \dots{}
+ \frac{a_k}{k!}$}

Also when $i<k$ such that,

\[\frac{1}{2!} + \frac{2}{3!}
+ \dots{}
+ \frac{i-1}{i!}
+ \frac{i}{(i+1)!}
+ \dots{}
+ \frac{k-1}{k!}
\]

Then we can conclude that,

\[\frac{1}{i!} - \frac{1}{k!}
= \frac{i}{(i+1)!}
+ \dots{}
+ \frac{k-1}{k!}
\]

Because,

From here we can deduce that any assignment
of values to the coefficients of a member of $\mathcal{S}$ produces a unique member
of the set, for if it didn't and
$\frac{p}{q} = \frac{a_2}{2!} + \frac{a_3}{3!} + \dots{} + \frac{a_k}{k!}$ is not uniquely 
determined by the coefficients
$a_2, a_3, \dots{}, a_k$. That is, suppose there is a second DIFFERENT sequence of coefficients
$b_2, b_3, \dots{}, b_k$ such that
$\frac{p}{q} = \frac{b_2}{2!} + \frac{b_3}{3!} + \dots{} + \frac{b_k}{k!}$.

The number of values that the coefficient $a_2$ can assume is 2, $a_3$ can take on 3 values, \dots{},
up to $a_k$ which can take on $k$ values. So the total number of combinations of values
that can be assigned to all the coefficients is $2\cdot{}3\cdot{}4\cdot\dots{}\cdot{}k = k!\,$.



The size of $\mathcal{F}$ is clearly $k!$

\end{document}
