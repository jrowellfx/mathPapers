\documentclass{article}

\usepackage[utf8]{inputenc} % set input encoding (not needed with XeLaTeX)

%%% PAGE DIMENSIONS
\usepackage{geometry} % to change the page dimensions
\geometry{letterpaper} % or letterpaper (US) or a5paper or....
\geometry{margin=1.35in} % for example, change the margins to 2 inches all round

\usepackage{graphicx} % support the \includegraphics command and options

\usepackage[parfill]{parskip} % Activate to begin paragraphs with an empty line rather than an indent

%%% PACKAGES
\usepackage{booktabs} % for much better looking tables
\usepackage{array} % for better arrays (eg matrices) in maths
\usepackage{paralist} % very flexible & customisable lists (eg. enumerate/itemize, etc.)
\usepackage{verbatim} % adds environment for commenting out blocks of text & for better verbatim
\usepackage{subfig} % make it possible to include more than one captioned figure/table in a single float
% These packages are all incorporated in the memoir class to one degree or another...

%%% HEADERS & FOOTERS
\usepackage{fancyhdr} % This should be set AFTER setting up the page geometry
\pagestyle{fancy} % options: empty , plain , fancy/
\renewcommand{\headrulewidth}{0pt} % customise the layout...
\lhead{}\chead{}\rhead{}
\lfoot{}\cfoot{\thepage}\rfoot{}

%%% SECTION TITLE APPEARANCE
\usepackage{sectsty}
%\allsectionsfont{\sffamily\mdseries\upshape} % (See the fntguide.pdf for font help)
% (This matches ConTeXt defaults)

%%% ToC (table of contents) APPEARANCE
\usepackage[nottoc,notlof,notlot]{tocbibind} % Put the bibliography in the ToC
\usepackage[titles,subfigure]{tocloft} % Alter the style of the Table of Contents
\renewcommand{\cftsecfont}{\rmfamily\mdseries\upshape}
\renewcommand{\cftsecpagefont}{\rmfamily\mdseries\upshape} % No bold!

% JPR added
\usepackage{fontawesome}
\usepackage{amsfonts}
\usepackage{amssymb}
%\usepackage{amsmath}
\usepackage{mathtools}% includes amsmath
\usepackage{changepage}
\usepackage{enumerate}
%\usepackage{setspace}
\usepackage{relsize}
\usepackage{wasysym}
%\usepackage{romannum}
 
\usepackage[pdftex,
            pdfauthor={James Rowell},
            pdftitle={\jobname},
            pdfsubject={Magic Numbers},
            pdfkeywords={decimal, base-ten, binary, base-two, integers,
	    theorem, proof, mathematics, number-theory, divisors, magic number},
            pdfproducer={Latex},
            pdfcreator={miktex or lualatex}]{hyperref}
\hypersetup{
    colorlinks=true,
    linkcolor=black,
    filecolor=magenta,      
    urlcolor=blue,
}
\usepackage{hyperxmp}
\hypersetup{
    pdfauthor={James Philip Rowell},
    pdfcopyright={Copyright  2017 by James Philip Rowell. All rights reserved.}
}
\usepackage{lipsum}

\newenvironment{jprIn}{\begin{adjustwidth}{2em}{}}{\end{adjustwidth}}
\addtolength{\skip\footins}{6pt}

\usepackage{alphalph}
\makeatletter
\newalphalph{\fnsymbolwrap}[wrap]{\@fnsymbol}{}
\makeatother
\renewcommand*{\thefootnote}{%
  \fnsymbolwrap{\value{footnote}}%
}

\usepackage{perpage}
\MakePerPage{footnote}

%%% END Article customizations

\title{Magic Numbers}
\author{James Rowell}
%\date{} % Activate to display a given date or no date (if empty),
% otherwise the current date is printed 

\begin{document}
\maketitle
%\bigskip

In the delightful film ``\href{https://en.wikipedia.org/wiki/School_of_Rock}{School
of Rock}'' Jack Black's character Dewey Finn,
pretending to be substitute teacher Ned Schneebly, is put on the spot when
Miss. Mullins, the school principle, comes into class and demands that he show
her his teaching methods which apparently involve an electric guitar.
So Mr. S sings ``\href{https://www.youtube.com/watch?v=aa8U0nL-KXg}{The
Math Song}'' wherein the final line of his
improvised tune he sings ``\dots{}yes it's 9, and that's a magic-number\dots{}''

It seems that Jack Black was making a
tip-of-the-hat to the ``Schoolhouse Rock'' song ``\href{https://youtu.be/aU4pyiB-kq0}{3
Is A Magic Number}''
which most kids from the 70's/80's know.
However referring to 9 as a magic-number isn't
as well known\footnote{According to my own \emph{highly scientific}
informal poll conducted Facebook, %\today{}
``9'' is NOT the first thing that pops into people's heads when they
hear the phrase ``magic number'' - in fact no one answered ``9'' in my poll.}.

However it's not a total stretch, there is precedent
for using the term ``\href{http://mathworld.wolfram.com/MagicNumber.html}{Magic
Number}'' when referring to to the number 9.
There are a host of
\href{http://mathematics-in-europe.eu/?p=144}{mathematical-magic-tricks}
that rely on properties of the number 9.
There is also a process called
``\href{https://en.wikipedia.org/wiki/Casting_out_nines}{casting out nines}'' and a related one 
called calculating the
``\href{http://mathworld.wolfram.com/DigitalRoot.html}{digital root}''
for a number where you take its ``digital sum''. 9
plays a similar role in all these processes - so within these contexts sometimes 9
is referred to as a magic-number.

So Mr. S's musical-math-lesson is legit, if indeed he is referring to this fact about 9:
If you add up the digits of
any number then if that sum is divisible by 9, then the original number
must be divisible by 9. It works the other way too, that is, if we know
that a number is divisible by 9 then the sum of its digits must also be
divisible by 9.

For example:
\begin{jprIn}
$9^5=9\times{}9\times{}9\times{}9\times{}9=59049$
and $5+9+0+4+9=27$ and 9 divides 27 because $27 = 3\times{}9$.
\end{jprIn}

But 3 is also has this magic-number property, for example:
\[ 3\times{}67=201 \text{ and } 2+0+1=3\]
In this case, 9 does not divide 201\footnote{Why is it immediately
apparent that 9 does not divide 201? Hint: both 3 and 67 are prime numbers.
Check the ``\href{https://en.wikipedia.org/wiki/Fundamental_theorem_of_arithmetic}{Fundamental
Theorem of Arithmetic}'' wiki page for further insight.}
but 3 does (67 times).
We can see that adding up the digits of 201 gives us 3,
which of course is divisible by 3 (but not by 9).

So given that the term ``magic number'' hasn't got
a formal definition in the math world, I think it's time to give it one.
How about if we call ANY number that has qualities like 3 and 9 (as 
described above) a magic-number?
After all both ``Schoolhouse Rock'' AND ``The School of Rock'' sanction using
the term ``magic number'' for 3 and 9, so we may be on to something big here!

Too bad 3 and 9 are the only magic-numbers in base-10.
Darn - perhaps our new term isn't going to be all that useful after all, \dots{} or 
maybe it will!
We happen to use base-10 because we have ten fingers\footnote{To be honest, base-10 is
not ``universal'' among humans, there are plenty of
\href{https://www.youtube.com/watch?v=l4bmZ1gRqCc}{other bases} in use, albeit not
as widely used nor as mathematically useful as base-10. This doesn't count
octal, hexadecimal and binary which are used in the computer world
so by necessity are, mathematically speaking, useful.} but what about other bases?
The heptapods in the film ``Arrival'' have 7 limbs, 3 arms and 4 legs, with 7 fingers/toes each.
I'm going to guess that at least at some point in their history they used positional
notation like base-10, but instead of 10 they probably base-7 or base-21,
maybe even base-49. So perhaps an understanding of magic-numbers in alternate
bases could come in handy.

Let's examine some magic-numbers
%(according to our proposed usage of the term)
in other bases.
In the way that 9 is a magic-number in base-10,
then 4 is a magic-number in base-5;
Also 7 is a magic-number in base-8;
15 is a magic-number in base-16;
\dots{}and 30 is a magic-number in base-31, as you can see with these
examples\footnote{We are going to use the conventions for specifying
numbers in alternate bases as outlined in the paper
``\href
{https://dl.dropboxusercontent.com/u/2157321/basisReprThm.pdf}
{SesameStreet++}''}:
\begin{itemize}
\item $4\times{}67=268$ written in base-5 is $(2033)_5$ and $2+0+3+3=8$ (4 divides 8).
\item $7\times{}181=1267$ written in base-8 is $(2363)_8$ and $2+3+6+3=14$ (7 divides 14).
\item $15\times{}23=345$ written in base-16 is $(159)_{16}$ and $1+5+9=15$.
\item $30\times{}35951=1078530$ written in base-31 is $(15699)_{31}$ and $1+5+6+9+9=30$
\end{itemize}

Notice the pattern: the biggest single-digit in any base
(which is always one less than the base) seems to
qualify as a magic-number. For example,
4 is the biggest digit in base-5,
7 is the biggest digit in base-8,
and 9 is the biggest digit in base-10;
To state this in a general way
we could say that if $b$ is our base, then $(b-1)$ is a magic-number in base-$b$.

So what about 3 also being a magic-number in base-10?
Let's take a look at the interesting (but probably never used outside this paper) base-31.
It appears that 2, 3, 5, 6, 10, 15 (along with 30)
are \emph{all} magic-numbers in base-31, as we we can see
with the following examples:
\begin{itemize}
\item $2\times{}409\times{}1129=923522$ written in base-31 is $(10001)_{31}$ and $1+0+0+0+1=2$.
\item $3\times{}641=1923$ written in base-31 is $(201)_{31}$ and $2+0+1=3$.
\item $5\times{}139\times{}2659=1848005$ written in base-31 is $(20102)_{31}$ and $2+0+1+0+2=5$.
\item $6\times{}10091=60546$ written in base-31 is $(2103)_{31}$ and $2+1+0+3=6$.
\item $10\times{}197\times{}941=1853770$ written in base-31 is $(20701)_{31}$ and $2+0+7+0+1=10$.
\item $15\times{}71503=1072545$ written in base-31 is $(15027)_{31}$ and $1+5+0+2+7=15$.
\end{itemize}

See the pattern? All the divisors of 30 are magic-numbers in base-31.
It turns out that in base-$b$, then any divisor of $(b-1)$ is a magic-number in base-$b$.

I guess that begs the question: What about 1? Isn't 1 a magic-number in all bases?
Since we already know that all numbers are divisible by 1,
then we don't need to go to any trouble of adding up digits to find this out.
So let's exclude 1 from being considered as a magic-number - it's not helpful.

\break
Let's quickly review what it means for a number to be divisable 
by another number.  Recall from the paper ``\href
{https://dl.dropboxusercontent.com/u/2157321/basisReprThm.pdf}
{SesameStreet++}'':

\section*{Euclidean Division Theorem}
\begin{jprIn}
For all integers $a$ and $b$ such that $b\ne0$,
there exist \emph{unique} integers $q$ and $r$ such that:
\[a=qb+r  \text{ such that } 0\le{}r<\lvert{}b\rvert\]
Definition: In the above equation\footnote{Recall
that $\lvert{}b\rvert$ means the ``absolute value'' of $b$ and is always positive. For
example $\lvert{}{-13}\rvert = 13$.}:
\begin{jprIn}
\begin{tabular}{l l}
a is the \emph{dividend} & (``the number being divided'')\\
b is the \emph{divisor} & (``the number doing the dividing'')\\
q is the \emph{quotient} & (``the result of the division'')\\
r is the \emph{remainder} & (``the leftover'')
\end{tabular}
\end{jprIn}
\end{jprIn}

So to say that a number $a$ is divisible by $b$ simply means that $r$ is zero.

In other words, to say an integer $a$ is divisible by another integer $b$
(not zero) means there is a third integer $q$ such that\footnote{Paraphrased
from ``An Introduction to the Theory of Numbers" by G. H. Hardy
and E. M. Wright. Pg 1.}:
\[a=q\cdot{}b\text{\hspace{2.5em}or\hspace{2.5em}}\frac{a}{b}=q\]
We express the fact that $a$ is divisible by $b$, and say
``$b$ divides $a$'' using this notation:
\[b\mid{}a\]
So the following is always true:
\[1\mid{}a,
\text{\hspace{0.5em}and\hspace{0.5em}}a\mid{}a;\]
also
\[b\mid{}0\text{ for every }b\text{ except }0\]
%We sometimes use the following symbol to say that ``$b$ does not divide $a$'',
%\[b\nmid{}a\]
%
%Also it should be plain that:
%\[\text{If }c\mid{}b\text{ and }b\mid{}a\text{, then }c\mid{}a\]
%\[b\mid{}a\text{ implies }bc\mid{}ac\text{, when }c\ne0\]
%And lastly check for yourself that this is always true:
%\[\text{If }c\mid{}a\text{ and }c\mid{}b\text{, then }c\mid{}ma+nb\text{, for all integers }m,n.\]

In order to create our therorem with a 
nice definition for ``magic number''
we need to remind ourselves 
what it means to write a number in a given base\footnote{Also from
``\href
{https://dl.dropboxusercontent.com/u/2157321/basisReprThm.pdf}
{SesameStreet++}''}:

\section*{Basis Representation Theorem}
\begin{jprIn}
Let $b$ be a positive integer greater than 1.

For every positive integer $n$ there is a unique sequence
of integers $d_0, d_1, d_2,\dots{},d_k$ such that:

\hspace{3em}$n=d_kb^k+d_{k-1}b^{k-1}+\dots+d_2b^2+d_1b^1+d_0b^0$,

where $0\le{}d_i<b$ for all $i$ in $\{0,1,2,\dots{},k\}$ and $d_k\ne0$.

Definition: $n$ is represented in base-$b$ by the string
of base-$b$-digits $(d_kd_{k-1}{\cdots}d_2d_1d_0)_b$
\end{jprIn}
%\bigskip
%So given the Basis-Representation-Theorem and our new notation for divisability
%then here's our
%proposed theorem and definition for ``magic number'':
%\footnote{Recall that the double-arrow, that
%is $\Leftrightarrow$, should be read as ``if and only if''.}
%
\section*{Magic Number Theorem}
\begin{jprIn}
Let $n$ be a positive integer represented in base-$b$.

If $m$ is a positive integer such that $m\mid{}(b-1)$ then,
%\[m\mid{}n \Leftrightarrow m\mid{}(d_k+\dots+d_2+d_1+d_0)\]
\[m\mid{}n \text{\hspace{2.5em}if and only if\hspace{2.5em}}m\mid{}(d_k+\dots+d_2+d_1+d_0)\]
Definition: We call $m\ne1$ a ``magic-number'' in base-$b$.
\end{jprIn}
\bigskip

As an interesting side note - according to our definition, binary,
or base-2, doesn't have a magic-number.
No great loss, as we said above 1 isn't helpful to consider as a magic-number. 

Let's prove the theorem. All we need to do is show that if you divide $n$ by $m$
that the remainder of the division is always zero.

The proof is quite simple when using 
modular arithmetic. However if you aren't familiar with modular arithmetic
then I recommend that you check out Khan Academy's
``\href{https://www.khanacademy.org/computing/computer-science/cryptography/modarithmetic/a/what-is-modular-arithmetic}{What is modular arithmetic?}'' to
get a clear-cut-crash-course in the subject, it's excellent, as is the 
entire Khan Academy site in case you've never checked it out.

%In summary, modular arithmetic is the kind of arithmetic you use when
%you tell time with a 12 hour clock.
%
%For example, if it's 7 o'clock and I ask you to ``meet me in 7 hours"
%then you know I mean ``meet me at 2 o'clock".  Suppose instead I said ``meet me in 31 hours"
%then you would know that our meeting time
%is also going to be 2 o'clock (albeit 1 day hence) because:
%\[7\text{ o'clock}+ 31\text{ hours}= 7\text{ o'clock}+ (7 + 24)\text{ hours}= 2\text{ o'clock}\]
%When we do ``clock arithmetic" we add some
%number of hours to the current time then divide that sum by 12;
%The remainder of the division then
%tells us the clock time.
%
%Any time on the clock, say $T$ o'clock, plus 12 hours is also $T$ o'clock.  Furthermore,
%adding ANY multiple of 12 hours (eg 24, 36, 48, \dots{}) to $T$ o'clock gets us back to $T$ o'clock.
%We can say that $T, T{+}12, T{+}24, T{+}36, T{+}48, \dots{}$ are all equivalent times
%on the clock, or that these numbers are all equivalent modulo 12. It just means the hour hand
%has spun around the clock exactly 1, 2, or 3, etc.\ more times.
%
%So numbers being equivalent modulo 12 (i.e. some number of rotations
%of the hour hand comes back to the same time), means that they have the same
%remainder when divided by 12. For example 1 is equivalent to 13 modulo 12.
%If your drill sargent barks an order at you to meet him at 13:00 hours,
%you'd better show up at 1 o'clock in the afternoon!

Assuming you're now familiar with modular-arithmetic you understand
that it gives us very powerful tools to deal with divisiability
and remainders of integers.
To say that 
the remainder of
dividing $n$ by $m$
is zero is the same as saying:
\[n \equiv 0 \pmod{m}\]



where $a,b$ and $c$ are
integers such that $c>1$.

(a+b) mod c=(a mod c+b mod c) mod c
ab mod c=((a mod c)(b mod c)) mod c

As a generalization of 2), we get:

3) a mod c=b mod c akmod c=bk mod c

Importantly, a mod c=0 c divides a.

In other words, there exists a number n such that nc=a.
Lemma:
if d>1 is a divisor of m then:
(m+1)k mod d=1
Proof of Lemma:
Let d be a divisor of m. Soc such that cd=m.
Hence, (cd + 1) mod d = cd mod d + 1 mod d
= (c mod d) (d mod d) + 1 mod d
= (c mod d) 0 + 1 mod d
= 1 mod d
By property 3) above, since (cd+1) mod d=1 mod d then:
(m + 1)k mod d
= (cd + 1)k mod d
= (cd+1) mod d
= 1
QED

Proof of ``Magic Number Theorem'':

Let N=M+1.

That is; we are going to consider representations of numbers in base-M+1, so think of M as our magic-number.

Also let M=1, so our smallest possible base will be 2.
Then every integer n is uniquely expressible in base-M+1 as follows:
n=ak(M+1)k+ak-1(M+1)k-1+?+a2(M+1)2+a1(M+1)+a0
where 0=ai=M for all the ``digits'' ai where 0=i=k.
n is divisible by M if and only if n mod M=0. 
Substitute the base-M+1 expression of n into the above relationship, therefor
0=(ak(M+1)k+ak-1(M+1)k-1+?+a2(M+1)2+a1(M+1)+a0)mod M


%Anytime I want to check to see if a number is divisible
%by 3 (but don't feel like busting out a calculator)
%I use this little trick of checking to see if the digits add up to
%some multiple of 3. Along with the more obvious tricks to see if the number is
%divisible by 2 (last digit is even) or is divisible 
%by 5 (last digit is 0 or 5) - I can decide if I
%can factor out 2, 3 or 5 from any number pretty quickly.
%7 takes real work \dots{}but I digress.


\end{document}
