\documentclass{article}

\usepackage[utf8]{inputenc} % set input encoding (not needed with XeLaTeX)
\usepackage{lmodern}

%%% PAGE DIMENSIONS
\usepackage{geometry} % to change the page dimensions
\geometry{letterpaper} % or letterpaper (US) or a5paper or....
\geometry{margin=1.35in} % for example, change the margins to 2 inches all round

\usepackage{graphicx} % support the \includegraphics command and options

\usepackage[parfill]{parskip} % Activate to begin paragraphs with an empty line rather than an indent

%%% PACKAGES
\usepackage{booktabs} % for much better looking tables
\usepackage{array} % for better arrays (eg matrices) in maths
\usepackage{paralist} % very flexible & customisable lists (eg. enumerate/itemize, etc.)
\usepackage{verbatim} % adds environment for commenting out blocks of text & for better verbatim
\usepackage{subfig} % make it possible to include more than one captioned figure/table in a single float
% These packages are all incorporated in the memoir class to one degree or another...

%%% HEADERS & FOOTERS
\usepackage{fancyhdr} % This should be set AFTER setting up the page geometry
\pagestyle{fancy} % options: empty , plain , fancy/
\renewcommand{\headrulewidth}{0pt} % customise the layout...
\lhead{}\chead{}\rhead{}
\lfoot{}\cfoot{\thepage}\rfoot{}

%%% SECTION TITLE APPEARANCE
\usepackage{sectsty}
\allsectionsfont{\mdseries} % (See the fntguide.pdf for font help)
% (This matches ConTeXt defaults)

%%% ToC (table of contents) APPEARANCE
\usepackage[nottoc,notlof,notlot]{tocbibind} % Put the bibliography in the ToC
\usepackage[titles,subfigure]{tocloft} % Alter the style of the Table of Contents
\renewcommand{\cftsecfont}{\rmfamily\mdseries\upshape}
\renewcommand{\cftsecpagefont}{\rmfamily\mdseries\upshape} % No bold!

% JPR added
\usepackage{fontawesome}
\usepackage{amsfonts}
%\usepackage{amsmath}
\usepackage{mathtools}% includes amsmath
\usepackage{changepage}
\usepackage{enumerate}
%\usepackage{setspace}
\usepackage{relsize}
\usepackage{wasysym}
%\usepackage{romannum}
\usepackage{perpage}
\usepackage[bottom]{footmisc}
\MakePerPage{footnote}

\newcommand{\jprVersion}{v09} % version
 
\usepackage[pdftex,
            pdfauthor={James Philip Rowell},
            pdftitle={\jobname.\jprVersion},
            pdfsubject={Proof of Chapter 1, Section 5, Examples III, Number 3 from G.H. Hardy `A Course of Pure Mathematics'.},
            pdfkeywords={approximation, square root of two, rational numbers, theorem, proof, mathematics, number theory, Hardy, A Course of Pure Mathematics},
            pdfproducer={Latex},
            pdfcreator={miktex or pdflatex}]{hyperref}
\hypersetup{
    colorlinks=true,
    linkcolor=black,
    filecolor=magenta,      
    urlcolor=blue,
}
\usepackage{hyperxmp}
\hypersetup{
    pdfauthor={James Philip Rowell},
    pdfcopyright={Copyright 2018 by James Philip Rowell. All rights reserved.}
}
\usepackage{lipsum}

\newenvironment{jprIn}{\begin{adjustwidth}{2em}{}}{\end{adjustwidth}}
\addtolength{\skip\footins}{6pt}

\usepackage{alphalph}
\makeatletter
\newalphalph{\fnsymbolwrap}[wrap]{\@fnsymbol}{}
\makeatother
\renewcommand*{\thefootnote}{%
  \fnsymbolwrap{\value{footnote}}%
}

\DeclarePairedDelimiter\abs{\lvert}{\rvert}

%%% END Article customizations

\author{James Philip Rowell}
\title{\vspace{-1.5cm}Rational Approximations to $\sqrt{2}$}
\date{\vspace{-0.5cm}\footnotesize\today\ (\jprVersion)}
\begin{document}
\maketitle

In Chapter 1, Examples\footnote{Hardy doesn't call them `Exercises'
or `Questions', but that's what they are, exercises for the student.}
III, no.\,2, from
G. H. Hardy's `A Course of Pure Mathematics',
Hardy has
us examine a sequence of rational approximations to $\sqrt{2}$, which are:
{\small
\bgroup                                  %% open the group
\setlength{\abovedisplayskip}{0pt}       %% effective inside the group
\begin{center}
\begin{equation*}
\frac{1}{1},
\enspace\frac{3}{2},
\enspace\frac{7}{5},
\enspace\frac{17}{12},
\enspace\frac{41}{29},
\enspace\frac{99}{70}
\end{equation*}
\end{center}
\egroup
}

These numbers squared are:
{\small
\bgroup                                  %% open the group
\setlength{\abovedisplayskip}{0pt}       %% effective inside the group
\begin{center}
\begin{equation*}
\frac{1}{1},
\enspace\frac{9}{4},
\enspace\frac{49}{25},
\enspace\frac{289}{144},
\enspace\frac{1681}{841},
\enspace\frac{9801}{4900}
\end{equation*}
\end{center}
\egroup
}

And we discover that their differences from 2 are:
{\small
\bgroup                                  %% open the group
\setlength{\abovedisplayskip}{0pt}       %% effective inside the group
\begin{center}
\begin{equation*}
-1,
\enspace\frac{1}{4},
\enspace-\frac{1}{25},
\enspace\frac{1}{144},
\enspace-\frac{1}{841},
\enspace\frac{1}{4900}
\end{equation*}
\end{center}
\egroup
}

In the next `Example no.\,3' we are asked to prove a theorem relating to the above sequence, noting
it is only an \emph{example} of an application of the theorem below.

\section*{Theorem}

Show that if $\frac{m}{m}$ is a good approximation to $\sqrt{2}$, then 
$\frac{(m+2n)}{(m+n)}$ is a better one, and that the errors in the 
two cases are in opposite directions.

\bigskip
The proof follows.

\section*{Lemma}

Let \(m, n\) and \(a\) be integers, where \(m\) and \(n\) are positive and \(a \ne{} 0\), then

\[\frac{m^2}{n^2}+\frac{a}{n^2}=2 \text{\hspace{2.5em}} \Leftrightarrow \text{\hspace{2.5em}}\frac{(m+2n)^2}{(m+n)^2}-\frac{a}{(m+n)^2}=2\]

\section*{Proof of Lemma}

\begin{alignat*}{2}
  &&\frac{(m+2n)^2}{(m+n)^2}-\frac{a}{(m+n)^2}
  \text{\hspace{0.5em}}&=\text{\hspace{0.5em}}
  2\\
  &\Leftrightarrow\quad
  &(m+2n)^2 - a
  \text{\hspace{0.5em}}&=\text{\hspace{0.5em}}
  2(m+n)^2\\
  &\Leftrightarrow\quad
  &m^2 + 4mn + 4n^2
  \text{\hspace{0.5em}}&=\text{\hspace{0.5em}}
  2m^2 + 4mn + 2n^2 + a\\
  &\Leftrightarrow\quad
  &4n^2 - 2n^2
  \text{\hspace{0.5em}}&=\text{\hspace{0.5em}}
  2m^2 - m^2 + a\\
  &\Leftrightarrow\quad
  &2n^2
  \text{\hspace{0.5em}}&=\text{\hspace{0.5em}}
  m^2 + a\\
  &\Leftrightarrow\quad
  &2
  \text{\hspace{0.5em}}&=\text{\hspace{0.5em}}
  \frac{m^2}{n^2}+\frac{a}{n^2}\\
\end{alignat*}
\hspace*{\fill}QED

\section*{Proof of Theorem}
Define the `error' of a rational approximation to $\sqrt{2}$ as the difference between the
approximation-squared and 2.
So we learn from the Lemma that the error for the $\frac{m}{n}$ approximation of $\sqrt{2}$ is\newline
\centerline{$\abs*{\frac{a}{n^2}} = \abs*{2 - \frac{m^2}{n^2}}$,}

and the error for the $\frac{(m+2n)}{(m+n)}$ approximation of $\sqrt{2}$ is\newline
\centerline{$\abs*{\frac{-a}{(m+n)^2}} = \abs*{2 - \frac{(m+2n)^2}{(m+n)^2}}$.}

\bigskip
The denominator $(m+n)^2$ is always larger than $n^2$ therefore,
\[\abs*{\frac{-a}{(m+n)^2}} < \abs*{\frac{a}{n^2}} \text{,}\]
showing that the approximation $\frac{(m+2n)}{(m+n)}$ for
$\sqrt{2}$ 
has a smaller error that that of $\frac{m}{n}$ and is therefore
a better approximation to $\sqrt{2}$.

Furthermore, we also learn from the Lemma that the successive approximations flip
to either side of 2 because:

If $\frac{m^2}{n^2} < 2$ then $a$ is positive therefore $\frac{(m+2n)}{(m+n)} > 2$, similarly,

If $\frac{m^2}{n^2} > 2$ then $a$ is negative therefore $\frac{(m+2n)}{(m+n)} < 2$.

\bigskip
\hspace*{\fill}QED

\end{document}
