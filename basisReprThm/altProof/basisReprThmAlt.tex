\documentclass{article}

\usepackage[utf8]{inputenc} % set input encoding (not needed with XeLaTeX)

%%% PAGE DIMENSIONS
\usepackage{geometry} % to change the page dimensions
\geometry{letterpaper} % or letterpaper (US) or a5paper or....
\geometry{margin=1.35in} % for example, change the margins to 2 inches all round

\usepackage{graphicx} % support the \includegraphics command and options

\usepackage[parfill]{parskip} % Activate to begin paragraphs with an empty line rather than an indent

%%% PACKAGES
\usepackage{booktabs} % for much better looking tables
\usepackage{array} % for better arrays (eg matrices) in maths
\usepackage{paralist} % very flexible & customisable lists (eg. enumerate/itemize, etc.)
\usepackage{verbatim} % adds environment for commenting out blocks of text & for better verbatim
\usepackage{subfig} % make it possible to include more than one captioned figure/table in a single float
% These packages are all incorporated in the memoir class to one degree or another...

%%% HEADERS & FOOTERS
\usepackage{fancyhdr} % This should be set AFTER setting up the page geometry
\pagestyle{fancy} % options: empty , plain , fancy/
\renewcommand{\headrulewidth}{0pt} % customise the layout...
\lhead{}\chead{}\rhead{}
\lfoot{}\cfoot{\thepage}\rfoot{}

%%% SECTION TITLE APPEARANCE
\usepackage{sectsty}
%\allsectionsfont{\sffamily\mdseries\upshape} % (See the fntguide.pdf for font help)
% (This matches ConTeXt defaults)

%%% ToC (table of contents) APPEARANCE
\usepackage[nottoc,notlof,notlot]{tocbibind} % Put the bibliography in the ToC
\usepackage[titles,subfigure]{tocloft} % Alter the style of the Table of Contents
\renewcommand{\cftsecfont}{\rmfamily\mdseries\upshape}
\renewcommand{\cftsecpagefont}{\rmfamily\mdseries\upshape} % No bold!

% JPR added
\usepackage{fontawesome}
\usepackage{amsfonts}
%\usepackage{amsmath}
\usepackage{mathtools}% includes amsmath
\usepackage{changepage}
\usepackage{enumerate}
%\usepackage{setspace}
\usepackage{relsize}
\usepackage{wasysym}
%\usepackage{romannum}

\newcommand{\jprVersion}{v01} % version
 
\usepackage[pdftex,
            pdfauthor={James Philip Rowell},
            pdftitle={\jobname.\jprVersion},
            pdfsubject={Counting or The Basis Representation Theorem},
            pdfkeywords={decimal, base-ten, binary, base-two, integers, theorem, proof, mathematics, number-theory},
            pdfproducer={Latex},
            pdfcreator={miktex or pdflatex}]{hyperref}
\hypersetup{
    colorlinks=true,
    linkcolor=black,
    filecolor=magenta,      
    urlcolor=blue,
}
\usepackage{hyperxmp}
\hypersetup{
    pdfauthor={James Philip Rowell},
    pdfcopyright={Copyright  2017 by James Philip Rowell. All rights reserved.}
}
\usepackage{lipsum}

\newenvironment{jprIn}{\begin{adjustwidth}{2em}{}}{\end{adjustwidth}}
\addtolength{\skip\footins}{6pt}

\usepackage{alphalph}
\makeatletter
\newalphalph{\fnsymbolwrap}[wrap]{\@fnsymbol}{}
\makeatother
\renewcommand*{\thefootnote}{%
  \fnsymbolwrap{\value{footnote}}%
}

\usepackage{perpage}
\usepackage[bottom]{footmisc}
\MakePerPage{footnote}

%%% END Article customizations

\title{\vspace{-1.5cm}Basis Representation Theorem - Alternate Proof}
\author{James Philip Rowell}
\date{\vspace{-0.5cm}\footnotesize\today\ (\jprVersion)} % Activate to display a given date or no date (if empty),
% otherwise the current date is printed

\begin{document}
\maketitle

\section*{Basis Representation Theorem}
\begin{jprIn}
Let \(b\) be a positive integer greater than 1.

For every positive integer \(n\) there is a unique sequence
of integers \(d_0, d_1, d_2,\dots{},d_k\) such that:

\hspace{3em}\(n=d_kb^k+d_{k-1}b^{k-1}+\dots+d_2b^2+d_1b^1+d_0b^0\),

where \(0\le{}d_i<b\) for all \(i\) in \(\{0,1,2,\dots{},k\}\) and \(d_k\ne0\).

Definition: \(n\) is represented in base-\(b\) by the string
of base-\(b\)-digits \((d_kd_{k-1}{\cdots}d_2d_1d_0)_b\)
\end{jprIn}
\bigskip


The paper ``\href {https://www.dropbox.com/s/bwmrffmkcidnf27/basisReprThm.pdf?dl=0}{Counting}''
proves the ``Basis Representation Theorem'' by induction but suggests that it could
also be proven by generalizing the technique used in exercise 2-iii; that proof follows.

\section*{Lemma}
\begin{jprIn}
Let \(b\) be an integer where \(b\ne{}0\) and \(c_0, c_1, c_2, \dots{}, c_n\) be a sequence of integers, then:
{\footnotesize
\bgroup                                  %% open the group
\setlength{\abovedisplayskip}{0pt}       %% effective inside the group
\begin{center}
\begin{equation*}
(((\dots{}(((c_0)b + c_1)b + c_2)b + \dots{} c_{n-2})b + c_{n-1})b + c_n)
=
c_0b^n + 
c_1b^{n-1} + 
c_2b^{n-2} + 
\dots{} +
c_{n-2}b^2 + 
c_{n-1}b^1 + 
c_{n}b^0
\end{equation*}
\end{center}
\egroup
}
\end{jprIn}

\section*{Proof of Lemma by Induction}
Base case:
\begin{jprIn}
When \(n=1\) we have \((c_0)b + c_1 = c_0b^1 + c_1b^0\), and
also note that the lemma holds for \(n=0\) since \((c_0) = c_0b^0\).
\end{jprIn}
Induction step:
\begin{jprIn}
Assume the lemma is true for \(n=k\) and prove it true for \(n=k+1\).
{\small
\begin{alignat*}{1}
&((((\dots{}(((c_0)b + c_1)b + c_2)b + \dots{} c_{k-2})b + c_{k-1})b + c_k)b + c_{k+1})\\
=\  &((c_0b^k + c_1b^{k-1} + c_2b^{k-2} + \dots{} + c_{k-2}b^2 + c_{k-1}b^1 + c_{k}b^0)b + c_{k+1})\\
=\  &c_0b^{k+1} + c_1b^{k} + c_2b^{k-1} + \dots{} + c_{k-2}b^3 + c_{k-1}b^2 + c_{k}b^1 + c_{k+1}b^0
\end{alignat*}
}
\end{jprIn}
QED

As a reminder, a statement of the ``Euclidean Division Theorem'' follows,
\section*{Euclidean Division Theorem}
\begin{jprIn}
For all integers \(a\) and \(b\) such that \(b>0\),
there exist \emph{unique} integers \(q\) and \(r\) such that:
\[a=qb+r  \text{ such that } 0\le{}r<b\]
Definition: In the above equation:
\begin{jprIn}
\begin{tabular}{l l}
\(a\) is the \emph{dividend} & (``the number being divided'')\\
\(b\) is the \emph{divisor} & (``the number doing the dividing'')\\
\(q\) is the \emph{quotient} & (``from Latin \emph{quotiens} `how many times' \(b\) goes into \(a\)'')\\
\(r\) is the \emph{remainder} & (``what's left over after the division'')
\end{tabular}
\end{jprIn}
\end{jprIn}

\section*{Proof of Basis Representation Theorem}

Let \(b\) be a positive integer greater than 1 and
let \(n\) be a positive integer.

Dividing \(n\) by \(b\) we get non-negative integers \(q_1\) and \(d_0\) such that,
\begin{alignat*}{1}
n\ &= q_1b + d_0; \text{ where, } 0 \le{} d_0 < b.\\
\intertext{If \(q_1 \ne{} 0\) we continue this process by dividing \(b\) into \(q_1\) to get integers \(q_2\) and \(d_1\) such that,}
q_1\ &= q_2b + d_1; \text{ where, } 0 \le{} d_1 < b,\\
\intertext{As long as the new quotient (i.e., \(q_2\)) is non-zero, we continue this process
until we get a quotient, say \(q_{k+1} = 0\), as follows,}
q_2\ &= q_3b + d_2; \text{ where, } 0 \le{} d_2 < b\\
q_3\ &= q_4b + d_3; \text{ where, } 0 \le{} d_3 < b\\
\dots{}\\
q_{k-1}\ &= q_{k}b + d_{k-1}; \text{ where, } 0 \le{} d_{k-1} < b\\
q_{k}\ &= q_{k+1}b + d_k; \text{ where, } 0 \le{} d_k < b
\end{alignat*}

There \emph{must} be an integer \(k\) for which \(q_{k+1} = 0\) because for
any \(q_i = q_{i+1}b + d_i\) we have,
\begin{alignat*}{1}
q_i\  &= q_{i+1}b + d_i\\
\ &\ge{} q_{i+1}b + 0\\
\ &\ge{} 2q_{i+1}\\
\ &> q_{i+1}
\end{alignat*}

Let \(q_0 = n\), then the above argument shows that we have a sequence of inequalities,
\[q_0 > q_1 > q_2 > q_3 > \dots{} > q_k > q_{k+1},\]
which must terminate with \(q_{k+1} = 0\) for some \(k \ge{} 0\) since no quotient can be negative.
(As an interesting aside, \(k = {\lfloor}log_b(n){\rfloor} + 1\).)

\break
By back-substituting each expression for \(q_{i+1}\) into the previous expression for \(q_i\),
starting with our last expression \(q_k\), we get:

\[n = (((\dots{}(((d_k)b + d_{k-1})b + d_{k-2})b + \dots{} d_{2})b + d_{1})b + d_0)\]

By an application of our lemma (noting the change of indices: \(d_k = c_0,
d_{k-1} = c_1, \dots{},\newline d_1 = c_{k-1}, d_0 = c_k\)), then we can conclude that:

\[n = d_kb^k+d_{k-1}b^{k-1}+\dots+d_2b^2+d_1b^1+d_0b^0.\]

Furthermore \(0\le{}d_i<b\) for all \(i\) in \(\{0,1,2,\dots{},k\}\) and \(d_k\ne0\).

\(d_k\) not zero because otherwise \(q_i\) would have been zero and the process would have stopped at step \(k-1\)

Since Eucl Div Thm guarantees uniqueness, then QED.

\end{document}
