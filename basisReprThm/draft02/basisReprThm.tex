% !TEX TS-program = pdflatex
% !TEX encoding = UTF-8 Unicode

% This is a simple template for a LaTeX document using the "article" class.
% See "book", "report", "letter" for other types of document.

%\documentclass[11pt]{article} % use larger type; default would be 10pt
\documentclass{article}

\usepackage[utf8]{inputenc} % set input encoding (not needed with XeLaTeX)

%%% Examples of Article customizations
% These packages are optional, depending whether you want the features they provide.
% See the LaTeX Companion or other references for full information.

%%% PAGE DIMENSIONS
\usepackage{geometry} % to change the page dimensions
\geometry{a4paper} % or letterpaper (US) or a5paper or....
% \geometry{margin=2in} % for example, change the margins to 2 inches all round
% \geometry{landscape} % set up the page for landscape
% read geometry.pdf for detailed page layout information

\usepackage{graphicx} % support the \includegraphics command and options

\usepackage[parfill]{parskip} % Activate to begin paragraphs with an empty line rather than an indent

%%% PACKAGES
\usepackage{booktabs} % for much better looking tables
\usepackage{array} % for better arrays (eg matrices) in maths
\usepackage{paralist} % very flexible & customisable lists (eg. enumerate/itemize, etc.)
\usepackage{verbatim} % adds environment for commenting out blocks of text & for better verbatim
\usepackage{subfig} % make it possible to include more than one captioned figure/table in a single float
% These packages are all incorporated in the memoir class to one degree or another...

%%% HEADERS & FOOTERS
\usepackage{fancyhdr} % This should be set AFTER setting up the page geometry
\pagestyle{fancy} % options: empty , plain , fancy
\renewcommand{\headrulewidth}{0pt} % customise the layout...
\lhead{}\chead{}\rhead{}
\lfoot{}\cfoot{\thepage}\rfoot{}

%%% SECTION TITLE APPEARANCE
\usepackage{sectsty}
%\allsectionsfont{\sffamily\mdseries\upshape} % (See the fntguide.pdf for font help)
% (This matches ConTeXt defaults)

%%% ToC (table of contents) APPEARANCE
\usepackage[nottoc,notlof,notlot]{tocbibind} % Put the bibliography in the ToC
\usepackage[titles,subfigure]{tocloft} % Alter the style of the Table of Contents
\renewcommand{\cftsecfont}{\rmfamily\mdseries\upshape}
\renewcommand{\cftsecpagefont}{\rmfamily\mdseries\upshape} % No bold!

% JPR added
\usepackage{fontawesome}
\usepackage{amsfonts}
\usepackage{amsmath}
\usepackage{changepage}
\usepackage{enumerate}
%\usepackage{setspace}

\newenvironment{jprIn}{\begin{adjustwidth}{2em}{}}{\end{adjustwidth}}
\addtolength{\skip\footins}{6pt}

\usepackage{alphalph}
\makeatletter
\newalphalph{\fnsymbolwrap}[wrap]{\@fnsymbol}{}
\makeatother
\renewcommand*{\thefootnote}{%
  \fnsymbolwrap{\value{footnote}}%
}

\usepackage{perpage}
\MakePerPage{footnote}


%%% END Article customizations

%%% The "real" document content comes below...

\title{SesameStreet++}
\author{James Rowell}
%\date{} % Activate to display a given date or no date (if empty),
% otherwise the current date is printed 

\begin{document}
\maketitle
\begin{em}
\small{}There are 10 sorts of people in the world: those who understand binary and those who don't.
%\par
%\setlength{\parskip}{0pt}
\end{em}
\normalsize
\bigskip

Most of us think about ``whole numbers'' not too differently from the way
we learned to count by watching Sesame Street,
the difference being that now we can count a little higher.
How we've trained ourselves it's automatic to think that the way
we write a number or say a number \emph{is} the number.

If I owe you 13 cents and I give you one dime and three pennies then 
after thanking me profusely for repaying this staggering debt,
we'll agree that it's settled with those coins equaling 13 pennies. 
We identify the symbol ``13'' very strongly with this particular number - it would 
be tough to get through life in the modern world without such an automatic process
running in our brains. 
This example highlights what this particular symbol ``13'' actually means -
one dime $(1 \times 10)$ plus three pennies $(3 \times 1)$.

Let's look at the number 13 in some alternative ways - it's the 
number of months in a year plus one month;
what I'm suggesting is that there is no need for the symbol ``13'' in order
to think about this particular number of months. Similarly, 13 is this many apples
\faApple{}\faApple{}\faApple{}\faApple{}\faApple{}\faApple{}\faApple{}\faApple{}\faApple{}\faApple{}\faApple{}\faApple{}\faApple{};
or 13 is the sixth prime number. None of these ways of thinking about the number
13 require that we represent it using the digits 1 and 3 butted up next to each other.

Each number exists independently from any symbol or word that might represent it.
Numbers are an idea - perhaps such a strong idea that the universe wouldn't exist without them!
Anyway, for our purposes whole numbers exist in some abstract realm -
Each number is one whole unit more than the previous number,
starting at nothing, that is ``zero'', and jumping to something, that is ``one'',
then one more, which gets us to ``two'',
then again to ``three'', etc. Continuing in this way forever\dots{} we get them all.

To get back to the idea of what a whole number really is,
try to forget about the symbols or words we use and picture a pile of apples.
There's zero apples (it's hard to show no-apples),
then we introduce an ``\faApple{}" to get our very first,
and smallest, non-empty pile of apples.
Then add another apple to get a pile of ``\faApple{}\faApple{}",
then ``\faApple{}\faApple{}\faApple{}",
then ``\faApple{}\faApple{}\faApple{}\faApple{}"
then some big pile of
``\faApple{}\faApple{}\faApple{}\faApple{}\faApple{}\dots{}\faApple{}\faApple{}\faApple{}"
after we've been adding apples for a while.
Each successively bigger pile of apples corresponds with each successive whole number.

We expand this entire set of whole numbers to include their
negative-counterparts and call this larger set ``integers".
We denote the set of integers with this symbol: $\mathbb{Z}$.
If we only want to talk about positive integers along with zero,
we use this symbol: $\mathbb{Z}_{\ge 0}$.

However, using a ``1" followed by a ``3" to represent
``\faApple{}\faApple{}\faApple{}\faApple{}\faApple{}\faApple{}\faApple{}\faApple{}\faApple{}\faApple{}\faApple{}\faApple{}\faApple{}"
is VERY handy. So we use Hindu-Arabic numbers and the positional notation of ``base-ten",
more commonly known as ``decimal", to represent each specific integer.
We slap a ``-" on the front if we need to talk about a negative integer.

Base-ten representation of an integer is far superior to ancient Roman numerals for example.
Try adding two numbers together in ancient Rome, or worse,
multiplying or dividing them.
What's XI times IX? Would you believe me if I told
you it's XCIX? Unless you convert those to Hindu-Arabic numerals to check,
you're just gonna have to trust me.
Truth is - I don't know how to multiply using Roman
numerals - nor did most Romans! Not only that,
but I'll bet that most kids who graduated from Sesame Street
can count higher than any Roman could - as the
Roman system only effectively allowed counting up to 4999.

Using base-ten for us is automatic,
we barely think about it when we're adding numbers or multiplying
them - but it's worth looking carefully at how base-ten
works - so let's examine it from the ground up\footnote{Please forgive the incredibly obvious nature of much
of the following discussion, but I want to take a good running start
at some more difficult notions, so I figured that it's helpful
to start with stuff that everyone already knows, but looking at it with fresh eyes.}.

It's useful to have simple symbols to represent each of the integers from one to nine,
namely our familiar 1, 2, 3, 4, 5, 6, 7, 8 and 9
which have an interesting history and predate their use in base-ten.

Slightly more modern, but still quite ancient,
is the symbol ``0" for ``zero", originally meaning ``empty".
Zero also predates its use in base-ten but without zero,
base-ten wouldn't be possible.

Base-ten uses the idea of stringing a series of digits together
(a digit being one of the numbers 0, 1, 2, \dots{} 9),
one after the other to be able to represent any whole number.
Let's look at the first two-digit number, that is, ten,
which as you well know looks like this: ``10".
This extra digit on the left tells us how many tens we have and the last,
or rightmost digit says how many additional units to add to it.

So our very first two-digit number 10 means ``one lot of ten - plus zero units".
When we see ``11" - we interpret it to mean ``one lot of ten - plus one unit",
and ``12" is ``one lot of ten - plus two units", etc.
Continuing on; ``20" - we interpret to mean ``two lots of ten,
plus zero units", etc. up to ``90" meaning ``nine lots of ten,
plus zero units".

Following this line of reasoning since ``10" now means the integer ten,
then ``100" must mean ``ten lots of ten,
plus zero units"- which is exactly what it means.
We have a special word for this number we call it ``one hundred" or ``one lot of a hundred,
plus zero lots of tens, plus zero units".
Similarly ``200" means ``two lots of a hundred, plus zero lots of ten,
plus zero units", etc.

We can keep going by one-hundred until we similarly get to ``1000" or ``ten lots of a hundred,
plus zero lots of ten, plus zero units" otherwise known
as ``a thousand" or more specifically ``one lot of a thousand,
plus zero lots of a hundred, plus zero lots of ten, plus zero units".

It get's a little tedious to be so specific when reading out
a number so our language has developed quite a few verbal shortcuts.
Furthermore it doesn't take long before we run out of fancy names
for these ``powers of ten" like, million, billion, trillion,
zillion etc. So let's introduce some nice clean mathematical notation
to describe these powers of ten and let's forget the fancy words.

\begin{align*}
100&=10\times10=10^2,\\
1000&= 10\times10\times10=10^3,\\
10000&= 10\times10\times10\times10=10^4,\\
\dots{}\\
\underbrace{10\dots{}000}_\text{k zeros}&= \underbrace{10\times10\times10\times10\times\dots{}\times10}_\text{k 10s}=10^k
\end{align*}
$10^k$ means there are $k$ 10's multiplied together - 
also written as a 1 followed by $k$ zeros.
The above list shows the cases for $k = 2$, 3 and 4.
Using the $k$ like that is just a way to show that we can pick ANY whole number,
i.e., there is no limit on how big $k$ can be.

The notation of $10^k$ is very handy, in fact it extends
to the case when $k=0$ and $k=1$.\footnote{$10^k$ also extends to the cases
when $k$ is negative as in $10^{-1}$, or $10^{-2}$, etc. which means $\frac{1}{10}$ and $\frac{1}{100}$ respectively
but those are called ``Rational'' numbers and we aren't concerning ourselves with those kinds of numbers in this paper.}

So $10^1$ means that there is only one 10 multiplied together, or one ``0" following the ``1",
in other words just the number 10 itself.

%\looseness=-1
How about when $k=0$.
Examining the pattern of how the power $k$ relates to how many zeros follow the ``1"
(eg, $10^1=10$, $10^2=100$, $10^3=1000$,
etc.) then it makes sense that $10^0=1$,
i.e., no zeros follow the ``1", which is exactly right.
Actually any number raised to the 0\textsuperscript{th} power~is~1.\footnote{Proof:
 $a^b=a^{b+0}=a^ba^0$ so because of the uniqueness of the multiplicative identity, then $a^0=1$ for all $a\ne0$. Furthermore
 mathematicians just go ahead and define $0^0=1$ because it's not inconsistent to do so, you can make a couple of decent arguments as to why
 it make sense, and it makes certain equations and relationships much simpler and more elegant to express.}

Let's look at an example.
Reading the number 46307 out according to our technique we can see
that it's ``four lots of ten-thousand,
plus six lots of a thousand, plus three lots of a hundred,
plus zero lots of ten, plus seven units":

\begin{center}
\begin{tabular}{r r r r c r r}
\phantom & 4 & $\times$ & 10000 & \phantom & \phantom & 40000\\
+ & 6 & $\times$ & 1000 & \phantom & + & 6000\\
+ & 3 & $\times$ & 100 & \; \; = \; \; & + & 300\\
+ & 0 & $\times$ & 10 & \phantom & + & 00\\
+ & 7 & $\times$ & 1 & \phantom & + & 7\\
\cline{6-7}
\phantom & \phantom & \phantom & \phantom & \phantom & = & 46307
\end{tabular}
\end{center}

Written in terms of powers of ten: $46307=4\times10^4+6\times10^3+3\times10^2+0\times10^1+7\times10^0$.

You can think of each digit as being a little dial that controls how many lots
of its corresponding power of ten will contribute to the value of the integer.

Claim: Given that we can use as high as power of ten as we like
and we can string together as LONG A LIST of digits as pleases us,
that means that we can create ANY INTEGER WE WANT no matter how big it is.

That's a pretty tall claim.
How do we know that we can create ALL the nonnegative
integers with this scheme? For example,
how do we know that we didn't miss one? Or how do we know that some string of
digits doesn't represent two different integers? I know it seems silly to ask that,
but attention to these kinds of details is what is referred to as ``rigor"
in Mathematics - it's necessary so we don't end up fooling ourselves or spouting
bullshit - or if we are full of it then it's easy
for other Mathematicians to call us on our nonsense.

We're going to jump into the deep end and make our claim in a careful mathy kind of way.
Such a careful statement is called a theorem - theorems require proof which we will get to below.

\section*{Base-Ten Representation Theorem}

\begin{jprIn}
Let $n,k\in \mathbb{Z}_{\ge 0}$. Then every $n$ can be uniquely expressed as follows:

\hspace{3em}$n=d_k10^k+d_{k-1}10^{k-1}+\dots+d_210^2+d_110^1+d_010^0$

for some $k$ such that $0 \le d_i \le 9$ where $d_i,i\in\mathbb{Z}$ and $0 \le i \le k$.

Furthermore $d_k\ne0$ except when $n=0$.

Definition: $n$ is represented in base-ten as $d_kd_{k-1}\dots{}d_2d_1d_0$
\end{jprIn}
\bigskip

A difficulty many folks have with math is the notation - it's
a kind of a language unto itself - like a computer program is a language.
Let's take our theorem statement by statement and turn it into english.

\begin{enumerate}[i)]
\item ``Let $n,k\in \mathbb{Z}_{\ge 0}$"

This means we are going to talk about two distinct numbers
that we are labeling $n$ and $k$.
That strange looking $\in$ means ``is an element of" (or ``is a member of")
and is always followed by something that is a ``set".
We talked above about the symbol $\mathbb{Z}_{\ge 0}$ which
we defined as being the set of nonnegative integers.
So, in other words, $n$ can be one of 0 or 1 or 2 or 3 or \dots{} 
any number - no matter how large - and the same goes for $k$.

This might be what it would sound like to read that line out loud:\\
``Let $n$ and $k$ be elements of the set of nonnegative Integers."

\item ``Then every $n$ can be uniquely expressed as follows"

What we are about to say applies to ALL nonnegative integers
and furthermore the expression is going to be unique for each integer.

\item ``$n=d_k10^k+d_{k-1}10^{k-1}+\dots+d_210^2+d_110^1+d_010^0$"

This is the expression in question. It equates $n$ with a series of multiplications
of some numbers (the $d_i$ terms where $i$ can be any
number from 0 to $k$) times descending powers of 10,
and adds them all together.

It's useful to point out the meaning of our ``$d_0,d_1,d_2,\dots{},d_k$'' and $d_i$ terms.
Mathematical formulas such as this make judicious use of ``subscripts''
when coming up with names for lists of variables or constants.
Subscripts following a letter or symbol, such as $d_0,d_1,d_2,\dots{}$
are a handy way to get a list of variable or constant names that are similar looking to each other,
and is meant to imply that they each fulfill a similar role to each other.
Note here how the value of the subscript on each ``$d$" corresponds to its power of ten,
even in the single digit case when the subscript is 0,
or the highest power case when the subscript is $k$. Please also note that it
isn't necessary that the subscript indices match with the powers, but it seems helpful! 

If we had to read the line out loud it might sound something like this:\\
``$n$ is equal to\dots{} $dee$-$kay$ times ten-to-the-$kay$,
plus $dee$-$kay$-minus-one times ten-to-the-$kay$-minus-one,
plus etc. etc., down to\dots{} $dee$-$two$ times ten-squared,
plus $dee$-$one$ times ten, plus $dee$-$zero$ times one".

\item ``for some $k$ such that $0 \le d_i \le 9$ where $d_i,i\in\mathbb{Z}$ and $0 \le i \le k$"

The ``for some $k$" means that each integer $n$ has a specific $k$ associated with it.

It then states that those $d_i$ terms are integers,
and can ONLY take on the values 0, 1, 2, 3, 4, 5, 6, 7, 8 or 9.
Note that our uniqueness claim above means that each
integer $n$ has it's own unique list of $d$'s.

It also is very fastidiously pointing out that the little ``$i$'' we just introduced
in the subscript of the $d$'s is also an integer and can be as small as zero but
only as large as our highest power $k$ - whatever $k$ might be.
This is very picky stuff - like a computer program spelling things out
very precisely so the computer knows exactly what you mean.
(That's right - you are the computer).

Sounding it out might sound like this:\\
``for some $kay$ such that zero is less-than-or-equal-to $dee$-$i$
which is less-than-or-equal-to nine, for each $dee$-$i$ and $i$,
which are integers; also $i$ is between zero and $k$ inclusive"

\item ``Furthermore $d_k\ne0\dots{}$"

This is spelling out one more important fastidious detail.
We want to make sure that the ``most significant $d$", that is,
our $d_k$ that goes along with the highest power $10^k$ is not 0,
in other words it must be one of 1, 2, 3, 4, 5, 6, 7, 8 or 9.
This is necessary so that we can get our uniqueness property,
otherwise we could say 13 = 013 = 0000013 which are all the integer 13,
so let's outlaw this uninteresting and annoying possibility.

\item ``\dots{}except when $n=0$"

\dots{}completing that last statement which allows for
one exception to the requirement that the ``most significant digit"
is not allowed to be zero, and that's exactly when the
integer $n$ in question IS zero.

\item ``Definition: $n$ is represented in base-ten as $d_kd_{k-1}\dots{}d_2d_1d_0$"

This is introducing what it means to write the number out in base-ten; that is,
we toss out all the extraneous stuff from our expression in (iii) above,
and string all the ``digits" one after another,
from most significant digit $d_k$ on the left down to least
significant digit $d_0$ on the right.
\end{enumerate}

Before we prove our theorem,
consider that base-ten is not the only base in use these days.
Since the introduction of the EDVAC\footnote{You might be thinking, don't you
mean ENIAC which was eariler? Actually no - the ENIAC
used base-ten accumulators, not binary!} computer, around 1950,
there have been many orders of magnitude more calculations done
in base-two (otherwise known as binary) by computers than have EVER
been done by people in base-ten for the entirety of human history.
(This might even be true if we only count one-day's worth of
binary computer calculations - someone needs to check this.)

Binary-computer logic gates (the building blocks of the modern computer)
can only take one of two states, that is; ``off" or ``on".
We interpret these two states to represent these two numbers: 0 and 1.
By doing so, in the same way that base-ten uses ten numbers 0,
1, 2, 3, 4, 5, 6, 7, 8, and 9 for its digits; we can represent integers
in base-two with just the digits 0 and 1. How is this possible?
Let's find out with an imaginary trip into space.

Consider distant Planet-Nova on which the emergent
intelligent species only have nine fingers.
They have three hands with three fingers each - anyway,
that's why they use base-nine, so they only need the numbers 0,
1, 2, 3, 4, 5, 6, 7 and 8 for their digits.
So like we Earthlings do for the integer ten,
instead of making up a new symbol for nine,
they use ``10" to represent the integer nine - which
for them means ``One lot of nine, plus zero units".

Similarly on Planet-Ocho, since they only have eight fingers,
then they use base-eight and only use numbers 0, 1,
2, 3, 4, 5, 6 and 7 for their digits. For them ``10''
means ``One lot of eight, plus zero units".

On and on past Planet-Gary-Seven, and Planet-Secks, Planet-Penta, \dots{}

Finally we come upon Planet-Claire (well someone
has to come from Planet-Claire,
I know she came from there),
where the poor blighters only have two fingers
so they only use the digits 0 and 1 and base-two,
so for them ``10" means ``one lot of two and zero units".
So on Planet-Claire ``10" means two.
Recall above how we arrived at our 100 in base-ten,
being ``ten lots of ten,
plus zero units" - similarly on Planet-Claire ``100''
in base-two for them means ``Two lots of two plus zero units" in other words,
four! What is ``11" in base-two? Using our technique to
describe the digits we see that it's ``One lot of two, plus one unit",
in other words three.

Here's how they count on Planet-Claire using base-two:
\begin{center}
\begin{tabular}{r l c r l}
%\phantom & 4 & $\times$ & 10000 & \phantom & \phantom & 40000\\
base-two & base-ten & \; \; \; \; & base-two & base-ten\\
0 & 0 & \phantom& (\dots cont)\\
1 & 1 & \phantom& 1101 & 13\\
10 & 2 & \phantom& 1110 & 14\\
11 & 3 & \phantom& 1111 & 15\\
100 & 4 & \phantom& 10000 & 16\\
101 & 5 & \phantom& 10001 & 17\\
110 & 6 & \phantom& \dots{}\\
111 & 7 & \phantom& 11111 & 31\\
1000 & 8 & \phantom& 100000 & 32\\
1001 & 9 & \phantom& \dots{}\\
1010 & 10 & \phantom & 1000000 & 64\\
1011 & 11 & \phantom& 10000000 & 128\\
1100 & 12 (cont\dots) & \phantom& 100000000 & 256\\
\end{tabular}
\end{center}

Note something interesting in the list above - the powers of two,
written in base-two,
resemble our powers of 10 in base-ten! That is:
$1=2^0=1$,
$2=2^1=10$\textsubscript{(base-2)},
$4=2^2=100$\textsubscript{(base-2)},
$8=2^3=1000$\textsubscript{(2)},
$16=2^4=10000$\textsubscript{(2)},
$32=2^5=100000$\textsubscript{(2)},
$64=2^6=1000000$\textsubscript{(2)},
$128=2^7=10000000$\textsubscript{(2)},
$256=2^8=100000000$\textsubscript{(2)},
\dots{}

\break
Let's look at the binary number 11010 for example.
Using our wordy technique to describe the number
we can see that it's ``One lot of sixteen,
plus one lot of eight, plus zero lots of four,
plus one lot of two, plus zero units": 

\begin{center}
\begin{tabular}{r r r r c r r l}
\phantom & 1 & $\times$ & 10000 & \phantom & \phantom & 10000 & (16)\\
+ & 1 & $\times$ & 1000 & \phantom & + & 1000 & (8)\\
+ & 0 & $\times$ & 100 & \; \; = \; \; & + & 000 & \\
+ & 1 & $\times$ & 10 & \phantom & + & 10 & (2)\\
+ & 0 & $\times$ & 1 & \phantom & + & 0\\
\cline{6-8}
\phantom & \phantom & \phantom & \phantom & \phantom & = & 11010 & (26)\\
\end{tabular}
\end{center}

Written in terms of powers of two: $11010=1\times2^4+1\times2^3+0\times2^2+1\times2^1+0\times2^0$.

Each digit in base-two can be thought of as a little switch that
turns on or off the contribution of its corresponding power of two.

Claim: Given that the inhabitants of Planet-Claire can use as high a power of two as they like,
and that they can string together as LONG A LIST of binary-digits as pleases them,
that means that they can create ANY INTEGER THEY WANT no matter how big it is.

Sound familiar? Let's restate our theorem for base-ten but rewritten for base-two.

\section*{Base-Two Representation Theorem}

\begin{jprIn}
Let $n,k\in \mathbb{Z}_{\ge 0}$. Then every $n$ can be uniquely expressed as follows:

\hspace{3em}$n=d_k2^k+d_{k-1}2^{k-1}+\dots+d_22^2+d_12^1+d_02^0$

for some $k$ such that $0 \le d_i \le 1$ where $d_i,i\in\mathbb{Z}$ and $0 \le i \le k$.

Furthermore $d_k\ne0$ except when $n=0$.

Definition: $n$ is represented in base-two as $(d_kd_{k-1}\dots{}d_2d_1d_0)_2$
\end{jprIn}
\bigskip

Try reading the above out loud in your head,
line by line, item by item, like we did above when
we ``sounded it out" for the base-ten theorem - it's
helpful to turn the ``math-code" into understandable
english and a useful habit to get into when reading mathematical statements.

Before we go on,
I want to introduce a little notation to help avoid confusion.
How do you know what I'm talking about if I just
write ``1000"? Do I mean $10^3$ or $2^3$?
If there is any possibility for confusion we write
the number like this $(1000)_{10}$ 
for the base-ten version meaning one-thousand and $(1000)_2$
for the binary version meaning eight.

As is hinted by the habits of our various alien friends
above it seems that we can use ANY integer greater than
or equal to 2 as a base (base-one doesn't really make
sense - think about it for a while).
In fact computer graphics artists are known
to stumble upon numbers written in hexadecimal (usually relating to specifying a color-channel),
which is base-sixteen.

Base-sixteen introduces some new single-character symbols to the usual numbers 0, 1,
2, thru 9,
to represent the numbers 10, 11, 12, 13, 14 and 15.
Base-sixteen adds the ``digits'' A, B, C, D, E and F where
A\textsubscript{16}=(10)\textsubscript{10},
B\textsubscript{16}=(11)\textsubscript{10},
C\textsubscript{16}=(12)\textsubscript{10},
D\textsubscript{16}=(13)\textsubscript{10},
E\textsubscript{16}=(14)\textsubscript{10},
F\textsubscript{16}=(15)\textsubscript{10}.
So if you see this number (80FB)\textsubscript{16} then
I bet at this point (if you take a little time with a
calculator and a pad of paper and pencil) then you can
figure out that it's (33019)\textsubscript{10}.

Note that if we omit the parentheses and subscript from a number,
it means we're talking about it in base-ten - our ``default" base.
Case in point: the subscripts that we use to denote the base
(like the ``16" in (80FB)\textsubscript{16}) are written in base-ten!

We really need to get on with proving our two theorems above.
But what about proving the ``base-nine" version of the theorem for the aliens on Planet-Nova,
or the ``base-eight" version for the inhabitants of Planet-Ocho?

To cover all bases (pun intended) let's restate our theorem for the general case,
call it ``base-$b$",
where $b$ is any number greater than or equal to two.
If we can prove that theorem,
then we'll automatically get all the cases of specific bases for free.

\section*{Basis Representation Theorem}

\begin{jprIn}
Let $n,k,b\in \mathbb{Z}_{\ge 0}$ such that $b\ge2$.
Then every $n$ can be uniquely expressed as follows:

\hspace{3em}$n=d_kb^k+d_{k-1}b^{k-1}+\dots+d_2b^2+d_1b^1+d_0b^0$

for some $k$ such that $0 \le d_i \le (b-1)$ where $d_i,i\in\mathbb{Z}$ and $0 \le i \le k$.

Furthermore $d_k\ne0$ except when $n=0$.

Definition: $n$ is represented in base-$b$ as $(d_kd_{k-1}\dots{}d_2d_1d_0)_b$
\end{jprIn}
\bigskip

As we discussed way up at the top of this essay,
we think about generating the set of positive integers as a process that builds them up one by one.
That is, each successive integer is one more than the previous one,
starting at 1, then one more
taking us to 2,
then 3, 4, 5, \dots{}ad~infinitum\footnote{``ad infinitum'' means ``to infinity'', or ``continue forever, without limit''.}\dots

This idea of being able to step one after the other,
beginning at 1 and going forever is called the ``Principle of Mathematical
Induction" and is a basic property of the positive integers.
This principle is more than just a way to generate the set of integers,
it's also a way of thinking about properties of the integers.

Suppose
that $P(n)$ means that the property $P$ holds
for the number $n$; where $n$ is a positive integer.
Then the principle of mathematical induction states that $P(n)$
is true for ALL positive integers $n$ provided that\footnote{This wording of the
definition of ``The Principle of Mathematical Induction'' is essentially borrowed
from ``Calculus'' by Michael Spivak - an fabulous introductory textbook to Analysis.}:

\begin{enumerate}[i)]
\item $P(1)$ is true
\item Whenever $P(k)$ is true, $P(k+1)$ is true.
\end{enumerate}

Why would these two conditions show that $P(n)$ is true for all
positive integers? Note that condition ii) only asserts the truth
of $P(k+1)$ under the assumption that $P(k)$ is true.
However if we also know that $P(1)$ is true then condition ii) implies that $P(2)$ is true,
which again implies that $P(3)$ is true,
which in turn leads to the truth of $P(4)$,
etc., over and over for all positive integers.

Some people picture an infinite row of dominoes.
Having condition i) (called the ``base case") is like being
able to knock over the first domino.
Then knowing condition ii) is also true is like the
fact that any one domino has the ability to knock over the next.
Once you've knocked over the first domino,
they all fall.

Let's look at a simple example:
Perhaps you've heard the story of young Carl Friedrich Gauss
as a boy in
the 1780s who was assigned (along with all his classmates)
the tedious task of summing the first 100 integers -
presumably to keep them quiet and busy while the
teacher corrected some papers. Anyway,
young Gauss immediately produced the answer,
5050, before most of the boys had summed the first couple of numbers.
It wasn't young Gauss's extraordinary computational speed which allowed
him to perform this dazzling task,
but he had the deeper insight that instead of adding 1 plus 2,
then adding 3, then 4, etc.
he saw that if you paired 1 with 100,
and 2 with 99,
and 3 with 98,
etc.,
that each of those pairs added up to 101,
furthermore he knew he'd have 50 such pairs,
meaning he could state the result in a heartbeat - tada - ``5050"!
Gauss is widely regarded as being one of the greatest
mathematicians who has ever lived - the young eight-year old was just getting started.

Anyway,
to generalize Gauss's insight we can write the expression like this:

\[1+2+3+\ldots+n=\frac{n(n+1)}{2}\]

So let's prove this relationship using the principle of mathematical induction.
\bigskip

Let $n=1$ for the ``base case'',
then

\[\frac{1(1+1)}{2}=\frac{2}{2}=1\]

Which is the trivial sum of the first positive integer 1.
\bigskip

Now let's assume the relationship is true for $n$,
and prove that it must also be true for $n+1$:

\begin{align*}
&(1+2+3+\ldots+n) + (n+1)\\
= & \; \frac{n(n+1)}{2} + (n+1)\\
= & \; \frac{n(n+1)}{2} + \frac{2(n+1)}{2}\\
= & \; \frac{n^2+n+2n+2}{2}\\
= & \; \frac{n^2+3n+2}{2}\\
= & \; \frac{(n+1)(n+2)}{2}\\
= & \; \frac{(n+1)((n+1)+1)}{2}
\end{align*}

Which proves young Gauss's expression is true for
the positive integer $n+1$ whenever it's true for $n$ - then
by the principle of mathematical induction,
the expression is true for all positive integers.
\bigskip

We are going to use the principle of mathematical induction to prove the Basis Representation Theorem.

First we will prove that there is such a representation for all
integers $n$ (existence proof).
Meaning that every integer has a way of being written
in the form described by the theorem - especially as
relates to the restrictions on the values that the ``digits" can take on.

Here's a little insight into how the existence proof works, but applied a specific number in base-ten:
All we want to show is that for any number, if you add 1 to it, that it's also possible to express it as a valid number in base-ten.

For example, adding 1 to 69412995 gives us 69412996, which is pretty trivial to show that it's valid in base-ten, only the least-significant digit was
changed, and it's clearlly within the range of 0\dots{}9.

But what about dealing with a ``carry'', for example 
if we were adding 1 to 69412999? We'd need to algebraically capture the idea of the carry.  They way
we do it in the proof is essentially to say that $69412999 = 69410000 + 2999 = 69410000 + (3000-1)$ so that when we add one to it, then
it's clear that the answer is just:
\[69412999 + 1 = 69410000 + (3000-1) + 1 = 69410000 + 3000 + (-1 + 1) = 69413000\]

After the existence proof has been established we will use another technique, called proof by contradiction,
to prove that each such representation is unique - in other
words there aren't two (or more) ways to represent the same integer in base-$b$.

\section*{Existence Proof of the Basis Representation Theorem}

Base case:
\begin{jprIn}
Since $n=0$ is a slightly special case in the theorem, then lets look at both $n=0$ and $n=1$ for our base case.

Let $n=0$.

\begin{jprIn}
We can choose $k=0$ and $d_0=0$.
(This is the one exception spelled out in the theorem in
which the most significant digit of n is allowed to be zero.) Then,
\[n=d_0b^0=0\times{}b^0=0\]
showing that we have a valid representation for 0 in base-$b$ since our only digit\\
$d_0=0\le(b-1)$, for all $b\ge2$.
\end{jprIn}

Now Let $n=1$.

\begin{jprIn}
In this case, we can choose $k=0$ and $d_0=1$.  Then,
\[n=d_0b^0=1\times{}b^0=1\times{}1=1\]
showing that we have a valid representation for 1 in base-$b$ since\\
$d_0=1\le(b-1)$, for all $b\ge2$.
\end{jprIn}
\end{jprIn}
\bigskip

\break
Induction Case:

\begin{jprIn}
Assume that $n$ has a valid representation in base-$b$,
that is, $n$ can be written thus: 
\[n=d_kb^k+d_{k-1}b^{k-1}+\dots+d_2b^2+d_1b^1+d_0b^0\]
with all the appropriate conditions holding for the
values of $d_i$, $b$ and $k$;
and we will prove that $n+1$ also has a valid representation in base-$b$.

We're going to break this step into two cases which cover all possibilities.

Case 1) $d_0\le(b-2)$

\begin{jprIn}
This case examines when the least significant
digit of $n$ is \emph{strictly-less-than} the largest value it can take in base-$b$.
For example, in base-two $d_0$ can only be zero;
In base-five $d_0$ can be at most three;
In base-ten $d_0$ can be at most eight, etc.
This case is
quite easy to deal with, so let's quickly dispense with it\footnote{It will be helpful at this point to recall the axiom of ``Distribution'' that is $a(b+c)=ab+ac$.}.
\begin{align*}
n &= d_kb^k+d_{k-1}b^{k-1}+\dots+d_2b^2+d_1b^1+d_0b^0\\
&\;\;\;\;\;\;\;\;\;\;\;\;\;\;\;\;\;\;\;\;\;\;\;\;\text{if and only if,}\\
n+1 &= d_kb^k+d_{k-1}b^{k-1}+\dots+d_2b^2+d_1b^1+d_0b^0 + 1\\
&= d_kb^k+d_{k-1}b^{k-1}+\dots+d_2b^2+d_1b^1+d_0b^0 + b^0\\
&= d_kb^k+d_{k-1}b^{k-1}+\dots+d_2b^2+d_1b^1 + (d_0+1)b^0
\end{align*}
and our by assumption that $d_0\le(b-2)$, then
\[(d_0+1)\le(b-2)+1=(b-1)\]
showing us that the ``least significant digit"
of $n+1$, being $(d_0+1)$, is less than or equal to $(b-1)$ which means that
$(d_0+1)$ is a valid digit in base-$b$.

Since all the other $d_i$ terms ($d_1,d_2,\dots,d_k$)
for $n+1$ are unchanged from their values for
$n$ then all the digits of $n+1$ are valid in base-$b$.

Therefore we've established the truth of ``Case 1" for the integer $n+1$.
\end{jprIn}
\bigskip

Case 2) $d_0=(b-1)$
\begin{jprIn}
Now we'll look at the case when the least significant digit of $n$ is equal
to the largest value it can take in base-$b$, that is,
$d_0=(b-1)$.
(Note that between ``Case 2" here and ``Case 1" above,
we're covering all the possibilities for what $d_0$ can be.)
For example in base-two
$d_0=1$; in base-five
$d_0=4$; in base-ten $d_0=9$, etc.

%\newcommand\numberthis{\addtocounter{equation}{1}\tag{\theequation}}

Let $j\in \mathbb{Z}_{\ge 0}$ be the lowest power of $b$ such that $d_j<(b-1)$,
meaning we can write $n$ as follows for some $j$:
\[n = d_kb^k+d_{k-1}b^{k-1}+\dots+d_jb^j+(b-1)b^{j-1}+\dots+(b-1)b^1+(b-1)b^0\]
For example,
if $n=69412999$,
then $j=3$,
since $10^3$ is the lowest power of 10 such that its digit $d_3$ is less than 9 (it's 2).\footnote{It will
be helpful at this point to recall some rules of exponents, that is $a^ba^c=a^{b+c}$.}
\begin{align*}
n &=d_kb^k+\dots+d_jb^j+(b-1)b^{j-1}+\dots+(b-1)b^1+(b-1)b^0\\
&= d_kb^k+\dots+d_jb^j+(b^j-b^{j-1})+(b^{j-1}-b^{j-2})+\dots+(b^2-b^1)+(b^1-b^0)\\
&= d_kb^k+\dots+(d_jb^j+b^j)+(-b^{j-1}+b^{j-1})+\dots+(-b^2+b^2)+(-b^1+b^1)-b^0\\
&= d_kb^k+\dots+(d_j+1)b^j-b^0\\
&= d_kb^k+\dots+(d_j+1)b^j-1
\end{align*}
Therefore,
\begin{align}
n+1 &=d_kb^k+\dots+(d_j+1)b^j-1+1 \nonumber \\
&= d_kb^k+\dots+(d_j+1)b^j \label{eqnA}
\end{align}
Since we picked $j$ such that $d_j<(b-1)$,
less restate the inequality as\\
$d_j\le(b-2)$ therefore,
\[(d_j+1)\le(b-2)+1=(b-1)\]
meaning the $j$\textsuperscript{th} digit of $n+1$ is a valid base-$b$ digit.

All digits $d_k,\dots,{}d_{j+1}$ remain unchanged from the base-$b$ representation of $n$,
and all digits $d_{j-1},\dots{},d_0$ are 0.

Therefore all the digits of the base-$b$ representation of $n+1$ are valid in base-$b$.
\bigskip

If you've been fastidiously following the conditions on our subscript $j$ above, then
you may notice that
our proof doesn't leave room for the case that \emph{all} the digits are equal to $(b-1)$ because of
how we defined $j$. For example when $n=99999$.

Let's attend to this remaining detail.

Suppose $d_i=(b-1)$ for all $0\le{}i\le{}k$, then let $d_{k+1}=0$ and $j=k+1$.

All the arguments we just made are essentially the same so picking up at equation (1) above, with our new terms, we have:
\begin{align*}
n+1 &=(d_j+1)b^j\\
&=(d_{k+1}+1)b^{k+1}\\
&=(0+1)b^{k+1}\\
&=b^{k+1}
\end{align*}
Meaning that $n+1$ now has a $(k+1)$\textsuperscript{st} digit and it's equal to 1,
with all the rest of the digits being 0 - which is a valid representation for $n+1$ in base-$b$ for all $b\ge2$.

QED\footnote{``QED'' - is often used at the conclusion of a proof to state that it’s
done - it’s an acronym for the latin phrase ``quod erat demonstrandum'' which means ``that which was to be demonstrated''.
In other words we’ve proven the existence part of the Basis Representation Theorem.} - existence proof.
\end{jprIn}
\end{jprIn}
\bigskip

\break
In order to proceed
% \footnote{I don't want to misrepresent that the proof in
% this paper is the only way to prove the
% Basis Representation Theorem, in fact I know of at least two others. 
% 
% One is more constructive,
% and makes extensive use of the Euclidean Division Theorem but it's just as lengthy as ours here. However it does give
% some nifty insight in how to convert any integer into a specific base.
% 
% Another is quite elegant and short but abstract, and not particularly illuminating for our purposes.
% However, you can find it in the book ``Number Theory'' by George E. Andrews, and also on
% the website ``proofwiki'', just search for the ``Basis Representation Theorem'' to find the proof.}
with proving the uniqueness aspect of the Basis Representation Theorem, we
need to make use of a well established theorem
called the ``Euclidean Division Theorem''.
It sounds onerous, but don’t worry, you all learned it
in the third grade but perhaps not so formally - you called it ``long division''. It simply states the following:

\section*{Euclidean Division Theorem}
\begin{jprIn}
For all $a,b\in{}\mathbb{Z}$ where $b>0$, there exists unique integers $q$ and $r$ such
that
%that\footnote{Actually the theorem, the constraint on $b$ only being that $b\ne0$, however to keep the remainder positive, the restriction on $r$ }%is stated like this $0 \le r < \left|b\right|$.}:
% that\footnote{Actually the theorem
% is stronger, allowing for $b$ to be negative, so the constraints
% on $r$ are actually expressed as $0\le r \le(b-1)$ but that involves introducing the
% concept of ``absolute value'' which is easy,
% but unnecessary for us right now, our weaker version of the theorem will suit us fine.}:

\[a=qb+r  \text{ and } 0\le{}r\le{}(b-1)\]

Definition: In the above equation:
\begin{jprIn}
\begin{tabular}{l l}
a is the \emph{dividend} & (``the number being divided'')\\
b is the \emph{divisor} & (``the number doing the dividing'')\\
q is the \emph{quotient} & (``the result of the division'')\\
r is the \emph{remainder} & (``the leftover'')
\end{tabular}
\end{jprIn}
\end{jprIn}
This is how you first learned to divide.
For example if someone asks you ``What is nineteen divided by three?'', you’d
answer ``six with one remaining''. Here 19 is the \emph{dividend},  3 is the \emph{divisor},
6 is the \emph{quotient} and 1 is the \emph{remainder}. Written in the form of the theorem:
\[19=6\times3+1\]

Often proofs make use of little mini-theorems of their own.
Creating these mini-theorems is way to simplify a step in
the main proof by establishing a useful non-trivial
intermediary result. It makes reading the main proof
easier to follow by not having us get side tracked with
the technicalities of a step we want to make.
These mini-theorems are called ``Lemmas''
and we’re going to make one to help with proving the ``uniqueness'' part
of the ``Basis Representation Theorem''. We're going to make use of the
``Euclidean Division Theorem'' in proving our lemma.

\section*{Lemma}
\begin{jprIn}
Let $b, q, r \in{} \mathbb{Z}$ such that $b>0$ and $0\le{}r\le(b-1)$, then
\[0=qb+r\]
if and only if $q=0$ and $r=0$.
\end{jprIn}

\section*{Proof of Lemma}
\begin{jprIn}
Let $b, q, r \in{} \mathbb{Z}$ such that $b>0$.

Let $q=0$ and $r=0$, then since
\[qb+r=0\cdot{}b+0=0\]
then by the uniqueness property of $q$ and $r$ according to the Euclidean Division theorem,
then there are no other such values for $q$ and $r$ that would satisfy this equation.

QED
\end{jprIn}

\section*{Uniqueness Proof of the Basis Representation Theorem}

Assume $n$ is not unique and that,
\[n=d_kb^k+d_{k-1}b^{k-1}+\dots+d_2b^2+d_1b^1+d_0b^0\]
and,
\[n=c_kb^k+c_{k-1}b^{k-1}+\dots+c_2b^2+c_1b^1+c_0b^0\]

Let's further suppose that the index $j$ is the first power such that the digits $d_j\ne{}c_j$ and without any loss of generality,
let's assume that $dj>cj$.

Therefore:
\[c_kb^k+c_{k-1}b^{k-1}+\dots+c_2b^2+c_1b^1+c_0b^0=d_kb^k+d_{k-1}b^{k-1}+\dots+d_2b^2+d_1b^1+d_0b\]
if and only if,
\[0=(d_k-c_k)b^k+(d_{k-1}-c_{k-1})b^{k-1}+\dots+(d_j-c_j)b^j\]
if and only if,

\[\frac{0}{b^j}=\frac{(d_k-c_k)b^k+(d_{k-1}-c_{k-1})b^{k-1}+\dots+(d_j-c_j)b^j}{b^j}\text{, since }b\ne0\]
if and only if,
\[0=(d_k-c_k)b^{k-j}+(d_{k-1}-c_{k-1})b^{k-j-1}+\dots+(d_{j+1}-c_{j+1})b+(d_j-c_j)\]
if and only if,
\[0=\big((d_k-c_k)b^{k-j-1}+(d_{k-1}-c_{k-1})b^{k-j-2}+\dots+(d_{j+1}-c_{j+1})\big)b+(d_j-c_j)\]
Let $q=\big((d_k-c_k)b^{k-j-1}+(d_{k-1}-c_{k-1})b^{k-j-2}+\dots+(d_{j+1}-c_{j+1})\big)$, then
\[0=qb+(d_j-c_j)\]
Since $0\le(d_j-c_j)\le(b-1)$ and $b>0$ then by our lemma we know that\\
$q=0$ and $d_j-c_j = 0$.

But $d_j-c_j = 0$ if and only if $d_j = c_j$
which contradicts our assumption that $d_j\ne{}c_j$. This implies that the initial assumption that $n$
is not unique is \emph{false}, in other words:

The base-$b$ representation of $n$ is unique.

QED

\break
Thanks for taking that little ride - if you got this far,
then I'd love to give you 50 gold stars!
In addition, being Canadian, I feel compelled to apologize for a couple of things.

First, if I turned you off math more than you were before,
then I failed miserably!  For that you deserve a HUGE Canadian apology.

Second, if this paper did turn you off,
then I might not blame you, as this proof is hardly what one would call elegant.
In fact here's what another\footnote{besides aforementioned Gauss and Euclid.}
of the
world's greatest Mathematicians, ``G.H. Hardy'' had to say about
proofs like this\footnote{coincidentally in an essay he wrote called ``A Mathematician's Apology''.}:

\begin{jprIn}
``We do not want many `variations' in the proof of a mathematical
theorem: `enumeration of cases', indeed, is one
of the duller forms of mathematical argument. A mathematical proof
should resemble a simple and clear-cut constellation, not a scattered
cluster in the Milky Way.''
\end{jprIn}

I couldn't agree more, however, a more elegant proof along the lines
of what Hardy argues for, I think is provided by George E. Andrews in his
introductory book called ``Number Theory''. Andrew's proof is cool,
and the logic inescapable, but not an obvious way to approach the problem.
Perhaps it's a little too out there for our Sesame-Street-plus-plus
approach that this paper is going for. Like I mentioned in a footnote above,
if you want to check it out, it's easy to find on the web at the site ``ProofWiki''.

If you want a taste of an elegant proof, then surely this is the
classic go to example, also from Euclid in 300 BC, and his reasoning (translated to
modern Algebra) goes like this:

\section*{Infinitude of Primes Theorem}
\begin{jprIn}
There are infinitely many primes.
\end{jprIn}

\section*{Proof of the Infinitude of Primes Theorem}
\begin{jprIn}
Assume that the number of primes is finite. %That is, there exists a set $\mathbb{P}=\set{p_0, p_1, p_2,\dots,p_n}$ 
\end{jprIn}

\end{document}
