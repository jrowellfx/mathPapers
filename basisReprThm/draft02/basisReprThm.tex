% !TEX TS-program = pdflatex
% !TEX encoding = UTF-8 Unicode

% This is a simple template for a LaTeX document using the "article" class.
% See "book", "report", "letter" for other types of document.

%\documentclass[11pt]{article} % use larger type; default would be 10pt
\documentclass{article}

\usepackage[utf8]{inputenc} % set input encoding (not needed with XeLaTeX)

%%% Examples of Article customizations
% These packages are optional, depending whether you want the features they provide.
% See the LaTeX Companion or other references for full information.

%%% PAGE DIMENSIONS
\usepackage{geometry} % to change the page dimensions
\geometry{letterpaper} % or letterpaper (US) or a5paper or....
\geometry{margin=1.35in} % for example, change the margins to 2 inches all round
% \geometry{landscape} % set up the page for landscape
% read geometry.pdf for detailed page layout information

\usepackage{graphicx} % support the \includegraphics command and options

\usepackage[parfill]{parskip} % Activate to begin paragraphs with an empty line rather than an indent

%%% PACKAGES
\usepackage{booktabs} % for much better looking tables
\usepackage{array} % for better arrays (eg matrices) in maths
\usepackage{paralist} % very flexible & customisable lists (eg. enumerate/itemize, etc.)
\usepackage{verbatim} % adds environment for commenting out blocks of text & for better verbatim
\usepackage{subfig} % make it possible to include more than one captioned figure/table in a single float
% These packages are all incorporated in the memoir class to one degree or another...

%%% HEADERS & FOOTERS
\usepackage{fancyhdr} % This should be set AFTER setting up the page geometry
\pagestyle{fancy} % options: empty , plain , fancy/
\renewcommand{\headrulewidth}{0pt} % customise the layout...
\lhead{}\chead{}\rhead{}
\lfoot{}\cfoot{\thepage}\rfoot{}

%%% SECTION TITLE APPEARANCE
\usepackage{sectsty}
%\allsectionsfont{\sffamily\mdseries\upshape} % (See the fntguide.pdf for font help)
% (This matches ConTeXt defaults)

%%% ToC (table of contents) APPEARANCE
\usepackage[nottoc,notlof,notlot]{tocbibind} % Put the bibliography in the ToC
\usepackage[titles,subfigure]{tocloft} % Alter the style of the Table of Contents
\renewcommand{\cftsecfont}{\rmfamily\mdseries\upshape}
\renewcommand{\cftsecpagefont}{\rmfamily\mdseries\upshape} % No bold!

% JPR added
\usepackage{fontawesome}
\usepackage{amsfonts}
%\usepackage{amsmath}
\usepackage{mathtools}% includes amsmath
\usepackage{changepage}
\usepackage{enumerate}
%\usepackage{setspace}
\usepackage{relsize}
\usepackage{wasysym}

\usepackage{hyperref}
\hypersetup{
    colorlinks=true,
    linkcolor=black,
    filecolor=magenta,      
    urlcolor=blue,
}

\newenvironment{jprIn}{\begin{adjustwidth}{2em}{}}{\end{adjustwidth}}
\addtolength{\skip\footins}{6pt}

\usepackage{alphalph}
\makeatletter
\newalphalph{\fnsymbolwrap}[wrap]{\@fnsymbol}{}
\makeatother
\renewcommand*{\thefootnote}{%
  \fnsymbolwrap{\value{footnote}}%
}

\usepackage{perpage}
\MakePerPage{footnote}


%%% END Article customizations

%%% The "real" document content comes below...

\title{SesameStreet++}
\author{James Rowell}
%\date{} % Activate to display a given date or no date (if empty),
% otherwise the current date is printed 

\begin{document}
\maketitle
\begin{em}
\small{}There are 10 sorts of people in the world: those who understand binary and those who don't.
%\par
%\setlength{\parskip}{0pt}
\end{em}
\normalsize
\bigskip

Most of us think about ``whole numbers'' not too differently from the way
we learned to count by watching Sesame Street,
the difference being that now we can count a little higher.
How we've trained ourselves, it's automatic to think the way
that we write a number or say a number \emph{is} the number.

If I owe you 13 cents and I give you one dime and three pennies then 
after thanking me profusely for repaying this staggering debt,
we'll agree that it's settled with those coins equaling 13 pennies. 
We identify the symbol ``13'' very strongly with this particular number - it would 
be tough to get through life in the modern world without such an automatic process
running in our brains. 
This example highlights what this particular symbol ``13'' actually means -
one dime $(1 \times 10)$ plus three pennies $(3 \times 1)$.

Let's look at the number 13 in some alternative ways - it's the 
number of months in a year plus one month;
what I'm suggesting is that there is no need for the symbol ``13'' in order
to think about this particular number of months. Similarly, 13 is this many apples
\faApple{}\faApple{}\faApple{}\faApple{}\faApple{}\faApple{}\faApple{}\faApple{}\faApple{}\faApple{}\faApple{}\faApple{}\faApple{};
or 13 is the sixth prime number. None of these ways of thinking about the number
13 require that we represent it using the digits 1 and 3 butted up next to each other.

Each number exists independently from any symbol or word that might represent it.
Numbers are an idea - perhaps such a strong idea that the universe wouldn't exist without them!
Anyway, for our purposes whole numbers exist in some abstract realm -
Each number is one whole unit more than the previous number,
starting at nothing, that is ``zero'', and jumping to something, that is ``one'',
then one more, which gets us to ``two'',
then again to ``three'', etc. Continuing in this way forever\dots{} we get them all.

To get back to the idea of what a whole number really is,
try to forget about the symbols or words we use and picture a pile of apples.
There's zero apples (it's hard to show no-apples),
then we introduce an ``\faApple{}" to get our very first,
and smallest, non-empty pile of apples.
Then add another apple to get a pile of ``\faApple{}\faApple{}",
then ``\faApple{}\faApple{}\faApple{}",
then ``\faApple{}\faApple{}\faApple{}\faApple{}"
then some big pile of
``\faApple{}\faApple{}\faApple{}\faApple{}\faApple{}\dots{}\faApple{}\faApple{}\faApple{}"
after we've been adding apples for a while.
Each successively bigger pile of apples corresponds with each successive whole number.

We expand this entire set of whole numbers to include their
negative-counterparts and call this larger set ``integers".
We denote the set of integers with this symbol: $\mathbb{Z}$.

However, using a ``1" followed by a ``3" to represent the integer
``\faApple{}\faApple{}\faApple{}\faApple{}\faApple{}\faApple{}\faApple{}\faApple{}\faApple{}\faApple{}\faApple{}\faApple{}\faApple{}"
is VERY handy. So we use Hindu-Arabic numbers and the positional notation of ``base-ten",
more commonly known as ``decimal", to represent each specific integer.
We slap a ``-" on the front if we need to talk about a negative integer.

Base-ten representation of an integer is far superior to ancient Roman numerals for example.
Try adding two numbers together in ancient Rome, or worse,
multiplying or dividing them.
What's XI times IX? Would you believe me if I told
you it's XCIX? Unless you convert those to Hindu-Arabic numerals to check,
you're just gonna have to trust me.
Truth is - I don't know how to multiply using Roman
numerals - nor did most Romans! Not only that,
but I'll bet that most kids who graduated from Sesame Street
can count higher than any Roman could - as the
Roman system only effectively allowed counting up to 4999.

Using base-ten for us is automatic,
we barely think about it when we're adding numbers or multiplying
them - but it's worth looking carefully at how base-ten
works - so let's examine it from the ground up\footnote{Please forgive the incredibly obvious nature of much
of the following discussion, but I want to take a good running start
at some more unfamiliar notions. Perhaps looking at the familiar with fresh eyes will help in seeing the new ideas easier.}.

It's useful to have simple symbols to represent each of the integers from one to nine,
namely our familiar 1, 2, 3, 4, 5, 6, 7, 8 and 9
which have an interesting history and predate their use in base-ten.

Slightly more modern, but still quite ancient,
is the symbol ``0" for ``zero", originally meaning ``empty".
Zero also predates its use in base-ten but without zero,
base-ten wouldn't be possible.

Base-ten uses the idea of stringing a series of digits together
(a digit being one of the numbers 0, 1, 2, \dots{} 9),
one after the other to be able to represent any whole number.
Let's look at the first two-digit number, that is, ten,
which as you well know looks like this: ``10".
This extra digit on the left tells us how many tens we have and the last,
or rightmost digit says how many additional units to add to it.

So our very first two-digit number 10 means ``one lot of ten - plus zero units".
When we see ``11" - we interpret it to mean ``one lot of ten - plus one unit",
and ``12" is ``one lot of ten - plus two units", etc.
Continuing on; ``20" - we interpret to mean ``two lots of ten,
plus zero units", etc. up to ``90" meaning ``nine lots of ten,
plus zero units".

Following this line of reasoning since ``10" now means the integer ten,
then ``100" must mean ``ten lots of ten,
plus zero units"- which is exactly what it means.
We have a special word for this number we call it ``one hundred" or ``one lot of a hundred,
plus zero lots of tens, plus zero units".
Similarly ``200" means ``two lots of a hundred, plus zero lots of ten,
plus zero units", etc.

We can keep going by one-hundred until we similarly get to ``1000" or ``ten lots of a hundred,
plus zero lots of ten, plus zero units" otherwise known
as ``a thousand" or more specifically ``one lot of a thousand,
plus zero lots of a hundred, plus zero lots of ten, plus zero units".

It gets a little tedious to be so specific when reading out
a number so our language has developed quite a few verbal shortcuts.
Furthermore it doesn't take long before we run out of fancy names
for these ``powers-of-ten" like, million, billion, trillion,
zillion etc. So let's introduce some nice clean mathematical notation
to describe these powers-of-ten and let's forget the fancy words.

\begin{align*}
100&=10\times10=10^2,\\
1000&= 10\times10\times10=10^3,\\
10000&= 10\times10\times10\times10=10^4,\\
\dots{}\\
\underbrace{10\dots{}000}_\text{k zeros}&= \underbrace{10\times10\times10\times10\times\dots{}\times10}_\text{k 10s}=10^k
\end{align*}
$10^k$ means there are $k$ tens multiplied together - 
also written as a 1 followed by $k$ zeros\footnote{$k$ is called the ``exponent'' and you should
read the symbol $10^k$ as ``ten-raised-to-the-$k$\textsuperscript{th}-power'' or ``ten-to-the-k'',
so $10^2$ is ``ten raised to the second power'' or $10^4$ is ``ten-to-the-fourth''.
You may also see $10^2$ referred to as ``ten squared'', similarly $10^3$ as ``ten cubed'' - but since we 
don't live in 4 dimensional hyperspace, we don't have a way of saying $10^4$ that has geometric meaning.}.
The above list explicitly shows the cases for $k = 2$, 3 and 4.
Using the $k$ like that is just a way to show that we can pick ANY whole number,
i.e., there is no limit on how big $k$ can be.

The notation of $10^k$ is very handy, in fact it extends
to the case when $k=0$ and $k=1$.
%\footnote{$10^k$ also extends to the cases
%when $k$ is negative as in $10^{-1}$, or $10^{-2}$, etc. which means $\frac{1}{10}$ and $\frac{1}{100}$ respectively
%but those are called ``rational numbers''. We aren't concerning ourselves with rational numbers in this paper.}

So $10^1$ means\footnote{Don't forget to read $10^1$ as ``ten-to-the-one''.} that there is only one ten multiplied together, or one ``0" following the ``1",
in other words just the number ten itself.

%\looseness=-1
How about when $k=0$?
Examining the pattern of how the power $k$ relates to how many zeros follow the ``1"
(eg, $10^1=10$, $10^2=100$, $10^3=1000$,
etc.) then it must be the case that $10^0=1$,
i.e., no zeros follow the ``1", which is exactly right.
Furthermore any number raised to the 0\textsuperscript{th} power~is~1.\footnote{Proof: Since $a^{b+c}=a^ba^c$ consider when $c=0$; that is,
 $a^b=a^{b+0}=a^ba^0$ so because of the uniqueness of the multiplicative identity ``1'', then $a^0$ \emph{must} be 1 since it's behaving 
 like a ``1'' in the expression $a^b=a^ba^0$.}

Let's look at an example.
Reading the number 92507 out according to our technique we can see
that it's ``nine lots of ten-thousand,
plus two lots of a thousand, plus five lots of a hundred,
plus zero lots of ten, plus seven units":

\begin{center}
\begin{tabular}{r r r r c r r}
\phantom & 9 & $\times$ & 10000 & \phantom & \phantom & 90000\\
+ & 2 & $\times$ & 1000 & \phantom & + & 2000\\
+ & 5 & $\times$ & 100 & \; \; = \; \; & + & 500\\
+ & 0 & $\times$ & 10 & \phantom & + & 00\\
+ & 7 & $\times$ & 1 & \phantom & + & 7\\
\cline{6-7}
\phantom & \phantom & \phantom & \phantom & \phantom & = & 92507
\end{tabular}
\end{center}

Written\footnote{Recall the mnemonic ``bedmas'' for the ``\href{https://en.wikipedia.org/wiki/Order_of_operations}{Order of Operations}'' in
evaluating an expression, which is no different from what we did in our table above the expression.}
in terms of powers-of-ten: $92507=9{\times{}}10^4+2{\times{}}10^3+5{\times{}}10^2+0{\times{}}10^1+7{\times{}}10^0$.

You can think of each digit as being a little dial that controls how many lots
of its corresponding power-of-ten will contribute to the value of the integer. Please note
that we went ahead and multiplied the units digit (7 in our example) by $10^0$ (which is the
same as multiplying by 1) so that we can see that there's nothing special about the units digit,
it's just some number between 0 and 9, times a power-of-ten like any of the other digits.
It also adds a kind of beauty to the expression in that all
the digits can be treated in a similar matter.

Claim: Given that we can use as high a power-of-ten as we like
and we can string together \emph{as~long~a~list~of~digits} as pleases us,
that means that we can create \emph{any} positive integer we want no matter how big it is.

That's a pretty tall claim.

How do we know that we can create ALL the positive
integers with this scheme? For example,
how do we know that we didn't miss one? Or how do we know that some string of
digits doesn't represent two different integers? I know it seems silly to ask these kinds of questions - 
after all, people have been counting in base-ten for almost two thousand years,
if there was a problem, you'd think we'd have heard about it by now! ...so, obviously it works.

Here's the thing about mathematics - the \emph{only} ideas we take as obvious
are the axioms - those are the mathematical ideas that are so simple that they can't be expressed in
yet other even-simpler ideas. The axioms are the minimal set of simple, obvious, irrefutable ideas from which
everything else in mathematics is built\footnote{The axioms: For every integer $a,b\text{ and }c$:
Associativity: $(a+b)+c=a+(b+c)$ and $a(bc)=(ab)c$;
Commutativity: $a+b=b+a$ and $ab=ba$;
Distributive: $a(b+c)=(b+c)a=ab+ac$;
Identities: There are integers 0 and 1 such that, $a+0=0+a=a$ and $a\cdot{}1=1\cdot{}a=a$ and
Additive Inverse: $a+(-a)=0$. Note: in general integers do NOT have multiplicative inverses 
that are also integers. (eg. $\frac{1}{2}$ is the multiplicative inverse of 2 because $\frac{1}{2}\cdot{}2=1$ but  $\frac{1}{2}$ is not an integer!)}.

As obvious as it is, the fact that
we can use base-ten to represent the integers is NOT among the list of axioms.

As discussed in the opening paragraph, we think that the way
that we write a number, or say a number \emph{is} the number - no problem - it's totally correct to think this way.
Numbers written in base-ten are in
a perfect one-to-one correspondence with the integers so it's safe to think about numbers written 
in base-ten as the integers.

However, the reason that we know that it's safe to think this way is because we 
spelled out what it means to represent a number in base-ten with 
a theorem, then we've proven that the theorem is true.
A proof
is just a series of arguments that logically connects our theorem directly\footnote{directly \dots{}or indirectly
via other previously proven theorems.} to the axioms, so that the only way
that our theorem would be false is if the axioms themselves are false.

So let's think about what a ``Base-Ten Representation Theorem'' might look like
so that once we prove it we may go forward using base-ten with impunity in talking about the more abstract set of integers.

For starters, we intuitively know that counting with base-ten covers all the possible integers.
After all, how we count leaves no room for gaps.  We know that if we count up to say... 499,
that if we add 1, the 9's roll over to 0, and 1 gets added to the 4, like an odometer in your car,
getting us to the very next integer which is 500.  We know this always works no matter what number we add 1 to.

Our intuition is good - Let's write it down as something that has to be
in our theorem. We might say something like:

\emph{Base-Ten Representation Theorem (initial draft)}
\begin{jprIn}
Every integer is representable in base-ten.
\end{jprIn}
 
Something else we know intuitively is that any number written
in base-ten represents only ONE integer. It's almost feels silly to spell it out, but if we were to count out
4 piles of 100 apples, then 9 piles of 10 apples, then 9 additional apples, then
we'd always get the exact same big sized pile of apples.

It goes the other way too - that if we were handed a big pile of apples, and we started counting out
piles of 100, and we got 4 such piles, but we didn't have enough left to make another pile of 100 -
Then if we started counting out
piles of 10, and we got 9 such piles, but we didn't have enough left to make another pile of 10 -
Then if we finished by counting out 9 remaining apples
then we would always get ``499'' apples. There is NO other way to divvy up this big pile of apples
if we follow this kind of procedure.  In other words, each integer is representable by only ONE base-ten number.

Let's strengthen our theorem based on the last two observations.

\emph{Base-Ten Representation Theorem (second draft)}
\begin{jprIn}
Every integer is \emph{uniquely} representable in base-ten.
\end{jprIn}

We know that for every non-zero integer,
there exists another unique integer
such that if you add them together you get zero, or algebraically:
\[\text{For all integers }a\ne0\text{ there exists a unique integer }{-}a\text{ such that }a+(-a)=0\]
This means that EVERY positive integer has a unique negative counterpart.
So if we can prove the
Base-Ten Representation Theorem for the positive integers then we
can easily extend it to the negative integers with a corollary.
Our corollary will say something like ``Slap a `-' (minus) sign in front of the positive base-ten representation to 
get a unique representation for
it's negative counterpart'' \dots{}or some such words.  Also zero is easy to deal separately, so to keep our lives
simple and restrict our attention to only proving the theorem for positive integers.

Furthermore, moving forward it would be helpful to have a name, or label, to stand in for any 
positive integer so that we can refer to it directly -
So let's give it a name, how about $n$ for ``number'':

\emph{Base-Ten Representation Theorem (third draft)}
\begin{jprIn}
Let $n$ be a positive integer, then $n$ is \emph{uniquely} representable in base-ten.
\end{jprIn}

So from now on whenever we see $n$ we know we're talking about the positive integer in our theorem.

Back a few pages we looked at the real meaning of the number 92507 by adding together
various lots of powers-of-ten\footnote{It's time to replace our ``$\times{}$'' symbol for multiplication, with ``$\cdot{}$''
because ``$\times{}$'' might get confused for an ``$x$'' in an expression,
whereas ``$\cdot{}$'' never will be. Eg. $x\times{}2$ vs. $x\cdot{}2$, additionally ending up with something that is more aesthetically pleasing.
You may also see the ``$\cdot$'' omitted entirely as in $ab$ - which means $a\cdot{}b$ as you have seen in earlier footnotes.}:
\[92507=9{\cdot}10^4+2{\cdot}10^3+5{\cdot}10^2+0{\cdot}10^1+7{\cdot}10^0\]
\emph{Every} base-ten number implicitly describes an algebraic expression like this, so
let's come up
with a general expression of this form that can describe ANY positive integer $n$.

It's important to remember that we only allow `$n$'s digits to take on the values 0 through 9 - no other values are allowed.
Most noteworthy is there are at most 9 lots of any given
power-of-ten in our base-ten-representation of $n$.
For example, if we counted out 9 piles of ten-apples each, but still ten more apples to count, we would simply add that extra pile of ten-apples
to the other 9 piles of ten-apples and count it as 1 pile of a hundred-apples.  That's the rule - we are only allowed
at most 9 lots of a certain power-of-ten sized pile-of-apples.

Let the term $d$ represent one of the digits of $n$.
So $d$ represents how many lots of some power-of-ten contribute to the value of $n$.
$d$ must be between 0 and 9 inclusive (we could say 0 is less than or equal to d which is also less than
or equal to 9), so the contribution of the digit $d$ to the value of $n$ would be this amount:
\[d\cdot10^i\text{, where }0\le{}d\le{}9\text{ and }i\ge{}0\]
In other words, for some power-of-ten
(that's the $i$ which can be anything, like 0, 1, 2, 3, etc.) we have $d$ lots of it. Note that $d$
could be 0, which means that it doesn't contribute ANY lots of $10^i$ to the value of $n$, all the way up to the maximum
contribution of $9\cdot{}10^i$.

Our general expression for $n$ should only have one digit associated with each power-of-ten, in other words our expression should never allow this:
\[4\cdot10^2+7\cdot10^2\]
because we could group those powers-of-ten together like this:
\[4\cdot10^2+7\cdot10^2 = (4+7)10^2=11\cdot10^2\]
\dots{}which violates our rule of keeping the value of the digit in the range of 0 to 9.

$n$ is likely composed of \emph{many} digits, each of which is multiplied by a distinct power-of-ten.
Suppose $n$ is a five digit number. We've been using the specific five-digit number 92507 in our examples above,
but now suppose $n$ is any number from 10000 all
the way up to 99999 (which is a complete list of all the five-digit numbers possible). If we were to describe
this general five digit number, then we would need five different `$d$'s for each of the five digits.
Somehow we need to associate a different digit $d$ with each of the powers
$10^4$, $10^3$, $10^2$, $10^1$ and $10^0$.

Mathematics has a convention for coming up with
a list of terms (like variables or constants) for situations just like this -
we tack a subscript onto the name like so: $d_2$ which you read as ``dee-two''\footnote{\dots{}yes like artoo-detoo, which perhaps
George should have writen like ``$R_2D_2$'' and not ``R2-D2''!}.
$d_2$ is a term to represent a digit just like the $d$ we used above.  But now we can use that little subscript as a way to
associate it to a specific power-of-ten. Naturally we'll associate $d_2$ with $10^2$ (ten squared) as follows; $d_2$ contributes to
the value of $n$ by this amount:
\[d_2\cdot10^2\text{, where }0\le{}d_2\le{}9\]
If we define $d_2$ like this, then we know that when we refer to the digit $d_2$
that we are talking about the digit that is multiplied with $10^2$.

Let our five-digit-number $n$ use $d_0,d_1,d_2,d_3$ and $d_4$ for its digits. Then
the general expression for $n$ looks like this:
\[n=d_4{\cdot}10^4+d_3{\cdot}10^3+d_2{\cdot}10^2+d_1{\cdot}10^1+d_0{\cdot}10^0\]

Since we're getting something that looks like some hard-core math, let's take a moment
to read it out loud, as a Math-Professor might do in a lecture.  It might sound like this:
\begin{jprIn}
``$n$ is equal to dee-four times ten-to-the-fourth, \dots{}
%\emph{{\raise.17ex\hbox{$\scriptstyle\mathtt{\sim}$}}pause{\raise.17ex\hbox{$\scriptstyle\mathtt{\sim}$}}}
plus dee-three times ten-cubed, \dots{} plus dee-two times ten-squared,  \dots{} plus dee-one times ten,  \dots{} plus dee-zero times one.''
\end{jprIn}

When we write out our expression for our five-digit number $n$ we want to be sure
to let people know that each of the digits $d_0,d_1,d_2,d_3$ and $d_4$ must be between 0 and 9 inclusive.

Furthermore, to make sure $n$ is a legitimate five-digit number we have to call out the exception that $d_4$ can NOT be zero - it has to be at least 1.
Why? Because if $d_4$ were allowed to be zero then $n$ might only be a four-digit number, or a three-digit number, or only two-digits etc.

Let's call out this important information, along with the expression for $n$:
\[n=d_4{\cdot}10^4+d_3{\cdot}10^3+d_2{\cdot}10^2+d_1{\cdot}10^1+d_0{\cdot}10^0\text{, such that }0\le{}d_{\{0,1,2,3,4\}}\le{}9\text{, furthermore }d_4\ne0\]
I just made up the notation $d_{\{0,1,2,3,4\}}$ which I'm hoping coveys the idea that any of those five `$d$'s lies between 0 and 9 inclusive. A more 
conventional notation is to let another term like $i$, for perhaps ``index'', stand in for the subscript then specify what values $i$ can assume:
\begin{align*}
n =\ &d_4{\cdot}10^4+d_3{\cdot}10^3+d_2{\cdot}10^2+d_1{\cdot}10^1+d_0{\cdot}10^0,\\
& \text{such that }0\le{}d_i\le{}9\text{ where }0\le{}i\le{}4\text{, furthermore }d_4\ne0
\end{align*}
Perfect! That expression for $n$ describes ANY five-digit number - now we need to extend
our hard-coded five-digit expression for $n$ to an arbitrary
number of digits.
Consider the following progression:
\begin{center}
\begin{tabular}{r l}
expression for $n$ & number-of-digits\\
$d_0{\cdot}10^0$ & 1\\
$d_1{\cdot}10^1+d_0{\cdot}10^0$ & 2\\
$d_2{\cdot}10^2+d_1{\cdot}10^1+d_0{\cdot}10^0$ & 3\\
$d_3{\cdot}10^3+d_2{\cdot}10^2+d_1{\cdot}10^1+d_0{\cdot}10^0$ & 4\\
$d_4{\cdot}10^4+d_3{\cdot}10^3+d_2{\cdot}10^2+d_1{\cdot}10^1+d_0{\cdot}10^0$ & 5\\
\dots{} & \dots{}\\
$d_k{\cdot}10^k+\dots{}+d_4{\cdot}10^4+d_3{\cdot}10^3+d_2{\cdot}10^2+d_1{\cdot}10^1+d_0{\cdot}10^0$ & k+1\\
\end{tabular}
\end{center}
Using $k$ like this let's us specify any number of digits we want.  If we let $k=0$ we get the first ``single digit'' 
item on the list.  $k=4$ gives us our five-digit number above, or we could let $k$ be a billion, which would allow
us to specifcy an integer that has a billion-and-one digits in it\footnote{A billion-and-one digit number is 
\emph{ridiculously} large,
consider that our estimate of the number of molecules in the entire universe would only need a base-ten
number with the $k$ set to somewhere between 78 and 82 to count them all.}!

So there we have it, we found our expression for being able to express ANY positive integer, let's use
it in a revised draft of our theorem:

\emph{Base-Ten Representation Theorem (close to final draft)}
\begin{jprIn}
Let $n$ be a positive integer, then for every $n$ there exists a unique representation:

\hspace{3em}$n=d_k{\cdot}{1\hspace{-0.12em}0}^k+d_{k-1}{\cdot}{1\hspace{-0.12em}0}^{k-1}+\dots+d_2{\cdot}{1\hspace{-0.12em}0}^2+d_1{\cdot}{1\hspace{-0.12em}0}^1+d_0{\cdot}{1\hspace{-0.12em}0}^0$

such that $0\le{}d_i\le{}9$ where $0\le{}i\le{}k$, furthermore $d_k\ne0$.

Definition: $n$ is represented in base-ten by the string of digits $d_kd_{k-1}{\cdots}d_2d_1d_0$
\end{jprIn}

Our newly added ``Definition'' introduces exactly what it means to write the number out in base-ten; that is,
we toss out all the extraneous stuff from our expression
and string all the digits one after another.
Starting at the most-significant digit $d_k$ on the left, down to the next digit to its right which is $d_{k-1}$
(read as ``dee-kay-minus-one''\footnote{\dots{}and $d_{k-1}$ is multiplied by ``ten-to-the-power-of-(kay-minus-one)''.})
all the way down to the least-significant units-digit $d_0$ on the right.

We are so darn close, but there is one picky detail that we have to be careful about.
We can safely use the symbols 
for the integers from \emph{zero} to \emph{nine}, as they exist independently from base-ten\footnote{as we discussed earlier
these symbols: 0, 1, 2, 3, 4, 5, 6, 7, 8 and 9 predate base-ten - they stand for the integers that represent our successive
sized piles of apples from none, to \faApple{} up
to \faApple{}\faApple{}\faApple{}\faApple{}\faApple{}\faApple{}\faApple{}\faApple{}\faApple{}.}. \emph{However} this
theorem defines what base-ten means so how can we use ``10'' in our theorem?!
The answer is, we can't - at least not without acknowledging it somehow.

We have a couple of choices.  Since we don't have an actual symbol for the integer ten, we have to make one up. The first possibility is:
\begin{jprIn}
Let $n$ be a positive integer.  Let $T$ represent the integer ten,\\
then for every $n$ there exists a unique representation:

\hspace{3em}$n=d_k{\cdot}{T}^k+d_{k-1}{\cdot}{T}^{k-1}+\dots+d_2{\cdot}{T}^2+d_1{\cdot}{T}^1+d_0{\cdot}{T}^0$\\
\dots{} etc.
\end{jprIn}

The other choice is we could cheat a little, and define the symbol for \emph{ten} to be ``$1\hspace{-0.2em}0$''.
I favor this approach for now, don't worry, it's not breaking any rules, you can go ahead and define things
however you like in Mathematics so long as it doesn't create inconsistencies.  Anyway, we're not going
to prove the Base-Ten Representation Theorem - we're going to keep digging, and eventually proves something
more general where this problem disappears.  But let's write it out anyway, since we've worked so hard to get this far PLUS
it looks so nice!

\section*{Base-Ten Representation Theorem}

\begin{jprIn}
Let $n$ be a positive integer. Let ${1\hspace{-0.12em}0}$ represent the integer ten,\\
then for every $n$ there exists a unique representation:

\hspace{3em}$n=d_k{\cdot}{1\hspace{-0.12em}0}^k+d_{k-1}{\cdot}{1\hspace{-0.12em}0}^{k-1}+\dots+d_2{\cdot}{1\hspace{-0.12em}0}^2+d_1{\cdot}{1\hspace{-0.12em}0}^1+d_0{\cdot}{1\hspace{-0.12em}0}^0$

such that $0\le{}d_i\le{}9$ where $0\le{}i\le{}k$, furthermore $d_k\ne0$.

Definition: $n$ is represented in base-ten by the string of digits $d_kd_{k-1}{\cdots}d_2d_1d_0$
\end{jprIn}
\bigskip

Consider that base-ten is not the only base in use these days.
Since the introduction of the EDVAC\footnote{You might be thinking, don't you
mean ENIAC which was earlier? Actually no - the ENIAC
used base-ten accumulators, not binary!} computer, around 1950,
there have been many orders of magnitude more calculations done
in base-two (otherwise known as binary) by computers than have EVER
been done by people in base-ten for the entirety of human history.
(This might even be true if we only count one-day's worth of
binary computer calculations - someone needs to check this.)

Binary-computer logic gates (the building blocks of the modern computer)
can only take one of two states, that is; ``off" or ``on".
We interpret these two states to represent these two numbers: 0 and 1.
By doing so, in the same way that base-ten uses ten numbers 0,
1, 2, 3, 4, 5, 6, 7, 8, and 9 for its digits; we can represent integers
in base-two with just the digits 0 and 1. How is this possible?
Let's find out with an imaginary trip into space.

Consider distant Planet-Nova on which the emergent
intelligent species only have nine fingers.
They have three hands with three fingers each - anyway,
that's why they use base-nine, so they only need the numbers 0,
1, 2, 3, 4, 5, 6, 7 and 8 for their digits\footnote{Digit is another word for finger!}.
So like we Earthlings do for the integer ten,
instead of making up a new symbol for nine,
they use ``10" to represent the integer nine - which
for them means ``One lot of nine, plus zero units".

Similarly on Planet-Ocho, since they only have eight fingers,
then they use base-eight and only use numbers 0, 1,
2, 3, 4, 5, 6 and 7 for their digits. For them ``10''
means ``One lot of eight, plus zero units".

On and on past Planet-Gary-Seven, and Planet-Secks, Planet-Penta, \dots{}

Finally we come upon Planet-Claire (well someone
has to come from Planet-Claire,
I know she came from there),
where the poor blighters only have two fingers
so they only use the digits 0 and 1 and base-two,
so for them ``10" means ``one lot of two and zero units".
So on Planet-Claire ``10" means two.
Recall above how we arrived at our 100 in base-ten,
being ``ten lots of ten,
plus zero units" - similarly on Planet-Claire ``100''
in base-two for them means ``Two lots of two plus zero units" in other words,
four! What is ``11" in base-two? Using our technique to
describe the digits we see that it's ``One lot of two, plus one unit",
in other words three.

Here's how they count on Planet-Claire using base-two:
\begin{center}
\begin{tabular}{r l c r l}
base-two & base-ten & \; \; \; \; & base-two & base-ten\\
0 & 0 & \phantom& (\dots cont)\\
1 & 1 & \phantom& 1101 & 13\\
10 & 2 & \phantom& 1110 & 14\\
11 & 3 & \phantom& 1111 & 15\\
100 & 4 & \phantom& 10000 & 16\\
101 & 5 & \phantom& 10001 & 17\\
110 & 6 & \phantom& \dots{}\\
111 & 7 & \phantom& 11111 & 31\\
1000 & 8 & \phantom& 100000 & 32\\
1001 & 9 & \phantom& \dots{}\\
1010 & 10 & \phantom & 1000000 & 64\\
1011 & 11 & \phantom& 10000000 & 128\\
1100 & 12 (cont\dots) & \phantom& 100000000 & 256\\
\end{tabular}
\end{center}

Note something interesting in the list above - the powers of two,
written in base-two,
resemble our powers of 10 in base-ten! That is:
\begin{small}
\begin{center}
\begin{spreadlines}{0em}
\begin{align*}
1 = 2^0&= 1, & 32 = 2^5&= 100000_{(\text{base-2})},\\
2 = 2^1&= 10_{(\text{base-2})}, & 64 = 2^6&= 1000000_{(\text{base-2})},\\
4 = 2^2&= 100_{(\text{base-2})}, & 128 = 2^7&= 10000000_{(\text{base-2})},\\
8 = 2^3&= 1000_{(\text{base-2})},& 256 = 2^8&= 100000000_{(\text{base-2})},\\
16 = 2^4&= 10000_{(\text{base-2})},& & \dots{}\\
\end{align*}
\end{spreadlines}
\end{center}
\end{small}

Let's look at the binary number 11010 for example.
Using our wordy technique to describe the number
we can see that it's ``One lot of sixteen,
plus one lot of eight, plus zero lots of four,
plus one lot of two, plus zero units": 
\begin{center}
\begin{tabular}{r r r r c r r l}
\phantom & 1 & $\times$ & 10000 & \phantom & \phantom & 10000 & (16)\\
+ & 1 & $\times$ & 1000 & \phantom & + & 1000 & (8)\\
+ & 0 & $\times$ & 100 & \; \; = \; \; & + & 000 & \\
+ & 1 & $\times$ & 10 & \phantom & + & 10 & (2)\\
+ & 0 & $\times$ & 1 & \phantom & + & 0\\
\cline{6-8}
\phantom & \phantom & \phantom & \phantom & \phantom & = & 11010 & (26)\\
\end{tabular}
\end{center}
Written in terms of powers of two: $11010_{(\text{base-2})}=26=1\cdot2^4+1\cdot2^3+0\cdot2^2+1\cdot2^1+0\cdot2^0$.

Each digit in base-two can be thought of as a little switch that
turns on or off the contribution of its corresponding power of two.

Claim: Given that the inhabitants of Planet-Claire can use as high a power of two as they like,
and that they can string together as LONG A LIST of binary-digits as pleases them,
that means that they can create ANY INTEGER THEY WANT no matter how big it is.

Sound familiar? Let's restate our theorem for base-ten but rewritten for base-two.

\section*{Base-Two Representation Theorem}

\begin{jprIn}
Let $n$ be a positive integer, then for every $n$ there exists a unique representation:

\hspace{3em}$n=d_k\cdot{}2^k+d_{k-1}\cdot{}2^{k-1}+\dots+d_2\cdot{}2^2+d_1\cdot{}2^1+d_0\cdot{}2^0$

such that $0\le{}d_i\le{}1$ where $0\le{}i\le{}k$, furthermore $d_k\ne0$.

Definition: $n$ is represented in base-two by the string of binary-digits $(d_kd_{k-1}{\cdots}d_2d_1d_0)_2$
\end{jprIn}
\bigskip

Our new Base-Two Representation Theorem introduced some helpful new notation.
How do you know what I'm talking about if I just
write ``1000"? Do I mean $10^3$ or $2^3$?
If there is any possibility for confusion we write
the number like this $(1000)_{10}$ 
for the base-ten version meaning one-thousand and $(1000)_2$
for the binary version meaning eight.  That's what the ``Definition'' is spelling out with the ``$(\dots)_2$'' extra notation.

As is hinted by the habits of our various alien friends
above it seems that we can use ANY integer greater than
or equal to 2 as a base (base-one doesn't really make
sense - think about it for a while).
In fact computer graphics artists are known
to stumble upon numbers written in hexadecimal (usually relating to specifying a color-channel),
which is base-sixteen.

Base-sixteen introduces some new single-character symbols to the usual numbers 0, 1,
2, thru 9,
to represent the numbers 10, 11, 12, 13, 14 and 15.
Base-sixteen adds the digits A, B, C, D, E and F where
A\textsubscript{16}=(10)\textsubscript{10},
B\textsubscript{16}=(11)\textsubscript{10},
C\textsubscript{16}=(12)\textsubscript{10},
D\textsubscript{16}=(13)\textsubscript{10},
E\textsubscript{16}=(14)\textsubscript{10},
F\textsubscript{16}=(15)\textsubscript{10}.
So (80FB)\textsubscript{16} is a four digit number in base-sixteen. (As we'll see shortly it means (33019)\textsubscript{10} in base-ten).

Note that if we omit the parentheses and subscript from a number,
it means we're talking about it in base-ten; our ``default" base.
Case in point: the subscripts that we use to denote the base
(like the ``16" in (80FB)\textsubscript{16}) are written in base-ten!

We could go ahead and prove our ``Base-Ten'' and ``Base-Two'' theorems above,
but what about proving the ``Base-Nine" version of the theorem for the aliens on Planet-Nova,
or the ``Base-Eight" version for the inhabitants of Planet-Ocho?

To cover all bases (pun intended) let's restate our theorem for the general case,
call it ``base-$b$",
where $b$ is any number greater than or equal to two.
If we can prove that theorem,
then we'll automatically get all the cases of specific bases for free.

\section*{Basis Representation Theorem}

\begin{jprIn}
Let $n$ and $b$ be positive integers such that $b\ge2$,\\
then for every $n$ there exists a unique representation:

\hspace{3em}$n=d_kb^k+d_{k-1}b^{k-1}+\dots+d_2b^2+d_1b^1+d_0b^0$

such that $0\le{}d_i\le{}(b-1)$ where $0\le{}i\le{}k$, furthermore $d_k\ne0$.

Definition: $n$ is represented in base-$b$ by the string of base-$b$-digits $(d_kd_{k-1}{\cdots}d_2d_1d_0)_b$
\end{jprIn}
\bigskip

So to get the ``Base-Ten Representation Theorem'' let $b$ equal ten.
To get the ``Base-Two Representation Theorem'' let $b=2$; or the ``Base-Nine Representation Theorem''
let $b=9$; ``Base-Eight\dots{}'' let $b=8$''; ``Base-Sixteen\dots{}'' let $b=16$''; etc.

Bonus: Because of the general nature of the ``Basis Representation Theorem'' we also know that we can safely convert between
different bases. (Why? \dots{}exercise left for the student).  Recall how we defined $(\text{A})_{16}=10$
and $(\text{F})_{16}=15$ as base-sixteen digits, then:
\[9,937,906=9\cdot{}16^5+7\cdot{}16^4+10\cdot{}16^3+3\cdot{}16^2+15\cdot{}16^1+2\cdot{}16^0=(\text{97A3F2})_{16}\]

\section*{Epilogue}
At this point we really ought to get to the proof of the ``Basis Representation Theorem''.

However, truth be told, the proof that I have outlined below is a little heavy going - not difficult,
but not really a great introductory proof for the first time mathematician.
In fact here's what one of the
world's greatest Mathematicians\footnote{along with the Gauss and Euclid mentioned later in this paper.},
G.~H.~Hardy, had to say about such
proofs\footnote{in an essay he wrote called ``A Mathematician's Apology''.}:

\begin{jprIn}
``We do not want many `variations' in the proof of a mathematical
theorem: `enumeration of cases', indeed, is one
of the duller forms of mathematical argument. A mathematical proof
should resemble a simple and clear-cut constellation, not a scattered
cluster in the Milky Way.''
\end{jprIn}

I couldn't agree more which is why I'm pushing the proof to a section at the end of this paper, which you may
feel free to skip.
I still invite you to to take a stab at following it, I tried my best to make it clear, but don't feel bad if it
makes your head hurt. If you do try to follow it - it does have some points of interest - not the least
of which is that it is correct.
Yes it works,  perhaps not as elegantly as Mr. Hardy could provide to us, but I'm no G. H. Hardy.

I think at this point you are well equipped you to try your hands at a couple of exercises for fun. If you get stuck, or to
check your work, the
answers are also supplied below - but please don't peek until you try the questions yourself!

\section*{Exercises}
\begin{enumerate}
\item What are the following numbers expressed in base-ten?
\begin{enumerate}[i)]
\item $(110101)_2$
\item $(\text{A}053\text{D})_{16}$
\item $(1017)_{23}$
\end{enumerate}
\item What are the following base-ten numbers expressed in an alternate base?
\begin{enumerate}[i)]
\item 33 expressed in base-two?
\item 127 expressed in base-two? (Hint: $127 = (128-1)$)
\item 8079 expressed in base-sixteen?

Hint: For a moment, pretend that we don't use base-ten to
write out our numbers, instead picture a pile of apples.
Can you picture 7654 apples?  Yes?  Good let's use 7654 as our example.

Let's divide 7654 by 10 so we get the following:
\[7654 = 765\cdot{}10+4\]
Notice the remainder 4 is the least significant digit of our
integer 7654  (i.e. the $d_0$ digit in the theorem).

How do we get the next digit, i.e. the $d_1$ digit that corresponds to the $10^1$ term? Well, it’s kind of cheating, but
since we happen to be looking at that last expression written in
base-ten we can see it sitting right there in at the
end of the quotient ``765''. So, let's use the same technique and divide 765 by 10:
\[765 = 76\cdot{}10+5\]
So the remainder is 5 our $d_1$ digit.  Let's keep going, this time dividing the previous quotient 76 by 10.\dots	
\[76 = 7\cdot{}10+6\]
and finally,
\[7= 0\cdot{}10+7\]
So, our series of remainders happens to be the digits of the number in base 10.
Specifically $d_3 = 7$, $d_2=6$, $d_1=5$ and $d_0=4$.

Try doing that for 8079, but use 16 instead of 10 as the divisor.
\item Let $\text{A}_{23} = 10, \text{B}_{23} = 11, \text{C}_{23} = 12, \text{D}_{23} = 13, \text{E}_{23} = 14, \text{F}_{23} = 15,
\text{G}_{23} = 16,\\
\text{H}_{23} = 17, \text{I}_{23} = 18, \text{J}_{23} = 19, \text{K}_{23} = 20, \text{L}_{23} = 21$ and $\text{M}_{23} = 22$,\\
then what is 185190 expressed in base-twenty-three?
\item 291480 expressed in base-twenty-three?
\end{enumerate}

\end{enumerate}

\section*{Answers}
\begin{enumerate}
\item What are the following numbers expressed in base-ten?
\begin{enumerate}[i)]
\item $(110101)_2 = 53$
\item $(\text{A}053\text{D})_{16}=656701$
\item $(1017)_{23} = 12197$
\end{enumerate}
\item What are the following base-ten numbers expressed in an alternate base?
\begin{enumerate}[i)]
\item $33 = (100001)_2$
\item $127 = (1111111)_2$
\item $8079 = (1\text{F}8\text{F})_{16}$
\item $185190 = (\text{F}51\text{H})_{23}$
\item $291480 = (10\text{M}01)_{23}$
\end{enumerate}
\end{enumerate}

\break
\section*{The Principle of Mathematical Induction}

As we discussed way up at the top of this essay,
we think about generating the set of positive integers as a process that builds them up one by one.
That is, each successive integer is one more than the previous one,
starting at 1, then one more
taking us to 2,
then 3, 4, 5, \dots{}ad~infinitum\footnote{``ad infinitum'' means ``to infinity'', or ``continue forever, without limit''.}\dots

This idea of being able to step one after the other,
beginning at 1 and going forever is embodied within the ``Principle of Mathematical
Induction" and is a basic property of the positive integers.
This principle is more than just a way to generate the set of integers,
it's also a way of thinking about properties of the integers.

Suppose
that $P(n)$ means that the property $P$ holds
for the number $n$; where $n$ is a positive integer.
Then the principle of mathematical induction states that $P(n)$
is true for ALL positive integers $n$ provided that\footnote{This wording of the
definition of ``The Principle of Mathematical Induction'' is essentially borrowed
from ``Calculus'' by Michael Spivak - a fabulous introductory textbook on Analysis.}:

\begin{enumerate}[i)]
\item $P(1)$ is true
\item Whenever $P(k)$ is true, $P(k+1)$ is true.
\end{enumerate}

Why would these two conditions show that $P(n)$ is true for all
positive integers? Note that condition ii) only asserts the truth
of $P(k+1)$ under the assumption that $P(k)$ is true.
However if we also know that $P(1)$ is true then condition ii) implies that $P(2)$ is true,
which again implies that $P(3)$ is true,
which in turn leads to the truth of $P(4)$,
etc., over and over for all positive integers.

Some people picture an infinite row of dominoes.
Having condition i) (called the ``base case") is like being
able to knock over the first domino.
Then knowing condition ii) is also true is like the
fact that any one domino has the ability to knock over the next.
Once you've knocked over the first domino,
they all fall.

Let's look at a simple example:
Perhaps you've heard the story of young Carl Friedrich Gauss
as a boy in
the 1780s who was assigned (along with all his classmates)
the tedious task of summing the first 100 integers -
presumably to keep them quiet and busy while the
teacher corrected some papers. Anyway,
young Gauss immediately produced the answer,
5050, before most of the boys had summed the first couple of numbers.
It wasn't young Gauss's extraordinary computational speed which allowed
him to perform this dazzling task,
but he had the deeper insight that instead of adding 1 plus 2,
then adding 3, then 4, etc.
he saw that if you paired 1 with 100,
and 2 with 99,
and 3 with 98,
etc.,
that each of those pairs added up to 101,
furthermore he knew he'd have 50 such pairs,
meaning he could state the result in a heartbeat - tada - ``5050"!
Gauss is widely regarded as being one of the greatest
mathematicians who has ever lived - the young eight-year old was just getting started.

\break
Anyway,
to generalize Gauss's insight we can write the expression like this:
\[1+2+3+\ldots+n=\frac{n(n+1)}{2}\]
So let's prove this relationship using the principle of mathematical induction.
\bigskip

Let $n=1$ for the ``base case'',
then
\[\frac{1(1+1)}{2}=\frac{2}{2}=1\]
Which is the trivial sum\footnote{The word ``sum'' here is used in the context of the expression we are trying to prove. In this case we are summing only one item thus it's ``trivial''.} of the first positive integer 1.
\bigskip

Now let's assume the relationship is true for $n$,
and prove that it must also be true for $n+1$:
\begin{align*}
&(1+2+3+\ldots+n) + (n+1)\\
= & \; \frac{n(n+1)}{2} + (n+1)\\
= & \; \frac{n(n+1)}{2} + \frac{2(n+1)}{2}\\
= & \; \frac{n^2+n+2n+2}{2}\\
= & \; \frac{n^2+3n+2}{2}\\
= & \; \frac{(n+1)(n+2)}{2}\\
= & \; \frac{(n+1)((n+1)+1)}{2}
\end{align*}
Which proves young Gauss's expression is true for
the positive integer $n+1$ whenever it's true for $n$ - then
by the principle of mathematical induction,
the expression is true for all positive integers. QED\footnote{``QED'' - is often used at the conclusion of a proof to state that it’s
done - it’s an acronym for the Latin phrase ``quod erat demonstrandum'' which means ``that which was to be demonstrated''.
In other words we’ve proven what we set out to prove.}

\section*{Intermission: Extra Exercise}
If $b, n$ are nonnegative integers and $b\ne1$ then prove,
\[1+b+b^2+\dots+b^{n-1} = \frac{b^n-1}{b-1}\]
Hint: use induction on $n$, the base case being $n=1$.

\section*{Proof for Extra Exercise}
Base case: $n=1$
\[\frac{b^1-1}{b-1}=\frac{b-1}{b-1}=1=b^0=b^{1-1}\]

Induction step:
Assume the following
\[1+b+b^2+\dots+b^{n-1} = \frac{b^n-1}{b-1}\]
Then,
\begin{align*}
&(1+b+b^2+\dots+b^{n-1}) + b^n\\
= &\frac{b^n-1}{b-1}+b^n\\
= &\frac{b^n-1}{b-1}+\frac{b^n(b-1)}{b-1}\\
= &\frac{b^n-1+b^{n+1}-b^n}{b-1}\\
= &\frac{b^{n+1}+b^n-b^n-1}{b-1}\\
= &\frac{b^{n+1}-1}{b-1}\\
\end{align*}
QED
\section*{Proof of the  Basis Representation Theorem}
We are going to use two techniques to prove the Basis Representation Theorem.

First we will prove that there is such a representation for all
integers $n$ (existence proof) using the principle of mathematical induction.
Meaning that every integer has a way of being written
in the form described by the theorem - especially as
relates to the bounds on the values that the digit can take on.

Here's a little insight into how the existence proof works, but applied to a specific number in base-ten:
All we want to show is that for any number, if you add 1 to it, that it's also possible to express it as a valid number in base-ten.

For example, adding 1 to 69412995 gives us 69412996, which is pretty trivial to show that it's valid in base-ten, only the least-significant digit was
changed, and it's clearly within the range of 0\dots{}9.

But what about dealing with a ``carry'', for example 
if we were adding 1 to 69412999? We'd need to algebraically capture the idea of the carry.  The way
we do it in the proof is essentially to say that $69412999 = 69410000 + 2999 = 69410000 + (3000-1)$ so that when we add one to it, then
it's clear that the answer is just:
\[69412999 + 1 = 69410000 + (3000-1) + 1 = 69410000 + 3000 + (-1 + 1) = 69413000\]
Secondly, we will use another technique, called proof by contradiction,
to prove that each such representation is unique - in other
words there aren't two (or more) ways to represent the same integer in base-$b$.

\section*{Existence Proof of the Basis Representation Theorem}
Let $n$ and $b$ be positive integers such that $b\ge2$.\\
Also let $k$ be a integer such that $k\ge0$.\\
Furthermore let $i$ and $d_i$ be integers such that $0\le{}d_i\le{}(b-1)$ for all $i\ge0$.

Prove:
\begin{jprIn}
$n$ can be represented as:
\[n=d_kb^k+d_{k-1}b^{k-1}+\dots+d_2b^2+d_1b^1+d_0b^0\]
for all $n$.
\end{jprIn}
Proof by induction on $n$.

Base case:
\begin{jprIn}
Let $n=1$.

\begin{jprIn}
Choose $k=0$ and $d_0=1$.  Then,
\[n=d_0b^0=1\cdot{}b^0=1\cdot{}1=1\]
showing that we have a valid representation for 1 in base-$b$ since\\
$d_0=1\le(b-1)$, for all $b\ge2$.
\end{jprIn}
\end{jprIn}
\bigskip

Induction Case:

\begin{jprIn}
Assume that $n$ has a valid representation in base-$b$,
that is, $n$ can be expressed as: 
\[n=d_kb^k+d_{k-1}b^{k-1}+\dots+d_2b^2+d_1b^1+d_0b^0\]
with all the appropriate conditions holding for $d_i$ and $k$ as outlined at the beginning of the proof;
then we will prove that $n+1$ also has a valid representation in base-$b$.

We're going to break this step into two cases which cover all possibilities.

\break
Case 1) $d_0<(b-1)$

\begin{jprIn}
This case examines when the least significant
digit of $n$ is \emph{strictly-less-than} the largest value it can take in base-$b$.
For example, in base-two $d_0$ can only be zero;
In base-five $d_0$ can be at most three;
In base-ten $d_0$ can be at most eight, etc.
This case is
quite easy to deal with, so let's quickly dispense with it\footnote{Recall the axiom of ``Distribution'' that is $a(b+c)=ab+ac$}.
\begin{align*}
n &= d_kb^k+d_{k-1}b^{k-1}+\dots+d_2b^2+d_1b^1+d_0b^0\\
&\;\;\;\;\;\;\;\;\;\;\;\;\;\;\;\;\;\;\;\;\;\;\;\;\text{if and only if,}\\
n+1 &= d_kb^k+d_{k-1}b^{k-1}+\dots+d_2b^2+d_1b^1+d_0b^0 + 1\\
&= d_kb^k+d_{k-1}b^{k-1}+\dots+d_2b^2+d_1b^1+d_0b^0 + b^0\\
&= d_kb^k+d_{k-1}b^{k-1}+\dots+d_2b^2+d_1b^1 + (d_0+1)b^0
\end{align*}
we can restate our assumption that $d_0<(b-1)$ as $d_0\le(b-2)$, then
\[(d_0+1)\le(b-2)+1=(b-1)\]
showing us that the ``least significant digit"
of $n+1$, now being $(d_0+1)$, is less than or equal to $(b-1)$ which means that
$(d_0+1)$ is a valid digit in base-$b$.

Since all the other terms $d_k,\dots,d_2,d_1$
for $n+1$ are unchanged from their values for
$n$ then all the digits of $n+1$ are valid in base-$b$.

Therefore when $d_0$ is  \emph{strictly-less-than} $b-1$, then $n+1$ has a valid representation in base-$b$.
\end{jprIn}
\bigskip

Case 2) $d_0=(b-1)$
\begin{jprIn}
Now we'll look at the case when the least significant digit of $n$ is equal
to the largest value it can take in base-$b$, that is,
$d_0=(b-1)$.
(Note that  ``Case 1" and ``Case 2" cover all the possibilities for what $d_0$ can be.)
For example in base-two
$d_0=1$; in base-five
$d_0=4$; in base-ten $d_0=9$, etc.

%\newcommand\numberthis{\addtocounter{equation}{1}\tag{\theequation}}

Let $j$ be the lowest power of $b$ such that $d_j<(b-1)$,
meaning we can write $n$ as follows (\dots{}in other words all the digits to the right of $d_j$ are equal to $(b-1)$):
\[n = d_kb^k+d_{k-1}b^{k-1}+\dots+d_jb^j+(b-1)b^{j-1}+\dots+(b-1)b^1+(b-1)b^0\]
For example,
if $n=69412999$,
then $j=3$,
since $10^3$ is the lowest power of 10 such that its digit $d_3$ is less than 9 (it's 2).\footnote{It will
be helpful at this point to recall some rules of exponents, that is $a^ba^c=a^{b+c}$.}
\begin{align*}
n &=d_kb^k+\dots+d_jb^j+(b-1)b^{j-1}+(b-1)b^{j-2}+\dots+(b-1)b^1+(b-1)b^0\\
&= d_kb^k+\dots+d_jb^j+(b^j-b^{j-1})+(b^{j-1}-b^{j-2})+\dots+(b^2-b^1)+(b^1-b^0)\\
&= d_kb^k+\dots+(d_jb^j+b^j)+(-b^{j-1}+b^{j-1})+\dots+(-b^2+b^2)+(-b^1+b^1)-b^0\\
&= d_kb^k+\dots+(d_j+1)b^j-b^0\\
&= d_kb^k+\dots+(d_j+1)b^j-1
\end{align*}
Therefore,
\begin{align}
n+1 &=d_kb^k+\dots+(d_j+1)b^j-1+1 \nonumber \\
&= d_kb^k+\dots+(d_j+1)b^j \label{eqnA}
\end{align}
Since we picked $j$ such that $d_j<(b-1)$,
we can restate the inequality as\\
$d_j\le(b-2)$ therefore,
\[(d_j+1)\le(b-2)+1=(b-1)\]
meaning the $j$\textsuperscript{th} digit of $n+1$ is a valid base-$b$ digit.

All digits $d_k,\dots,{}d_{j+1}$ remain unchanged from the base-$b$ representation of $n$,
and all digits $d_{j-1},\dots{},d_0$ are 0.

Therefore all the digits of the base-$b$ representation of $n+1$ are valid in base-$b$.
\bigskip

If you've been fastidiously following the conditions on our subscript $j$ above, then
you may notice that
our proof doesn't leave room for the case that \emph{all} the digits are equal to $(b-1)$ because of
how we defined $j$. For example when $n=99999$ in base-ten.

Let's attend to this remaining detail.

Suppose $d_i=(b-1)$ for all $0\le{}i\le{}k$, then let $d_{k+1}=0$ and $j=k+1$.

All the arguments we just made are essentially the same so picking up at equation (1) above, with our new terms, we have:
\begin{align*}
n+1 &=(d_j+1)b^j\\
&=(d_{k+1}+1)b^{k+1}\\
&=(0+1)b^{k+1}\\
&=b^{k+1}
\end{align*}
Meaning that $n+1$ now has one more digit than $n$, namely $d_{k+1}$ and it's equal to 1,
with all the rest of the digits for $n+1$ being 0 - which is a valid representation for $n+1$ in base-$b$ for all $b\ge2$.
\end{jprIn}
Therefore 
by the principle of mathematical induction,
we have proven that there is a base-$b$ representation for all nonnegative integers.
\end{jprIn}
\bigskip

In order to proceed
with proving the uniqueness aspect of the Basis Representation Theorem, we
need to make use of a well established theorem
called the ``Euclidean Division Theorem''.
It sounds onerous, but don’t worry, you all learned it
in the third grade but perhaps not so formally, you called it ``long division''. It simply states the following:

\break
\section*{Euclidean Division Theorem\footnote{We introduce a new math symbol here, and use one that was
mentioned in passing much earlier in the paper.  Namely the $\in$ which means ``is an element of'' or
``is a member of'' and is always followed by something that is a set.  We introduced the symbol $\mathbb{Z}$ as the
``set of integers''.  So ``$a,b\in{}\mathbb{Z}$'' means that $a$ and $b$ are integers.}}
\begin{jprIn}
For all $a,b\in{}\mathbb{Z}$ where $b>0$, there exists \emph{unique} integers $q$ and $r$ such
that
that\footnote{Actually the theorem is stronger than we have stated here. Specifically, it only requires that $b\ne0$, however
to keep the remainder positive, the restriction on $r$ would have to be stated like
this $0 \le r \le (\left|b\right|-1)$ to deal with the possibility that $b$ might be negative.}:
\[a=qb+r  \text{ where } 0\le{}r\le{}(b-1)\]
Definition: In the above equation:
\begin{jprIn}
\begin{tabular}{l l}
a is the \emph{dividend} & (``the number being divided'')\\
b is the \emph{divisor} & (``the number doing the dividing'')\\
q is the \emph{quotient} & (``the result of the division'')\\
r is the \emph{remainder} & (``the leftover'')
\end{tabular}
\end{jprIn}
\end{jprIn}
This is how you first learned to divide.
For example if someone asks you ``What is nineteen divided by three?'', you’d
answer ``six with one remaining''. Here 19 is the \emph{dividend},  3 is the \emph{divisor},
6 is the \emph{quotient} and 1 is the \emph{remainder}. Written in the form of the theorem:
\[19=6\cdot3+1\]
Often proofs make use of little mini-theorems of their own.
Creating these mini-theorems is a way to simplify a step in
the main proof by establishing a useful non-trivial
intermediary result. It makes reading the main proof
easier to follow by not having us get sidetracked with
the technicalities of a step we want to make.
These mini-theorems are called ``Lemmas''
and we’re going to make one to help with proving the uniqueness part
of the Basis Representation Theorem, and we're going to make use of the
Euclidean Division Theorem in proving our lemma.

\section*{Lemma}
\begin{jprIn}
Let $b, q, r \in{} \mathbb{Z}$ such that $b>0$ and $0\le{}r\le(b-1)$, then
\[0=qb+r\]
if and only if $q=0$ and $r=0$.
\end{jprIn}

\section*{Proof of Lemma}
\begin{jprIn}
Let $b, q, r \in{} \mathbb{Z}$ such that $b>0$ and $0\le{}r\le{}(b-1)$.

If $q=0$ and $r=0$, then
\[qb+r=0\cdot{}b+0=0\]
but also, by the Euclidean Division Theorem since $q$ and $r$ are unique for every dividend and divisor $b>0$,
then we can also conclude that if $0=qb+r$ then $q=0$ and $r=0$ must be true, otherwise the quotient and remainder
would not be unique. QED
\end{jprIn}

\section*{Uniqueness Proof of the Basis Representation Theorem}
Let $n$ and $b$ be positive integers such that $b\ge2$.\\
Also let $k$ be a integer such that $k\ge0$.\\
Furthermore let $i, c_i, d_i\in\mathbb{Z}$ such that
$0\le{}c_i\le{}(b-1)$ and $0\le{}d_i\le{}(b-1)$ for all $i\ge0$.

By the ``Existence Proof of the Basis Representation Theorem'' we know that $n$ can
always be expressed in this form:
\[n=d_kb^k+\dots+d_2b^2+d_1b^1+d_0b^0\]
Assume this expression for $n$ is not unique and that it can also be expressed as
\[n=c_kb^k+\dots+c_2b^2+c_1b^1+c_0b^0\]
Let's further suppose that the index $j$ is the lowest power such that the digits $d_j\ne{}c_j$ and without any loss of generality,
let's assume that $d_j>c_j$.

Therefore\footnote{Please read the bidirectional arrow symbol $\Leftrightarrow$ as ``if and only if'' - it's like a logical ``equals'' sign}:
\begin{align*}
& c_kb^k+\dots+c_2b^2+c_1b^1+c_0b^0=d_kb^k+\dots+d_2b^2+d_1b^1+d_0b\\
\Leftrightarrow\;\;\;\;&0=(d_k-c_k)b^k+\dots+(d_j-c_j)b^j\\
\Leftrightarrow\;\;\;\;& \frac{0}{b^j}=\frac{(d_k-c_k)b^k+\dots+(d_j-c_j)b^j}{b^j}\text{, since }b\ne0\\
\Leftrightarrow\;\;\;\;& 0=(d_k-c_k)b^{k-j}+\dots+(d_{j+1}-c_{j+1})b+(d_j-c_j)\\
\Leftrightarrow\;\;\;\;& 0=\big((d_k-c_k)b^{k-j-1}+\dots+(d_{j+1}-c_{j+1})\big)b+(d_j-c_j)
\end{align*}
Let $q=\big((d_k-c_k)b^{k-j-1}+\dots+(d_{j+1}-c_{j+1})\big)$, then
\[0=qb+(d_j-c_j)\]
Since $0\le(d_j-c_j)\le(b-1)$ and $b>0$ then by our lemma we know that\\
$q=0$ and $d_j-c_j = 0$.

But $d_j-c_j = 0$ if and only if $d_j = c_j$
which contradicts our assumption that $d_j\ne{}c_j$. This implies that the initial assumption that ``$n$
is not unique'' is \emph{false}, in other words:
\begin{center}
The base-$b$ representation of $n$ is unique.
\end{center}

\bigskip
Therefore since we have proven that there exists a base-$b$ representation for ALL the nonnegative integers, \emph{and}
that this representation is unique for ALL integers, then we have proven the Basis Representation Theorem.

QED
\end{document}
