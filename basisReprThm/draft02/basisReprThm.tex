\documentclass{article}

\usepackage[utf8]{inputenc} % set input encoding (not needed with XeLaTeX)

%%% PAGE DIMENSIONS
\usepackage{geometry} % to change the page dimensions
\geometry{letterpaper} % or letterpaper (US) or a5paper or....
\geometry{margin=1.35in} % for example, change the margins to 2 inches all round

\usepackage{graphicx} % support the \includegraphics command and options

\usepackage[parfill]{parskip} % Activate to begin paragraphs with an empty line rather than an indent

%%% PACKAGES
\usepackage{booktabs} % for much better looking tables
\usepackage{array} % for better arrays (eg matrices) in maths
\usepackage{paralist} % very flexible & customisable lists (eg. enumerate/itemize, etc.)
\usepackage{verbatim} % adds environment for commenting out blocks of text & for better verbatim
\usepackage{subfig} % make it possible to include more than one captioned figure/table in a single float
% These packages are all incorporated in the memoir class to one degree or another...

%%% HEADERS & FOOTERS
\usepackage{fancyhdr} % This should be set AFTER setting up the page geometry
\pagestyle{fancy} % options: empty , plain , fancy/
\renewcommand{\headrulewidth}{0pt} % customise the layout...
\lhead{}\chead{}\rhead{}
\lfoot{}\cfoot{\thepage}\rfoot{}

%%% SECTION TITLE APPEARANCE
\usepackage{sectsty}
%\allsectionsfont{\sffamily\mdseries\upshape} % (See the fntguide.pdf for font help)
% (This matches ConTeXt defaults)

%%% ToC (table of contents) APPEARANCE
\usepackage[nottoc,notlof,notlot]{tocbibind} % Put the bibliography in the ToC
\usepackage[titles,subfigure]{tocloft} % Alter the style of the Table of Contents
\renewcommand{\cftsecfont}{\rmfamily\mdseries\upshape}
\renewcommand{\cftsecpagefont}{\rmfamily\mdseries\upshape} % No bold!

% JPR added
\usepackage{fontawesome}
\usepackage{amsfonts}
%\usepackage{amsmath}
\usepackage{mathtools}% includes amsmath
\usepackage{changepage}
\usepackage{enumerate}
%\usepackage{setspace}
\usepackage{relsize}
\usepackage{wasysym}
%\usepackage{romannum}
 
\usepackage[pdftex,
            pdfauthor={James Rowell},
            pdftitle={\jobname},
            pdfsubject={SesameStreet++ or The Basis Representation Theorem},
            pdfkeywords={decimal, base-ten, binary, base-two, integers, theorem, proof, mathematics, number-theory},
            pdfproducer={Latex},
            pdfcreator={miktex or lualatex}]{hyperref}
\hypersetup{
    colorlinks=true,
    linkcolor=black,
    filecolor=magenta,      
    urlcolor=blue,
}
\usepackage{hyperxmp}
\hypersetup{
    pdfauthor={James Philip Rowell},
    pdfcopyright={Copyright © 2017 by James Philip Rowell. All rights reserved.}
}
\usepackage{lipsum}

\newenvironment{jprIn}{\begin{adjustwidth}{2em}{}}{\end{adjustwidth}}
\addtolength{\skip\footins}{6pt}

\usepackage{alphalph}
\makeatletter
\newalphalph{\fnsymbolwrap}[wrap]{\@fnsymbol}{}
\makeatother
\renewcommand*{\thefootnote}{%
  \fnsymbolwrap{\value{footnote}}%
}

\usepackage{perpage}
\MakePerPage{footnote}

%%% END Article customizations

\title{SesameStreet++}
\author{James Rowell}
%\date{} % Activate to display a given date or no date (if empty),
% otherwise the current date is printed 

\begin{document}
\maketitle
\begin{em}
\centerline{\small{}There are 10 sorts of people in the world: those who understand binary and those who don't.}
%\par
%\setlength{\parskip}{0pt}
\end{em}
%\normalsize
\bigskip

Most of us think about ``whole numbers'' not too differently from the way
we learned to count by watching Sesame Street,
the difference being that now we can count a little higher.
How we've trained ourselves, it's automatic to think the way
that we write a number or say a number \emph{is} the number.

If I owe you 13 cents and I give you one dime and three pennies then 
after thanking me profusely for repaying this staggering debt,
we'll agree that it's settled with those coins equaling 13 pennies. 
We identify the symbol ``13'' very strongly with this particular number - it would 
be tough to get through life in the modern world without such an automatic process
running in our brains. 
This example highlights what this particular symbol ``13'' actually means -
one dime $(1 \times 10)$ plus three pennies $(3 \times 1)$.

Let's look at the number 13 in some alternative ways - it's the 
number of months in a year plus one month;
what I'm suggesting is that there is no need for the symbol ``13'' in order
to think about this particular number of months. Similarly, 13 is this many apples
\faApple{}\faApple{}\faApple{}\faApple{}\faApple{}\faApple{}\faApple{}\faApple{}\faApple{}\faApple{}\faApple{}\faApple{}\faApple{};
or 13 is the sixth prime number. None of these ways of thinking about the number
13 require that we represent it using the digits 1 and 3 butted up next to each other.

Each number exists independently from any symbol or word that might represent it.
Numbers are an idea - perhaps such a strong idea that the universe wouldn't exist without them!
Anyway, for our purposes whole numbers exist in some abstract realm -
Each number is one whole unit more than the previous number,
starting at nothing, that is ``zero'', and jumping to something, that is ``one'',
then one more, which gets us to ``two'',
then again to ``three'', etc. Continuing in this way forever\dots{} we get them all.

To get back to the idea of what a whole number really is,
try to forget about the symbols or words we use and picture a pile of apples.
There are zero apples (it's hard to show no-apples),
then we introduce an ``\faApple{}" to get our very first,
and smallest, non-empty pile of apples.
Then add another apple to get a pile of ``\faApple{}\faApple{}",
then ``\faApple{}\faApple{}\faApple{}",
then ``\faApple{}\faApple{}\faApple{}\faApple{}"
then some big pile of
``\faApple{}\faApple{}\faApple{}\faApple{}\faApple{}\dots{}\faApple{}\faApple{}\faApple{}"
after we've been adding apples for a while.
Each successively bigger pile of apples corresponds with each successive whole number.
%
We expand this entire set of whole numbers to include their
negative-counterparts and call this larger set ``integers".
%We denote the set of integers with this symbol: $\mathbb{Z}$.

However, using a ``1" followed by a ``3" to represent the integer
``\faApple{}\faApple{}\faApple{}\faApple{}\faApple{}\faApple{}\faApple{}\faApple{}\faApple{}\faApple{}\faApple{}\faApple{}\faApple{}"
is VERY handy. So we use Hindu-Arabic numbers and the positional notation of ``base-ten",
more commonly known as ``decimal", to represent each specific integer.
We slap a ``-" on the front if we need to talk about a negative integer.

\break
Base-ten representation of an integer is far superior to ancient Roman numerals for example.
Try adding two numbers together in ancient Rome, or worse,
multiplying or dividing them.
What's XI times IX? Would you believe me if I told
you it's XCIX? Unless you convert those to Hindu-Arabic numerals to check,
you're just gonna have to trust me.
Truth is - I don't know how to multiply using Roman
numerals - nor did most Romans! Not only that,
but I'll bet that most kids who graduated from Sesame Street
can count higher than any Roman could - as the
Roman system only effectively allowed counting up to 4999.

Using base-ten for us is automatic,
we barely think about it when we're adding numbers or multiplying
them - but it's worth looking carefully at how base-ten
works - so let's examine it from the ground up\footnote{Please forgive
the incredibly obvious nature of much
of the following discussion, but I want to take a good running start
at some more unfamiliar notions.
Perhaps looking at the familiar with fresh eyes will help in seeing the new ideas easier.}.

It's useful to have simple symbols to represent each of the integers from one to nine,
namely our familiar 1, 2, 3, 4, 5, 6, 7, 8 and 9
which have an interesting history and predate their use in base-ten.

Slightly more modern, but still quite ancient,
is the symbol ``0" for ``zero", originally meaning ``empty".
Zero also predates its use in base-ten but without zero,
base-ten wouldn't be possible.

Base-ten uses the idea of stringing a series of digits together
(a digit being one of the numbers 0, 1, 2, \dots{} 9),
one after the other to be able to represent each whole number.
Let's look at the first two-digit number, that is, ten,
which as you well know looks like this: ``10".
This extra digit on the left tells us how many tens we have and the last,
or rightmost digit says how many additional units to add to it.

So our very first two-digit number 10 means ``one lot of ten - plus zero units".
When we see ``11" - we interpret it to mean ``one lot of ten - plus one unit",
and ``12" is ``one lot of ten - plus two units", etc.

Continuing on; ``20" - we interpret to mean ``two lots of ten,
plus zero units", etc. up to ``90" meaning ``nine lots of ten,
plus zero units".

Following this line of reasoning since ``10" now means the integer ten,
then ``100" must mean ``ten lots of ten,
plus zero units"- which is exactly what it means.
We have a special word for this number we call it ``one hundred" or ``one lot of a hundred,
plus zero lots of tens, plus zero units".
Similarly ``200" means ``two lots of a hundred, plus zero lots of ten,
plus zero units", etc.

We can keep going by one-hundred until we similarly get to ``1000" or ``ten lots of a hundred,
plus zero lots of ten, plus zero units" otherwise known
as ``a thousand" or more specifically ``one lot of a thousand,
plus zero lots of a hundred, plus zero lots of ten, plus zero units".

It gets a little tedious to be so specific when reading out
a number so our language has developed quite a few verbal shortcuts.
Furthermore it doesn't take long before we run out of fancy names
for these ``powers-of-ten" like, million, billion, trillion,
zillion etc. So let's introduce some nice clean mathematical notation
to describe these powers-of-ten and let's forget the fancy words.

\begin{align*}
100&=10\times10=10^2,\\
1000&= 10\times10\times10=10^3,\\
10000&= 10\times10\times10\times10=10^4,\\
\dots{}\\
\underbrace{10\dots{}000}_\text{k zeros}&=
\underbrace{10\times10\times10\times10\times\dots{}\times10}_\text{k 10s}=10^k
\end{align*}
$10^k$ means there are $k$ tens multiplied together - 
also written as a 1 followed by $k$ zeros\footnote{$k$ is called the ``exponent'' and you should
read the symbol $10^k$ as ``ten-raised-to-the-$k$\textsuperscript{th}-power'' or ``ten-to-the-k'',
so $10^2$ is ``ten raised to the second power'' or $10^4$ is ``ten-to-the-fourth''.
You may also see $10^2$ referred to as ``ten squared'',
similarly $10^3$ as ``ten cubed'' - but since we 
don't live in 4 dimensional hyperspace,
we don't have a way of saying $10^4$ that has geometric meaning.}.
The above list explicitly shows the cases for $k = 2, 3$ and $4$.
Using the $k$ like that is just a way to show that we can pick ANY whole number,
i.e., there is no limit on how big $k$ can be.

The notation of $10^k$ is very handy, in fact it extends
to the case when $k=0$ and $k=1$.
%\footnote{$10^k$ also extends to the cases
%when $k$ is negative as in $10^{-1}$, or $10^{-2}$, etc.
%which means $\frac{1}{10}$ and $\frac{1}{100}$ respectively
%but those are called ``rational numbers''. We aren't concerning
%ourselves with rational numbers in this paper.}

So $10^1$ means\footnote{Don't forget to read $10^1$
as ``ten-to-the-one''.} that there is only one ten multiplied together,
or one ``0" following the ``1",
in other words just the number ten itself.

How about when $k=0$?
Examining the pattern of how the power $k$ relates to how many zeros follow the ``1"
(eg, $10^1=10$, $10^2=100$, $10^3=1000$,
etc.) then it must be the case that $10^0=1$,
i.e., no zeros follow the ``1", which is exactly right.
Furthermore every number raised to the 0\textsuperscript{th}
power~is~1.\footnote{Proof: Since $a^{b+c}=a^ba^c$ consider when $c=0$; that is,
$a^b=a^{b+0}=a^ba^0$ so because of the uniqueness
of the multiplicative identity ``1'', then $a^0$ \emph{must} be 1 since it's behaving 
like a ``1'' in the expression $a^b=a^ba^0$.}

Let's look at an example.
Reading the number 92507 out according to our technique we can see
that it's ``nine lots of ten-thousand,
plus two lots of a thousand, plus five lots of a hundred,
plus zero lots of ten, plus seven units":

\begin{center}
\begin{tabular}{r r r r c r r}
\phantom & 9 & $\times$ & 10000 & \phantom & \phantom & 90000\\
+ & 2 & $\times$ & 1000 & \phantom & + & 2000\\
+ & 5 & $\times$ & 100 & \; \; \; = \; \; \; & + & 500\\
+ & 0 & $\times$ & 10 & \phantom & + & 00\\
+ & 7 & $\times$ & 1 & \phantom & + & 7\\
\cline{6-7}
\phantom & \phantom & \phantom & \phantom & \phantom & = & 92507
\end{tabular}
\end{center}

Written\footnote{Recall the mnemonic ``bedmas'' for
the ``\href{https://en.wikipedia.org/wiki/Order_of_operations}{Order of Operations}'' in
evaluating an expression, which is no different from
what we did in our table above the expression.}
in terms of powers-of-ten:
$92507=9{\times{}}10^4+2{\times{}}10^3+5{\times{}}10^2+0{\times{}}10^1+7{\times{}}10^0$.

This way of breaking down the base-ten representation of a number into
an algebraic expression can be done for EVERY string of decimal digits. It's the key to 
understanding what a string of decimal digits means.

Recalling that $10^0 = 1$ you might wonder why we bother to multiply
$7{\times{}}10^0=7{\times{}}1=7$ since there is no actual effect
when multiplying by one.
Even though it's not necessary, including the $10^0$ in the
expression reveals a kind of mathematical symmetry.
Each successive digit is multiplied by an ever
decreasing power-of-ten, including
the units digit,
which is just some number from 0 to 9 times a power-of-ten
like any of the other digits.

Our example number 92507 only has five digits and it's biggest power-of-ten is $10^4$,
but there's
no limit on how big a power-of-ten could be involved in our expression.
Look at ``a trillion and one'', i.e.; 1,000,000,000,001 which can be expressed as:
\[1{\times{}}10^{12}
+0{\times{}}10^{11}
+0{\times{}}10^{10}
+\cdots{}
+0{\times{}}10^{2}
+0{\times{}}10^{1}
+1{\times{}}10^{0}\]
%
Or pushing that limit to silly heights we can also describe this next ludicrous number.
It's twenty-thousand-and-one digits long, a
``7'' followed by 9999 zeros, then a ``3''
followed by 9999 more zeros, then a ``5'', which means this:
\[7{\times{}}10^{20000}
+0{\times{}}10^{19999}
+\cdots{}
+0{\times{}}10^{10001}
+3{\times{}}10^{10000}
+0{\times{}}10^{9999}
+\cdots{}
+0{\times{}}10^{1}
+5{\times{}}10^{0}\]
Clearly we can keep going as high as we like.

But how do we \emph{know} that we can create ALL the positive
integers with this scheme?
How do we \emph{know} that we didn't miss one?
How do we \emph{know} that some string of
digits doesn't represent two different integers?
I know it seems silly to ask these kinds of questions - 
after all, people have been counting in base-ten for almost two thousand years,
if there was a problem, you'd think we'd have heard about it by now! ...so, obviously it works.

Here's the thing about modern mathematics - the \emph{only} ideas we take as obvious
and don't require deeper explanation
are the axioms - those are the mathematical ideas that
are so simple that they can't be expressed in
yet other even-simpler ideas. The axioms are the minimal set of simple,
obvious, irrefutable ideas from which
everything else in mathematics is built\footnote{The axioms:
For every integer $a,b\text{ and }c$:
Associativity: $(a+b)+c=a+(b+c)$ and $a(bc)=(ab)c$;
Commutativity: $a+b=b+a$ and $ab=ba$;
Distributive: $a(b+c)=(b+c)a=ab+ac$;
Identities: There are integers 0 and 1 such that,
$a+0=0+a=a$ and $a\cdot{}1=1\cdot{}a=a$ and
Additive Inverse: $a+(-a)=0$.
Note: in general integers do NOT have multiplicative inverses 
that are also integers. (eg. $\frac{1}{2}$ is the multiplicative
inverse of 2 because $\frac{1}{2}\cdot{}2=1$ but $\frac{1}{2}$ is not an integer!)}.

As obvious as it seems, the fact that
we can use base-ten to represent the integers is NOT among the list of axioms.

However because base-ten numbers are in
a perfect one-to-one correspondence with the integers then
it's safe to think of each
integer's unique base-ten label \emph{as} the integer.
But to firmly establish this fact we need to (at least once) take the time to
spell out exactly what it means to represent a number in base-ten with a theorem\footnote{I can't find where
the first such theorem was stated, I think it's quite recent.}. Then that theorem must be proven to
be true with a series of arguments that logically connects it
directly\footnote{directly \dots{}or indirectly
via other previously proven theorems.} to the axioms, so that the only way
that the theorem could be false is if the axioms themselves are false.

I could just state the ``Base-Ten Representation Theorem'' right here, but
how to get to its clear, simple and succinct statement is an interesting journey.
The rest of the paper is that journey. In fact we're going to end up generalizing upon our
intuitive understanding about
counting with base-ten and come up with a
better theorem that covers base-ten and more. At the very least this new understanding
will help you to ``get'' the cryptic joke about binary on page one!

\break
We intuitively know that counting with
base-ten covers all the positive integers.
For example, the odometer in your car that keeps churning out
new numbers for each mile you drive, starting from zero when it rolls
off the production line.  If your odometer was long enough that it 
stretched off past the horizon on your left, there's no limit on
how many miles you could count.

Our intuition is good - let's write it down
in our theorem. We might say:

\emph{Base-Ten Representation Theorem (initial draft)}
\begin{jprIn}
Every integer has a representation in base-ten.
\end{jprIn}

\bigskip
Something else we know intuitively is that each number written
in base-ten represents only ONE integer.
It almost feels silly to spell it out, but if we were to count out
4 piles of 100-apples-each-pile, then 9 piles of 10-apples-each-pile,
then count out 9 additional apples, \emph{then}
scoop them all into a big pile that
we'd always get the exact same size big-pile-of-apples.

It goes the other way too. If we were handed the aforementioned
big-pile-of-apples we could start counting out
piles of 100. We'd try to make as many piles of 100 as we could,
and we'd find that we'd have 4 piles of 100 before
we couldn't make another such pile.
Then we would start counting out piles of 10 with the remaining apples.
After we made as many piles of ten as we could out of those remaining apples,
we would discover that we'd have 9 such piles of ten
with 9 apples left over, in other words 499 apples! There is NO other
way to divvy up this big-pile-of-apples
if we follow this procedure.  In other words, each integer is
represented by only ONE base-ten number.

Let's strengthen our theorem based on the last two observations.

\emph{Base-Ten Representation Theorem (second draft)}
\begin{jprIn}
Every integer has a \emph{unique} representation in base-ten.
\end{jprIn}

\bigskip
We can also restrict our theorem to positive integers, becauase
we know that for every positive integer
there exists a unique negative integer
(such that if you add them together you get zero).
So if we can prove the
Base-Ten Representation Theorem for the positive integers then we
can easily extend it to the negative integers (and zero) with a corollary.
Our corollary could say something like ``Slap a `-' (minus)
sign in front of the positive base-ten representation to 
get a unique representation for
it's negative counterpart'' \dots{}or some such words.

Also, moving forward it would be helpful to have a name for our 
positive integer so that we can refer to it directly - how about $n$ for ``number'':

\emph{Base-Ten Representation Theorem (third draft)}
\begin{jprIn}
Every \emph{positive} integer $n$ has a unique representation in base-ten.
\end{jprIn}

At the moment it's not very helpful to have named $n$ (the theorem as it stands
doesn't say anything more about $n$ so why did we bother naming it?) but
as we flush out the remaining details of the theorem
we can refer to $n$ which carries the important information that it could be ANY positive integer.

Earlier we looked at the number 92507 by adding up
each digit times a power-of-ten\footnote{It's time
to replace our ``$\times{}$'' symbol for multiplication, with ``$\cdot{}$''
because ``$\times{}$'' might get confused for an ``$x$'' in an expression,
whereas ``$\cdot{}$'' never will be. Eg. $x\times{}2$
vs. $x\cdot{}2$, additionally ending up with something that is more aesthetically pleasing.
You may also see the ``$\cdot$'' omitted entirely as
in $ab$ - which means $a\cdot{}b$ as you have seen in earlier footnotes.}:
\[92507=9{\cdot}10^4+2{\cdot}10^3+5{\cdot}10^2+0{\cdot}10^1+7{\cdot}10^0\]
\emph{Every} base-ten number implicitly describes an algebraic expression like this, so
let's come up
with a general expression of this form that can describe ANY positive integer $n$.

Let's replace one of the digits in our example number 92507 with $d$, how about the 5 like this $92d07$. What I
mean becomes clear if I write it out:
\[n=92d07=9{\cdot}10^4+2{\cdot}10^3+d{\cdot}10^2+0{\cdot}10^1+7{\cdot}10^0\]
So $n$ is one of the following numbers:
92007,
92107,
92207,
92307,
92407,
92507,
92607,
92707,
92807 or
92907.

As you can see, the digit $d$ must be an integer between 0 and 9 inclusive 
which we can write as ``$0\le{}d\le{}9$'' however I suggest that ``$0\le{}d<10$" is better\footnote{Read 
$0\le{}d\le{}9$ as ``zero is less-than-or-equal-to dee which is less-than-or-equal-to nine"
and $0\le{}d<10$ as ``zero is less-than-or-equal-to dee which is (strictly) less-than ten".}!
It's logically equivalent to ``$0\le{}d\le{}9$''  but conveys more important information
to the reader. Why even talk about nine when the theorem is about base TEN?
\begin{align*}
n =\ &9{\cdot}10^4+2{\cdot}10^3+d{\cdot}10^2+0{\cdot}10^1+7{\cdot}10^0,\text{ where}\\
&d\text{ is an integer such that }0\le{}d<10
\end{align*}
This statement for $n$ only represents the integers 92007, 92107, \dots{} or 92907, so let's come up
with a statement for $n$ that will allow us to generate ANY five-digit number from 10000 
the way up to 99999 (which is a complete list of all the five-digit numbers).

In order to describe this general five-digit number, we need
five different `$d$'s, one for each of the five digits.
In other words, we need to associate a different term $d$ with each of the powers
$10^4$, $10^3$, $10^2$, $10^1$ and $10^0$.

Mathematics has a convention for coming up with
a list of terms for situations just like this -
we tack a subscript onto the name like so: $d_2$ which
you read as ``dee-two''\footnote{\dots{}yes like artoo-detoo, which perhaps
George should have writen as ``$R_2D_2$'' and not ``R2-D2''!}.
$d_2$ is a term to represent a digit just like the $d$ we used above.
But now we can use that little subscript as a way to
associate it to a specific power-of-ten.
Naturally we'll associate $d_2$ with $10^2$ (ten squared) as follows:
\begin{align*}
n =\ &9{\cdot}10^4+2{\cdot}10^3+d_2{\cdot}10^2+0{\cdot}10^1+7{\cdot}10^0,\text{ where}\\
&d_2\text{ is an integer such that }0\le{}d_2<10
\end{align*}
If we define $d_2$ like this, then we know that when we refer to the digit $d_2$
that we are talking about the digit that is multiplied with $10^2$.

\break
Let our five-digit-number $n$ use $d_0,d_1,d_2,d_3$ and $d_4$ for its digits. Then
the general expression for $n$ looks like this\footnote{That expression looks like
hard-core math, so let's take a moment
to read it out loud, as a Math-Professor might do in a lecture.  She might say:
``$n$ is equal to dee-four times ten-to-the-fourth, \dots{}
plus dee-three times ten-cubed, \dots{} plus dee-two times ten-squared,
  \dots{} plus dee-one times ten,  \dots{} plus dee-zero times one.''}:
\begin{align*}
n =\ &d_4{\cdot}10^4+d_3{\cdot}10^3+d_2{\cdot}10^2+d_1{\cdot}10^1+d_0{\cdot}10^0\text{, where}\\
& d_0\text{ is an integer such that }0\le{}d_0<10\text{, and}\\
& d_1\text{ is an integer such that }0\le{}d_1<10\text{, and}\\
& d_2\text{ is an integer such that }0\le{}d_2<10\text{, and}\\
& d_3\text{ is an integer such that }0\le{}d_3<10\text{, and}\\
& d_4\text{ is an integer such that }1\le{}d_4<10.
\end{align*}
Ok, hold on a minute - that's getting a little cumbersome. it's clunky and hard to read -
plus did  you notice how we slipped in that different range for $d_4$?

First let's deal with the different range on that $d_4$. To make sure $n$ is a legitimate five-digit number we have to
call out the exception that $d_4$ can NOT be zero - it has to be at least 1.
Why? Because if $d_4$ were zero then $n$ would only be a
four-digit number, or perhaps a three-digit number, or only two-digits etc.

Secondly, to clean up the presentation
a common convention
is to let another term like $i$, for perhaps ``index'', stand in for the subscript
when you want to talk about all your `$d$'s at once:
\begin{jprIn}
\begin{jprIn}
\begin{jprIn}
Let $d_0, d_1, d_2,d_3$ and $d_4$ be integers such that:

\hspace{3em}$n
=d_4{\cdot}{1\hspace{-0.12em}0}^4
+d_3{\cdot}{1\hspace{-0.12em}0}^3
+d_2{\cdot}{1\hspace{-0.12em}0}^2
+d_1{\cdot}{1\hspace{-0.12em}0}^1
+d_0{\cdot}{1\hspace{-0.12em}0}^0$

where $0\le{}d_i<10$ for all $i$ in $\{0,1,2,3,4\}$ and $d_4\ne0$.
\end{jprIn}
\end{jprIn}
\end{jprIn}
That's it! Those statements, and the expression for $n$ describe EVERY five-digit number.

Now let's extend
our five-digit expression for $n$ to an arbitrary
number of digits.
Consider the following progression:
\begin{center}
\begin{tabular}{r l}
expression for $n$ & number-of-digits\\
$d_0{\cdot}10^0$ & 1\\
$d_1{\cdot}10^1+d_0{\cdot}10^0$ & 2\\
$d_2{\cdot}10^2+d_1{\cdot}10^1+d_0{\cdot}10^0$ & 3\\
$d_3{\cdot}10^3+d_2{\cdot}10^2+d_1{\cdot}10^1+d_0{\cdot}10^0$ & 4\\
$d_4{\cdot}10^4+d_3{\cdot}10^3+d_2{\cdot}10^2+d_1{\cdot}10^1+d_0{\cdot}10^0$ & 5\\
\dots{} & \dots{}\\
$d_k{\cdot}10^k+\dots{}+d_4{\cdot}10^4+d_3{\cdot}10^3+d_2{\cdot}10^2+d_1{\cdot}10^1+d_0{\cdot}10^0$ & k+1
\end{tabular}
\end{center}
Using $k$ like this let's us specify any number of digits we want.
If we let $k=0$ we get the first ``single digit'' 
item on the list.  $k=4$ gives us our five-digit number above,
or we could let $k$ be a billion, which would allow
us to specifcy an integer that has a billion-and-one
digits in it\footnote{A billion-and-one digit number is 
\emph{ridiculously} large,
consider that our estimate of the number of molecules in
the entire universe would only need a base-ten
number with the $k$ set to somewhere between 78 and 82 to count them all.}!

\break
So there we have it, we found our expression for being able to
express each positive integer, let's use
it in a revised draft of our theorem:

\emph{Base-Ten Representation Theorem (close to final draft)}
\begin{jprIn}
For every positive integer $n$ there is a unique sequence
of integers $d_0, d_1, d_2,\dots{},d_k$ such that:

\hspace{3em}$n=d_k{\cdot}{1\hspace{-0.12em}0}^k+d_{k-1}{\cdot}{1\hspace{-0.12em}0}^{k-1}+\dots+d_2{\cdot}{1\hspace{-0.12em}0}^2+d_1{\cdot}{1\hspace{-0.12em}0}^1+d_0{\cdot}{1\hspace{-0.12em}0}^0$

where $0\le{}d_i<10$ for all $i$ in $\{0,1,2,\dots{},k\}$ and $d_k\ne0$.

Definition: $n$ is represented in base-ten by the string of digits $d_kd_{k-1}{\cdots}d_2d_1d_0$
\end{jprIn}

Our newly added ``Definition'' introduces exactly what it means to write
the number out in base-ten; that is,
we toss out all the extraneous stuff from our expression
and string all the digits one after another.
Starting at the most-significant digit $d_k$ on the left,
down to the next digit to its right which is $d_{k-1}$
(read as ``dee-kay-minus-one''\footnote{\dots{}and $d_{k-1}$ is
multiplied by ``ten-to-the-power-of-(kay-minus-one)''.})
all the way down to the least-significant units-digit $d_0$ on the right.

We are so darn close, but there is one super-picky detail
that we should be concerned about.
Our theorem establishes what it means to represent a number in base-ten, so until
we've proven it, how can we actually use the first two-digit number ``10" to represent
the integer ten!? We need a symbol for ten in the theorem,
so what can we do?

Apart from ``10" we don't have a symbol for the integer ten so
we have to make one up, how about $T$ for ``Ten'':
\begin{jprIn}
Let $T$ represent the integer ten.

For every positive integer $n$ there is a unique sequence
of integers $d_0, d_1, d_2,\dots{},d_k$ such that:

\hspace{3em}$n=d_k{\cdot}{T}^k+d_{k-1}{\cdot}{T}^{k-1}+\dots+d_2{\cdot}{T}^2+d_1{\cdot}{T}^1+d_0{\cdot}{T}^0$\\
\dots{} etc.
\end{jprIn}
That's a little confusing, so 
let's solve our problem by defining the 
two digit number ``10" (i.e., a one followed by a zero) to be the integer ten.
It's ok - we're not violating any rules by doing this. At this point we're just defining
a symbol to stand in for the integer ten. Here's our
FINAL draft of the theorem:
\section*{Base-Ten Representation Theorem}
\begin{jprIn}
Let the two digit number ``${1\hspace{-0.12em}0}$" represent the integer ten.

For every positive integer $n$ there is a unique
sequence of integers $d_0, d_1, d_2,\dots{},d_k$ such that:

\hspace{3em}$n
=d_k{\cdot}{1\hspace{-0.12em}0}^k
+d_{k-1}{\cdot}{1\hspace{-0.12em}0}^{k-1}
+\dots
+d_2{\cdot}{1\hspace{-0.12em}0}^2
+d_1{\cdot}{1\hspace{-0.12em}0}^1
+d_0{\cdot}{1\hspace{-0.12em}0}^0$

where $0\le{}d_i<{1\hspace{-0.12em}0}$ for all $i$ in $\{0,1,2,\dots{},k\}$ and $d_k\ne0$.

Definition: $n$ is represented in base-ten by the string of digits $d_kd_{k-1}{\cdots}d_2d_1d_0$
\end{jprIn}
\bigskip

\break
Consider that base-ten is not the only base in use these days.
Since the introduction of the EDVAC\footnote{You might be thinking, don't you
mean ENIAC which was earlier? Actually no - the ENIAC
used base-ten accumulators, not binary!} computer, around 1950,
there have been many orders of magnitude more calculations done
in base-two (otherwise known as binary) by computers than have EVER
been done by people in base-ten for the entirety of human history.
(This might even be true if we only count one-day's worth of
binary computer calculations - someone needs to check this.)

Binary-computer logic gates (the building blocks of the modern computer)
can only take one of two states, that is; ``off" or ``on".
We interpret these two states to represent these two numbers: 0 and 1.
By doing so, in the same way that base-ten uses ten numbers 0,
1, 2, 3, 4, 5, 6, 7, 8, and 9 for its digits; we can represent integers
in base-two with just the digits 0 and 1. How is this possible?
Let's find out with an imaginary trip into space.

Consider distant Planet-Nova on which the emergent
intelligent species only have nine fingers on their hands.
They have three hands with three fingers each - anyway,
that's why they use base-nine, so they only need the numbers 0,
1, 2, 3, 4, 5, 6, 7 and 8 for their digits\footnote{Digit
is another word for finger! Of course that's where the math term got its start.}.
So like we Earthlings do for the integer ten,
instead of making up a new symbol for nine,
they use ``10" to represent the integer nine - which
for them means ``One lot of nine, plus zero units".

Similarly on Planet-Ocho, since they only have eight fingers,
then they use base-eight and only use numbers 0, 1,
2, 3, 4, 5, 6 and 7 for their digits. For them ``10''
means ``One lot of eight, plus zero units".

On and on past Planet-Gary-Seven, and Planet-Secks, Planet-Penta, \dots{}

Finally we come upon Planet-Claire (well someone
has to come from Planet-Claire,
I know she came from there),
where the poor blighters only have two fingers
so they only use the digits 0 and 1 and base-two,
so for them ``10" means ``one lot of two and zero units".
So on Planet-Claire ``10" means two.
Recall above how we arrived at our 100 in base-ten,
being ``ten lots of ten,
plus zero units" - similarly on Planet-Claire ``100''
in base-two for them means ``Two lots of two plus zero units" in other words,
four! What is ``11" in base-two? Using our technique to
describe the digits we see that it's ``One lot of two, plus one unit",
in other words three.

Here's how they count on Planet-Claire using base-two:
\begin{center}
\begin{tabular}{r l c r l}
base-two & base-ten & \; \; \; \; & base-two & base-ten\\
0 & 0 & \phantom& (\dots cont)\\
1 & 1 & \phantom& 1101 & 13\\
10 & 2 & \phantom& 1110 & 14\\
11 & 3 & \phantom& 1111 & 15\\
100 & 4 & \phantom& 10000 & 16\\
101 & 5 & \phantom& 10001 & 17\\
110 & 6 & \phantom& \dots{}\\
111 & 7 & \phantom& 11111 & 31\\
1000 & 8 & \phantom& 100000 & 32\\
1001 & 9 & \phantom& \dots{}\\
1010 & 10 & \phantom & 1000000 & 64\\
1011 & 11 & \phantom& 10000000 & 128\\
1100 & 12 (cont\dots) & \phantom& 100000000 & 256
\end{tabular}
\end{center}
\break
Note something interesting in the list above - the powers of two,
written in base-two,
resemble our powers of 10 in base-ten! That is:
\begin{align*}
1 = 2^0&= 1, & 32 = 2^5&= 100000_{(\text{base-2})},\\
2 = 2^1&= 10_{(\text{base-2})}, & 64 = 2^6&= 1000000_{(\text{base-2})},\\
4 = 2^2&= 100_{(\text{base-2})}, & 128 = 2^7&= 10000000_{(\text{base-2})},\\
8 = 2^3&= 1000_{(\text{base-2})},& 256 = 2^8&= 100000000_{(\text{base-2})},\\
16 = 2^4&= 10000_{(\text{base-2})},& & \dots{}
\end{align*}
Let's look at the binary number 11010 for example.
Using our wordy technique to describe the number
we can see that it's ``One lot of sixteen,
plus one lot of eight, plus zero lots of four,
plus one lot of two, plus zero units": 
\begin{center}
\begin{tabular}{r r r r c r r l}
\phantom & 1 & $\times$ & 10000 & \phantom & \phantom & 10000 & (16)\\
+ & 1 & $\times$ & 1000 & \phantom & + & 1000 & (8)\\
+ & 0 & $\times$ & 100 & \; \; = \; \; & + & 000 & \\
+ & 1 & $\times$ & 10 & \phantom & + & 10 & (2)\\
+ & 0 & $\times$ & 1 & \phantom & + & 0\\
\cline{6-8}
\phantom & \phantom & \phantom & \phantom & \phantom & = & 11010 & (26)\\
\end{tabular}
\end{center}
Written in terms of powers of two:
$11010_{(\text{base-2})}=26=1\cdot2^4+1\cdot2^3+0\cdot2^2+1\cdot2^1+0\cdot2^0$.

Does that expression look familiar? It has exactly the same form
as the expression for our five-digit base-ten number 92507.
All the reasoning we used to come up with the statement of the ``Base-Ten
Representation Theorem" can be used again, but swapping powers-of-two
for powers-of-ten, and limiting the values for the digits
to be zero or one.
Following our line of reasoning this is what the Planet-Claire mathematicians would have
come up with:

\section*{Base-Two Representation Theorem}
\begin{jprIn}
For every positive integer $n$ there is a unique
sequence of integers $d_0, d_1, d_2,\dots{},d_k$ such that:

\hspace{3em}$n=d_k2^k+d_{k-1}2^{k-1}+\dots+d_22^2+d_12^1+d_02^0$,

where $0\le{}d_i<2$ for all $i$ in $\{0,1,2,\dots{},k\}$ and $d_k\ne0$.

Definition: $n$ is represented in base-two by the string
of binary-digits $(d_kd_{k-1}{\cdots}d_2d_1d_0)_2$
\end{jprIn}
\bigskip

Our new Base-Two Representation Theorem introduced some helpful new notation.
How do you know what I'm talking about if I just
write ``1000"? Do I mean $10^3$ or $2^3$?
If there is any possibility for confusion we write
the number like this $(1000)_{10}$ 
for the base-ten version meaning one-thousand and $(1000)_2$
for the binary version meaning eight.
That's what the ``Definition'' is spelling out with the ``$(\dots)_2$'' extra notation.

As is hinted by the habits of our various alien friends
above it seems that we can use ANY integer greater than
or equal to 2 as a base (base-one doesn't really make
sense - think about it for a while).
In fact computer graphics artists are known
to stumble upon numbers written in hexadecimal (usually relating to specifying a color-channel),
which is base-sixteen.

Base-sixteen introduces some new single-character symbols to the usual numbers 0, 1,
2, thru 9,
to represent the numbers 10, 11, 12, 13, 14 and 15.
Base-sixteen adds the digits A, B, C, D, E and F where
A\textsubscript{16}=(10)\textsubscript{10},
B\textsubscript{16}=(11)\textsubscript{10},
C\textsubscript{16}=(12)\textsubscript{10},
D\textsubscript{16}=(13)\textsubscript{10},
E\textsubscript{16}=(14)\textsubscript{10},
F\textsubscript{16}=(15)\textsubscript{10}.
So (80FB)\textsubscript{16} is a four digit number in base-sixteen.
(As we'll see shortly it means (33019)\textsubscript{10} in base-ten).

Note that if we omit the parentheses and subscript from a number,
it means we're talking about it in base-ten; our ``default" base.
Case in point: the subscripts that we use to denote the base
(like the ``16" in (80FB)\textsubscript{16}) are written in base-ten!

We could go ahead and prove our ``Base-Ten'' and ``Base-Two'' theorems above,
but what about proving the ``Base-Nine" version of the theorem for the aliens on Planet-Nova,
or the ``Base-Eight" version for the inhabitants of Planet-Ocho?

To cover all bases (pun intended) let's restate our theorem for the general case,
call it ``base-$b$",
where $b$ is some number greater than or equal to two.
If we can prove that theorem,
then we'll automatically get all the cases of specific bases for free.

\section*{Basis Representation Theorem}
\begin{jprIn}
Let $b$ be a positive integer greater than 1.

For every positive integer $n$ there is a unique sequence
of integers $d_0, d_1, d_2,\dots{},d_k$ such that:

\hspace{3em}$n=d_kb^k+d_{k-1}b^{k-1}+\dots+d_2b^2+d_1b^1+d_0b^0$,

where $0\le{}d_i<b$ for all $i$ in $\{0,1,2,\dots{},k\}$ and $d_k\ne0$.

Definition: $n$ is represented in base-$b$ by the string
of base-$b$-digits $(d_kd_{k-1}{\cdots}d_2d_1d_0)_b$
\end{jprIn}
\bigskip

So to get the ``Base-Ten Representation Theorem'' let $b$ equal ten.
To get the ``Base-Two Representation Theorem'' let $b=2$; or the ``Base-Nine Representation Theorem''
let $b=9$; etc.

\bigskip
The Basis Representation Theorem
implies that we can safely convert between
different bases. Why? (Exercise left for student).

Recall how we defined $(\text{A})_{16}=10$
and $(\text{F})_{16}=15$ as base-sixteen digits, then:
\[
(\text{97A3F2})_{16}
=9\cdot{}16^5
+7\cdot{}16^4
+10\cdot{}16^3
+3\cdot{}16^2
+15\cdot{}16^1
+2\cdot{}16^0
=9,937,906\]
\dots{}which gives you an idea of how you can convert from an alternate base into base-ten.

We should extend the Basis Representation Theorem to include all the integers.

We can take comfort in knowing that this shouldn't be too difficult.
0 itself is a perfectly acceptable representation for zero in all bases.
Actually, we could have included zero as part of the original theorem if
we had just allowed for the exception: ``\dots{}and $d_k\ne0$ except when $n=0$",
but at this point we're just talking about book-keeping type details.

Also, as we discussed earlier, since negative numbers are unique (they pair one-to-one with
their positive counterpart) then by simply adding a new symbol like ``-'', and 
defining what ``-1'' means, we should be able to make an extension to the 
theorem that defines unique representations for negative numbers fairly easily.

\break
\section*{Intermission}
At this point we really ought to get to the proof of the Basis Representation Theorem.

However truth be told, the proof supplied below, while elementary, isn't the most beautiful proof 
in the world.
In fact here's what one of the
world's greatest mathematicians\footnote{along with the Gauss
and Euclid mentioned later in this paper.},
G.~H.~Hardy, had to say about such
proofs\footnote{in an essay he wrote called ``A Mathematician's Apology''.}:

\begin{jprIn}
``We do not want many `variations' in the proof of a mathematical
theorem: `enumeration of cases', indeed, is one
of the duller forms of mathematical argument. A mathematical proof
should resemble a simple and clear-cut constellation, not a scattered
cluster in the Milky Way.''
\end{jprIn}

The proof of the Basis Representation Theorem supplied in this paper is of the duller type described by Hardy.  
I couldn't agree more with him so I apologize in advance for the proof's lack of elegance.
I still invite you to to take a stab at following it,
I tried my best to make it clear and interesting.
If you do try to follow it (it's at the end of this paper) - it does
have some points of interest - not the least
of which is that it works, some of the algebraic manipulations are amusing,
but most importantly you'll learn about a very powerful tool
called ``The Principle of Mathematical Induction''. 

Better than diving into the proof right away, please try
your hand at a couple of exercises for fun! If you get stuck, or to
check your work, the
answers are also supplied below - but please don't peek until you try the questions yourself.

If you feel bewildered when facing the exercises below, know that you are in good
company - this is a common feeling among mathematicians, you'll get used to it!
But like any ``workout'' if you push through you will succeed and get stronger, I promise.

\section*{Exercises}
\begin{enumerate}
\item What are the following numbers expressed in base-ten?
\begin{enumerate}[i)]
\item $(110101)_2$
\item $(\text{A}053\text{D})_{16}$
\item $(1017)_{23}$
\end{enumerate}
\item What are the following base-ten numbers expressed in an alternate base?
\begin{enumerate}[i)]
\item 33 expressed in base-two?
\item 127 expressed in base-two? (Hint: $127 = (128-1)$)
\item 8079 expressed in base-sixteen?

Hint: For a moment, pretend that we don't use base-ten to
write out our numbers, instead picture a pile of apples.
Can you picture 7654 apples?  Yes?  Good let's use 7654 as our example.

Let's divide 7654 by 10 so we get the following:
\[7654 = 765\cdot{}10+4\]
Notice the remainder 4 is the least significant digit of our
integer 7654  (i.e. the $d_0$ digit in the theorem).

How do we get the next digit, i.e. the $d_1$ digit that corresponds to the $10^1$ term? Well, it’s kind of cheating, but
since we happen to be looking at that last expression written in
base-ten we can see it sitting right there in at the
end of the quotient ``765''. So, let's use the same technique and divide 765 by 10:
\[765 = 76\cdot{}10+5\]
So the remainder is 5 our $d_1$ digit.  Let's keep going, this time dividing the previous quotient 76 by 10.\dots	
\[76 = 7\cdot{}10+6\]
and finally,
\[7= 0\cdot{}10+7\]
So, our series of remainders happens to be the digits of the number in base-ten.
Specifically $d_3 = 7$, $d_2=6$, $d_1=5$ and $d_0=4$.

Try doing that for 8079, but use 16 instead of 10 as the divisor.
\item Let $\text{A}_{23} = 10, \text{B}_{23} = 11, \text{C}_{23} = 12, \text{D}_{23} = 13, \text{E}_{23} = 14, \text{F}_{23} = 15,
\text{G}_{23} = 16,\\
\text{H}_{23} = 17, \text{I}_{23} = 18, \text{J}_{23} = 19, \text{K}_{23} = 20, \text{L}_{23} = 21$ and $\text{M}_{23} = 22$,\\
then what is 185190 expressed in base-twenty-three?
\item 291480 expressed in base-twenty-three?
\end{enumerate}

\end{enumerate}

\section*{Answers}
\begin{enumerate}
\item What are the following numbers expressed in base-ten?
\begin{enumerate}[i)]
\item $(110101)_2 = 53$
\item $(\text{A}053\text{D})_{16}=656701$
\item $(1017)_{23} = 12197$
\end{enumerate}
\item What are the following base-ten numbers expressed in an alternate base?
\begin{enumerate}[i)]
\item $33 = (100001)_2$
\item $127 = (1111111)_2$
\item $8079 = (1\text{F}8\text{F})_{16}$
\item $185190 = (\text{F}51\text{H})_{23}$
\item $291480 = (10\text{M}01)_{23}$
\end{enumerate}
\end{enumerate}

\break
\section*{The Principle of Mathematical Induction}

As we discussed way up at the top of this paper,
we think about generating the set of positive integers as a process that builds them up one by one.
That is, each successive integer is one more than the previous one,
starting at 1, then one more
taking us to 2,
then 3, 4, 5, \dots{}ad~infinitum\footnote{``ad
infinitum'' means ``to infinity'', or ``continue forever, without limit''.}.

This idea of being able to step one after the other,
beginning at 1 and going forever is embodied within the principle of mathematical
induction and is a basic property of the positive integers.
This principle is more than just a way to generate the set of integers,
it's also a way of thinking about properties of the integers.

Suppose
that $P(n)$ means that the property $P$ holds
for the number $n$; where $n$ is a positive integer.
Then the principle of mathematical induction states that $P(n)$
is true for ALL positive integers $n$ provided that\footnote{This wording of the
definition of ``The Principle of Mathematical Induction'' is essentially borrowed
from ``Calculus'' by Michael Spivak - a fabulous introductory textbook on Analysis.}:

\begin{enumerate}[i)]
\item $P(1)$ is true
\item Whenever $P(k)$ is true, $P(k+1)$ is true.
\end{enumerate}

Why would these two conditions show that $P(n)$ is true for all
positive integers? Note that condition ii) only asserts the truth
of $P(k+1)$ under the assumption that $P(k)$ is true.
However if we also know that $P(1)$ is true then condition ii) implies that $P(2)$ is true,
which again implies that $P(3)$ is true,
which in turn leads to the truth of $P(4)$,
etc., over and over for all positive integers.

Some people picture an infinite row of dominoes.
Having condition i) (called the ``base case") is like being
able to knock over the first domino.
Then knowing condition ii) is also true (called the ``induction step") is like the
fact that any one domino has the ability to knock over the next.
Once you've knocked over the first domino,
they all fall.

Let's look at a simple example:
Perhaps you've heard the story of young Carl Friedrich Gauss
as a boy in
the 1780s who was assigned (along with all his classmates)
the tedious task of summing the first 100 integers -
presumably to keep them quiet and busy while the
teacher corrected some papers. Anyway,
young Gauss immediately produced the answer,
5050, before most of the boys had summed the first couple of numbers.

It wasn't young Gauss's extraordinary computational speed which allowed
him to perform this dazzling task,
but he had the deeper insight that instead of adding 1 plus 2,
then adding 3, then 4, etc.
he saw that if you paired 1 with 100,
and 2 with 99,
and 3 with 98,
etc.,
that each of those pairs added up to 101,
furthermore he knew he'd have 50 such pairs,
meaning he could state the result in a heartbeat - tada - ``5050"!
Gauss is widely regarded as being one of the greatest
mathematicians who has ever lived - the young eight-year old was just getting started.

\break
Anyway,
to generalize young Gauss's insight we can write the expression like this:
\[1+2+3+\ldots+n=\frac{n(n+1)}{2}\]
So let's prove this relationship using the principle of mathematical induction.
\bigskip

Let $n=1$ for the ``base case'',
then
\[\frac{1(1+1)}{2}=\frac{2}{2}=1\]
Which is the trivial sum\footnote{The word ``sum'' here is used
in the context of the expression
we are trying to prove. In this case we are summing
only one item thus it's ``trivial''.} of the first positive integer 1.
\bigskip

Now let's assume the relationship is true for $n$,
and prove that it must also be true for $n+1$ for our ``induction step":
\begin{align*}
&(1+2+3+\ldots+n) + (n+1)\\
= & \; \frac{n(n+1)}{2} + (n+1)\\
= & \; \frac{n(n+1)}{2} + \frac{2(n+1)}{2}\\
= & \; \frac{n^2+n+2n+2}{2}\\
= & \; \frac{n^2+3n+2}{2}\\
= & \; \frac{(n+1)(n+2)}{2}\\
= & \; \frac{(n+1)((n+1)+1)}{2}
\end{align*}
Which proves young Gauss's expression is true for
the positive integer $n+1$ whenever it's true for $n$ - then
by the principle of mathematical induction,
the expression is true for all positive integers. QED\footnote{``QED'' - is
often used at the conclusion of a proof to state that it's
done - it's an acronym for the Latin phrase
``quod erat demonstrandum'' which means ``that which was to be demonstrated''.
In other words we've proven what we set out to prove.}

\section*{Extra Exercise: Geometric Series Theorem}
If $b, n$ are nonnegative integers and $b\ne1$ then prove,
\[1+b+b^2+\dots+b^{n-1} = \frac{b^n-1}{b-1}\]
Hint: use induction on $n$, the base case being $n=1$.  Before you turn the page, try proving this, you can do it!

\section*{Proof for Extra Exercise: Geometric Series Theorem}
Base case: $n=1$
\[\frac{b^1-1}{b-1}=\frac{b-1}{b-1}=1=b^0=b^{1-1}\]

Induction step:
Assume the following
\[1+b+b^2+\dots+b^{n-1} = \frac{b^n-1}{b-1}\]
Then,
\begin{align*}
&(1+b+b^2+\dots+b^{n-1}) + b^n\\
=\ &\frac{b^n-1}{b-1}+b^n\\
=\ &\frac{b^n-1}{b-1}+\frac{b^n(b-1)}{b-1}\\
=\ &\frac{b^n-1+b^{n+1}-b^n}{b-1}\\
=\ &\frac{b^{n+1}+b^n-b^n-1}{b-1}\\
=\ &\frac{b^{n+1}-1}{b-1}
\end{align*}
QED
\section*{Proof of the Basis Representation Theorem}
How do we prove our theorem? There are several ways to approach it.

The mathematician George E. Andrews (in his book
``Number Theory") has an interesting proof.
He asks us to imagine a function that given a positive integer $n$,
counts the number of base-$b$ representations of $n$.
Then with some fairly straightforward reasoning he shows that this counting-function
MUST produce a count of ``1" for every $n$, both establishing the uniqueness 
and the existence in one fell swoop.
It's a cool proof - short and sweet, I'll bet G. H. Hardy would approve of
it from an aesthetic standpoint.

We could also use the Euclidean Division Theorem to prove our theorem.
We could show that by dividing our integer $n$ by $b$, then dividing that
result by $b$ again, then again, and again, etc. that we'd get a series
of remainders which are the base-$b$-digits of $n$.
This is an interesting approach to the proof in that it also gives us a technique
to construct a base-$b$ representation of each integer.
If we were to go down this road
we would try to generalize how we solved
exercise number 2-iii) above, it's worth a try for fun.

However, we're going to follow a much more straightforward approach.
We'll use the principle of mathematical induction directly on $n$
to prove that
a base-$b$ representation of each positive integer exists.
Then we'll use a different approach to prove that each such representation is unique.

\bigskip
Using the
principle of mathematical induction 
to prove that a base-$b$ number exists for all the positive integers is pretty simple.
The crux of the approach is to show that if $n$ can be expressed in base-$b$,
then it necessarily follows that $n+1$ can also be represented in base-$b$.

This is best thought of as a magic
odometer in your new magic car.
As you pull it out of the dealership parking lot
on the first day you notice its odometer is only one digit wide.
That one and only display-digit shows a ``1", I guess the factory was only one mile away.

As you get close to the end of your drive home
you see you've clocked 9 miles.
Naturally you're a little curious about how it's going to
count to 10 since the odometer apparently only has one digit for counting.
While pondering this curiosity you see the 9 starting to roll
over back to 0, and a second digit to it's left magically materializes!
It rolls to a 1 as the 9 settles back to 0.
Nice new feature - the unlimited odometer!
Apparently this magic odometer only adds new digits to the left
as it needs them.
The magic odometer should be able to display ANY number of miles, no matter how large.
(Good thing too, this
new line of magic cars from Tesla are guaranteed to last forever.)

Our proof is going to take care of the three cases that happen with
the magic-odometer. Imagine that we've already driven some arbitrarily large
number of miles, and we're watching the odometer roll over to the
next mile. We have the following cases:

\begin{enumerate}[i)]
\item The units digit is less than 9, so driving one more mile only increments
the units digit, not affecting any of the other digits.

Example: We've driven 782995 miles so far, so the next mile driven is
$782995 + 1 = 782996$.

So that units digit turned from a ``5" to a ``6", all the rest of
the digits remaining unchanged,
so the new number is a legitimate base-ten number.
\item The units digit, and perhaps a few directly to the left of it, are all 9's,
so driving one more mile will spin all those 9's to 0,
at the same time incrementing
the rightmost digit \emph{that isn't a} 9 by one.

Example: We've driven 782999 miles so far, so the next mile driven is:
\begin{align*}
782999+1 &= 780000+2999+1\\
&=780000+(3000-1)+1\\
&=780000+3000+(-1+1)\\
&=780000+3000+0\\
&=783000
\end{align*}
That looks overly complicated, but I broke it down
like that to demonstrate the idea behind the algebraic ``trick" that's
used in the proof.

Anyway, it's clear that the rightmost digit that-isn't-a-9, in our case the 2, got changed
to a 3 and all the 9's got changed to 0's, while the rest of the digits remained unchanged
leaving us with a legitimate base-ten number.
\item ALL the digits from the units digit up to the highest digit are 9's,
so driving one more mile
engages the magic odometer feature, materializing a new leftmost digit.
The newly materialized digit turns to a 1 and all the 9's spin back to 0.
In this case the number of digits on the odometer will have been extended by one.

Example: We've driven 999999 miles so far, so the next mile driven is:
\begin{align*}
999999+1 &=(1000000-1)+1\\
&=1000000+(-1+1)\\
&=1000000+0\\
&=1000000
\end{align*}
Same ``trick" as the previous case, but this time our number grew from a
six digit number to a seven digit number. Ya, we just drove a million miles
which is definitely a legitimate base-ten number!
\end{enumerate}

\bigskip
Even though we can use induction to prove the existence of base-$b$
numbers for every integer, let's think about why it
doesn't prove the ``uniqueness" aspect of 
the theorem.
Imagine a magic-box
that also makes base-$b$ representations for each
integer.
In other words, if the odometer is the principle-of-mathematical-induction then the magic-box
is some different mechanism working in a different way.
Why not? We weren't careful to show that induction is the ONLY way
to generate a base-$b$ representation for each number. (It isn't.)

So we need to prove that if such a magic-box exists then it MUST
produce the same results as our odometer.

The easiest way to prove this, is to assume that there is a magic-box
that DOES NOT produce the same representation as our odometer, then we show that
this assumption leads to an irreconcilable contradiction.
So either our assumption is wrong or the axioms are.
Since we're VERY confident in the axioms being correct we can only 
conclude that our assumption must be wrong - meaning
that there is only \emph{one} way 
to make a base-$b$ representation for each integer.

Without further adieu, let's get to the actual proof!

\break
\section*{Existence Proof of the Basis Representation Theorem}
Let $b$ be a positive integer greater than 1.

We will show that for every positive integer $n$ there is a sequence
of integers $d_0, d_1, d_2,\dots{},d_k$ such that:

\hspace{3em}$n=d_kb^k+d_{k-1}b^{k-1}+\dots+d_2b^2+d_1b^1+d_0b^0$,

where $0\le{}d_i<b$ for all $i$ in $\{0,1,2,\dots{},k\}$ and $d_k\ne0$.

\bigskip
We will refer to the previous expression as the
``base-$b$-representation" for $n$ in the following induction proof.

\bigskip
Base case:
\begin{jprIn}
Consider when $n=1$.

Let $d_0=1$ be the only integer in the sequence.  Then,

\hspace{3em}$d_0b^0=1\cdot{}b^0=1\cdot{}1=1=n$

Since $d_0<b$ and $d_0\ne0$ (note: $k=0$) then this
shows that a 
base-$b$-representation exists for the integer 1.
\end{jprIn}

\bigskip
Induction Step:
\begin{jprIn}
Let $n$ be a positive integer, and assume that
there is a sequence
of integers $d_0, d_1, d_2,\dots{},d_k$ such that:

\hspace{3em}$n=d_kb^k+d_{k-1}b^{k-1}+\dots+d_2b^2+d_1b^1+d_0b^0$,

where $0\le{}d_i<b$ for all $i$ in $\{0,1,2,\dots{},k\}$ and $d_k\ne0$.

We will prove that $n+1$ also has a
base-$b$-representation by looking at two cases.

\bigskip
Case 1) $d_0<(b-1)$
\begin{jprIn}
This case examines when the least significant
digit of $n$ is \emph{strictly-less-than} the largest value it can take in base-$b$.
(For example, in base-two $d_0 = 0$;
In base-five $d_0\le{}3$;
In base-ten $d_0\le{}8$.)\footnote{Please read the following bidirectional arrow symbol $\Leftrightarrow$ as ``if and only if'' - it's like a logical ``equals'' sign}
%\footnote{Use
%the axiom of Distribution \big(i.e.; $ac+bc=(a+b)c$\big) to factor out the $b^0$ below.}
{\small
\begin{alignat*}{3}
  &&n
  &= d_kb^k+d_{k-1}b^{k-1}+\dots+d_2b^2+d_1b^1+d_0b^0 &&\quad\text{(Induction Assumption)}\\
  &\Leftrightarrow\quad
  &n+1
  &= d_kb^k+d_{k-1}b^{k-1}+\dots+d_2b^2+d_1b^1+d_0b^0 + 1 &&\quad\text{(Add 1 to both sides)}\\  
  &&&= d_kb^k+d_{k-1}b^{k-1}+\dots+d_2b^2+d_1b^1+d_0b^0 + b^0 &&\quad\text{(Restate 1 as }b^0\text{)}\\
  &&&= d_kb^k+d_{k-1}b^{k-1}+\dots+d_2b^2+d_1b^1 + (d_0+1)b^0 &&\quad\text{(Axiom of Distribution)}
\end{alignat*}
}The expression for $n+1$ uses the same sequence of
integers $d_k, d_{k-1},\dots{},d_2,d_1$ as $n$, with only
change being that the integer $d_0$ was altered to $(d_0{+}1)$, but since:
\begin{alignat*}{3}
  &&d_0
  &< (b-1) &&\quad\text{(Case 1 Assumption)}\\
  &\Leftrightarrow\quad
  &d_0+1
  &< (b-1)+1 &&\quad\text{(Add 1 to both sides)}\\  
  &\Leftrightarrow\quad
  &d_0+1
  &< b
\end{alignat*}
Therefore, given the conditions of ``Case 1", $n+1$ has a base-$b$-representation whenever $n$ does.
\end{jprIn}

\bigskip
Case 2) $d_0=(b-1)$
\begin{jprIn}
Now we'll look at the case when the least significant digit of $n$ is \emph{exactly-equal-to}
the largest value it can take in base-$b$.
%(Note that  ``Case 1" and ``Case 2" cover all the possibilities for what $d_0$ can be.)
(For example in base-two
$d_0=1$; in base-five
$d_0=4$; in base-ten $d_0=9$.)

We're going to split this case into two cases.

\begin{enumerate}[a)]
\item \emph{All} the digits of $n$ are equal-to $b{-}1$ (eg. 999999 in base-ten).
\item $d_0=(b-1)$ but \emph{at least one other} digit is not-equal-to $(b-1)$\\
(eg. 782999 in base-ten).
\end{enumerate}
%
%Then we're going to find out what happens when we add 1 to each case, thus giving us our
%resultant expression for $n+1$

Case 2a) ALL of the digits of $n$ are equal to $(b-1)$.
\begin{jprIn}
So $n$ can be expressed like this:
\begin{alignat*}{3}
  &\quad&n
  &= (b{-}1)b^k+\dots{}+(b{-}1)b^2+(b{-}1)b^1+(b{-}1)b^0 &&\quad\text{(Induction Assumption)}
\end{alignat*}
Recall in the ``Extra Exercise: Geometric Series Theorem"
we showed that:
\[1+b+b^2+\dots{}+b^k=\frac{b^{k+1}-1}{b-1}\]

We can use this theorem to show that $n+1=b^{k+1}$ as follows:
\smallskip
{\small
\begin{alignat*}{3}
  &&b^k+\dots{}+b^2+b+1
  &= \frac{b^{k+1}-1}{b-1} &&\quad\text{(Reorder terms of G.S.Thm.)}\\
  &\Leftrightarrow\quad
  &b^k+\dots{}+b^2+b^1+b^0
  &= \frac{b^{k+1}-1}{b-1} &&\quad\text{(Rewrite with }b^1, b^0\text{)}\\
  &\Leftrightarrow\quad
  &(b-1)(b^k+\dots{}+b^2+b^1+b^0)
  &= b^{k+1}-1 &&\quad\text{(Mult both side by }(b{-}1)\text{)}\\
  &\Leftrightarrow\quad
  &(b{-}1)b^k+\dots{}+(b{-}1)b^1+(b{-}1)b^0
  &= b^{k+1}-1 &&\quad\text{(Distribution of }(b{-}1)\text{)}\\
  &\Leftrightarrow\quad
  &n
  &= b^{k+1}-1 &&\quad\text{(Substitute }n\text{ for expression)}\\
  &\Leftrightarrow\quad
  &n+1
  &= b^{k+1} &&\quad\text{(Add 1 to both sides)}
\end{alignat*}
}Let's rewrite $n+1$:
\[n+1= 1\cdot{}b^{k+1}+0\cdot{}b^{k}+\dots{}+0\cdot{}b^2+0\cdot{}b^1+0\cdot{}b^0,\]
Therefore, given the conditions of ``case 2a", $n+1$ has
a valid base-$b$-representation whenever $n$ does.
We note that
the number of integers in the sequence associated
with $n+1$ is one longer than for $n$. i.e., $n+1$ is one digit longer that $n$.

\break
\end{jprIn}
Case 2b) $d_0=(b-1)$ but at least one other digit is not equal to $(b-1)$
\begin{jprIn}
Let $j$ be the lowest power-of-$b$ such that $d_j<(b{-}1)$,
i.e.; we can write $n$ as follows:
\[n = d_kb^k+d_{k-1}b^{k-1}+\dots+d_jb^j+(b{-}1)b^{j-1}+\dots+(b{-}1)b^1+(b{-}1)b^0\]
Similar to how we used the ``Geometric Series Theorem'' above,
we can simplify the expression for $n$ as follows:
{\small
\begin{alignat*}{3}
  &&n
  &= d_kb^k+\dots+d_jb^j+\big((b{-}1)b^{j-1}+\dots+(b{-}1)b^1+(b{-}1)b^0\big)&&\\
  &\Leftrightarrow\quad
  &n
  &= d_kb^k+\dots+d_jb^j+(b^j-1) &&\text{(Geom. Series Thm.)}\\
  &\Leftrightarrow\quad
  &n+1
  &=  d_kb^k+\dots+d_jb^j+b^j &&\text{(Add 1 to both sides)}\\
  &\Leftrightarrow\quad
  &n+1
  &=  d_kb^k+\dots+(d_j+1)b^j &&\text{(Distr Axiom)}
\end{alignat*}
}Rewriting the expression for $n$ in explicit terms:
\[n+1= d_kb^k+\dots+d_{j+1}b^{j+1}+(d_j+1)b^j+0{\cdot{}}{}b^{j-1}+\dots{}+0{\cdot{}}b^1+0{\cdot{}}b^0\]
We can see that the expression for $n+1$ uses the same
sequence of integers $d_k,\dots{},d_{j+1}$ as $n$.
The integer $d_j$ was altered to $(d_j{+}1)$, but since:
\begin{alignat*}{3}
  &&d_j
  &< (b-1) &&\quad\text{(Previous assumption)}\\
  &\Leftrightarrow\quad
  &d_j+1
  &< (b-1)+1 &&\quad\text{(Add 1 to both sides)}\\  
  &\Leftrightarrow\quad
  &d_j+1
  &< b
\end{alignat*}
Furthermore the remaining sequence of integers $d_{j-1}=\dots{}=d_2=d_1=d_0=0$.

Therefore, given the conditions of ``case 2b", $n+1$ has
a valid base-$b$-representation whenever $n$ does.
\end{jprIn}
\end{jprIn}
Taking ``Case 1" and ``Case 2" together proves that $n+1$
always has a base-$b$-representation whenever $n$ does.
Having also established the base-case, therefore by the principle of mathematical induction
all positive integers have a base-$b$-representation.

QED
\end{jprIn}

\bigskip
In order to proceed
with proving the uniqueness aspect of the Basis Representation Theorem, we
need to make use of a well established theorem
called the ``Euclidean Division Theorem''.
It sounds onerous, but don't worry, you learned it
in the third grade but perhaps not so formally, you called it ``long division''. It simply states the following\dots{}

\break
\section*{Euclidean Division Theorem}
\begin{jprIn}
For all integers $a$ and $b$ such that $b>0$,
there exist \emph{unique} integers $q$ and $r$ such that\footnote{Aside:
Actually the theorem is stronger than we have stated here.
Specifically, it only requires that $b\ne0$, however
to keep the remainder positive, the restriction on $r$
would have to be stated like
this $0\le{}r<\left|b\right|$ to deal with
the possibility that $b$ might be negative.}:
\[a=qb+r  \text{ such that } 0\le{}r<b\]
Definition: In the above equation:
\begin{jprIn}
\begin{tabular}{l l}
a is the \emph{dividend} & (``the number being divided'')\\
b is the \emph{divisor} & (``the number doing the dividing'')\\
q is the \emph{quotient} & (``the result of the division'')\\
r is the \emph{remainder} & (``the leftover'')
\end{tabular}
\end{jprIn}
\end{jprIn}
This is how you first learned to divide.
For example if someone asks you ``What is nineteen divided by three?'', you’d
answer ``six with one remaining''. Here 19 is the \emph{dividend},  3 is the \emph{divisor},
6 is the \emph{quotient} and 1 is the \emph{remainder}. Written in the form of the theorem:
\[19=6\cdot3+1\]
Often proofs make use of little mini-theorems of their own.
Creating these mini-theorems is a way to simplify a step in
the main proof by establishing a useful
intermediary result. It makes reading the main proof
easier to follow by not having us get sidetracked with
the technicalities of a step we want to make.
These mini-theorems are called ``Lemmas''.
We're going to make a lemma to help with proving the uniqueness part
of the Basis Representation Theorem. But first we're going to make use of the
Euclidean Division Theorem to prove our lemma.

\section*{Lemma}
\begin{jprIn}
Let $b, q$ and $r$ be integers such that $b>0$ and $0\le{}r<b$, then:
\begin{center}
$0=qb+r$ 
\hspace{1.5em}if and only if\hspace{1.5em}
$q=0$ and $r=0$.
\end{center}
\end{jprIn}

\section*{Proof of Lemma}
\begin{jprIn}
Let $b, q$ and $r$ be integers such that $b>0$ and $0\le{}r<b$.

If $q=0$ and $r=0$, then \\
\hphantom{2em}$qb+r=0\cdot{}b+0=0$

\dots{}and if $0=qb+r$ then \\
\hphantom{2em}because the Euclidean Division Theorem says that $q$ and $r$ are unique, therefore\\
\hphantom{2em}$q=0$ and $r=0$ must be true otherwise they would not be unique.

QED
\end{jprIn}

\section*{Uniqueness Proof of the Basis Representation Theorem}

Let $b$ be a positive integer greater than 1.

By the ``Existence Proof of the Basis Representation Theorem'' we know
that for every positive integer $n$ there is a sequence
of integers $d_0, d_1, d_2,\dots{},d_k$ such that:
\begin{jprIn}
$n=d_kb^k+d_{k-1}b^{k-1}+\dots+d_2b^2+d_1b^1+d_0b^0$, where\\
\hphantom{2em}$0\le{}d_i<b$ for all $i$ in $\{0,1,2,\dots{},k\}$ and $d_k\ne0$.
\end{jprIn}

Assume this expression for $n$ is not unique and that there also exists
a different sequence
of integers $c_0, c_1, c_2,\dots{},c_k$ such that:
\begin{jprIn}
$n=c_kb^k+c_{k-1}b^{k-1}+\dots+c_2b^2+c_1b^1+c_0b^0$, where\\
\hphantom{2em}where $0\le{}c_i<b$ for all $i$ in $\{0,1,2,\dots{},k\}$ and $c_k\ne0$.
\end{jprIn}

Let's further suppose that $j$ is the lowest power such that the integers $d_j\ne{}c_j$ and without any loss of generality
let's assume that $d_j>c_j$. Since both expressions are equal to $n$ then:
\begin{center}
$c_kb^k+\dots+c_2b^2+c_1b^1+c_0b^0=d_kb^k+\dots+d_2b^2+d_1b^1+d_0b,$

\smallskip
if and only if,
\end{center}
{\small
\begin{alignat*}{3}
  &&0
  &= (d_k-c_k)b^k+\dots+(d_j-c_j)b^j&&\quad\text{(Subtract LHS from both sides)}\\
  &\Leftrightarrow\quad
  &\frac{0}{b^j}
  &= \frac{(d_k-c_k)b^k+\dots+(d_j-c_j)b^j}{b^j} &&\quad\text{(Divide by }b^j\text{, since }b>0\text{)}\\
  &\Leftrightarrow\quad
  &0
  &=(d_k-c_k)b^{k-j}+\dots+(d_{j+1}-c_{j+1})b+(d_j-c_j) &&\quad\text{(Divide each term in numerator by }b^j{)}\\
  &\Leftrightarrow\quad
  &0
  &= \big((d_k-c_k)b^{k-j-1}+\dots+(d_{j+1}-c_{j+1})\big)b+(d_j-c_j) &&\quad\text{(Factor out common }b{)}
\end{alignat*}
}

Let $q=\big((d_k-c_k)b^{k-j-1}+\dots+(d_{j+1}-c_{j+1})\big)$, then
\[0=qb+(d_j-c_j)\]
Since $0\le(d_j-c_j)<b$ and $b>0$ then by our lemma we know that
$q=0$ and $d_j-c_j = 0$.

But $d_j-c_j = 0$ if and only if $d_j = c_j$
contradicting our assumption that $d_j\ne{}c_j$. This implies that the initial assumption that ``$n$
is not unique'' is \emph{false}, in other words:
\begin{center}
The base-$b$ representation of $n$ is unique.\\
QED
\end{center}
%QED - Uniqueness Proof of the B.R.T.

\bigskip
Since we have proven that a base-$b$-representation exists for ALL the postive integers, \emph{and}
that this representation is unique, then we
have proven the Basis Representation Theorem.

\break
\section*{Epilogue}
It is my great hope that this paper, while on a very simple subject (so simple that we rightfully take
it for granted), tickled your fancy even just a little bit, and opened you up to some new pleasures.

I honestly don't know
if I've taken a fruitful approach to draw new people into the world of mathematics. My reasoning
is that taking such a familiar subject and applying rigorous mathematical reasoning,
that you could get familiar with the ``language'' of mathematics and get a taste for the
sharpness of the thought process. Most importantly I hope
you found some amusement in and among the mathematical tidbits sprinkled here 
and there.

I take great pleasure in math, even though I'm no mathematician (by any stretch of the imagination)
I don't let that stop me digging in and trying to digest new ideas as I stumble across them. I feel that
not being able to enjoy mathematics is as tragic as someone who is blind who can't enjoy the MoMA, or
someone who is deaf who can't while away the hours listening to their favorite band. This paper is
my attempt at trying to open up the world of mathematics to you so that you can also enjoy one of the greatest
achievements humanity has to offer.

%The world of mathematics is rich and varied, and filled with such fascinating characters (the mathematicians
%themselves) who by rights ought to be treated like rock stars.
%
If this paper has the opposite effect, then I apologize from the bottom of my heart, as I have
done you a great disservice.
\end{document}
